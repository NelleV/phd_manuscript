% this file is called up by thesis.tex
% content in this file will be fed into the main document

\chapter{Discussion} % top level followed by section, subsection

In this thesis, I have presented contributions to the analysis of Hi-C data,
in particular \textit{3D structure inference} methods. To summarize:

\textbf{Biological contributions}\ - I studied, in collaboration with Le Roch
lab and Noble lab, the three-dimensional structure of the human malaria
parasite \textit{P. falciparum}, which led to a better understanding of links
between the genome architecture and gene expression and regulation.

\textbf{Methodological contributions}\ - I focused on several methodological
projects,
first in the domain of \textit{3D structure inference}, with a statistical
method to infer a consensus model of the genome architecture, second in the
use of Hi-C for \textit{genome annotation}, with an approach to detect
centromeric regions for organisms whose genome fold in a Rabl configuration
(with centromeres colocalizing).

\textbf{Software contributions}\ - In addition to the methodological
contributions and the biological contributions, I have also focused on the
implementations of several methods studied or developped during the course of
this thesis. I believe that high quality implementations are critical for the
analysis of the huge quantity of data available in biology, while challenging.
I have not only contributed to \textit{scikit-learn}, a machine learning
toolkit written in Python (with the inclusion of
the isotonic regression, the metric and non-metric MDS, \dots), but also
released three packages, specific to analysis of Hi-C data:
\textit{iced}, \textit{pastis} and
\textit{centurion}, which are all free and open-source softwares.

\section*{Research perspectives}

The field of genome 3D structure is a young yet fast moving field. When I
first started to work on 3D structure inference methods, only a handfull of
papers using Hi-C where published. Nowadays, more than a paper per week on
this subject is published. Yet, as the field is still young, well grounded
methods and publicly available software are still lacking. I here describe
some research perspectives.


\textbf{Quality control and normalization of Hi-C data}\ -
Between the first publication on Hi-C \citep{lieberman-aiden:comprehensive}
and recent work such as \citet{rao:3d, jin:high-resolution}, the resolution of
the Hi-C contact maps have increased from 1~Mb to 5~kb or even 1~kb. Not only
has the number of reads greatly increase, but also the protocol improved,
assessing contact counts in a more robust fashion. Yet, there is still no
satistifying quality control protocol or quality measures to identify what is
a "good" dataset. In particular, the choice of the resolution of the contact
maps is still not justified by any well-grounded method and is left to the
judgement of the researcher. Same goes for the normalization: four published
methods attempt to remove biases, yet there is no clear measure on how well
these perform. Developping quality control measurement seems a natural first
step to help scientists set up and compare reliable protocols to assess the 3D
structure of the genome.

\textbf{Inference of the 3D structure of polyploid structures}\ - So far, most
methods either only dealt with haploid 3D structures or ignored the diploidy
or polyploidy of genomes when building 3D models. To our knowledge, only two
methods incorporated the polyploidy of genomes: \citet{kalhor:genome} and
\citet{ay:identifying}. Neither have been validated on simulated data, and
one might wonder, considering the complexity of solving such deconvolution
problems, how accurate and reliable these are. Now that single-allele Hi-C
datasets become available \citep{deng:bipartite}, this challenge can be more
thoroughly investigated.

\textbf{Inference of a population of structures using single-cell data}\ -
\citet{nagano:single-cell} published a protocol to assess physical
interactions in single cells, laying the foundation for studying the variability
of structures amongst a population of cell. So far, only two methods exploit
this type of data for 3D structure reconstruction: \citet{nagano:single-cell}
proposes a constraint-based approach to infer structures, while
\citet{paulsen:manifold} proposes to infer low-rank psd matrices as close as
possible to sparse contact count maps, and apply manifold learning technics to
find a euclidean embedding of the data. Both methods suffer from several
drawbacks: in particular, neither attempt to leverage several single cell
datasets or population contact maps to alleviate the sparsity of single-cell
contact maps. Performing a joint optimization may improve the accuracy and
robustness of inferring structures from single-cell data.

\textbf{\textit{De novo} sequencing using Hi-C data}\ - Several methods have
been proposed to re-target Hi-C for \textit{de novo} scaffolding
\citep{burton:chromosome, kaplan:high-throughput, marie-nelly:high-quality},
none of which leverage the recent work on convex relaxation for permutation
problems. Yet \citet{fogel:convex} proposes to solve exactly the challenge
faced in \textit{de novo} sequencing using Hi-C data: finding a permutation
matrix to reorder rows and columns such that strongly interacting elements are
close one another. Instead of relying on heuristics, one may attempt to use
these recent convex relaxation approaches.

This list of research perspectives is of course a very incomplete list of
possible extensions to this thesis, and I believe that in a short period of
time, many more challenges will arise in the field of Hi-C analysis.

% ---------------------------------------------------------------------------
% ----------------------- end of thesis sub-document ------------------------
% ---------------------------------------------------------------------------
