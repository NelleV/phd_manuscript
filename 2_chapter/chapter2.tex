% this file is called up by thesis.tex
% content in this file will be fed into the main document



\chapter[Inferring the 3D structure of the genome]{A statistical approach for inferring the 3D structure of the genome}
\label{chap:statistical}
\graphicspath{{2_chapter/figures/}}

\begin{work}

This chapter has been published in a slightly modified form in
\citep{varoquaux:statistical} and presented at ISMB 2014, as joint work with
Ferhat Ay, William S. Noble and Jean-Philippe Vert.

\end{work}

% Résumé en francais

\begin{abstract}{Résumé}

De récents développements dans les protocoles biologiques permettent désormais
de mesurer les fréquences d'interaction entre toutes les paires de loci d'un
génome, et ce en une seule experience. Le défi suivant est donc d'inférer des
modèles tri-dimensionnels du repliement des chromosomes dans
le noyau de la cellule. Les inférences proposées jusqu'à présent reposent
souvent sur des méthodes dites de {\em positionnement multidimensionnelle} (en
anglais \textit{multidimensional scaling}) (MDS),
lesquelles optimisent une structure telle que les distances relatives entre
éléments soient les plus proches de celles dérivées directement des fréquences
d'interaction. Ces approches optimisent une fonction objective heuristique,
et reposent sur des hypothèses contestables sur la biophysique de l'ADN pour trouver
la fonction de transfert entre fréquences d'interaction et distances
euclidiennes, pouvant ainsi conduire à l'inférence de modèles
incorrects de la structure de l'ADN.

Dans ce travail, nous proposons une nouvelle méthode pour inférer une
structure 3D consensus du génome à partir de données Hi-C. Notre méthode
repose sur un modèle statistique des fréquences d'interaction. Nous modélisons
les fréquences d'interaction comme une distribution de Poisson dont
l'intensité dépend de la distance physique entre paires de loci. Notre
méthode peut automatiquement ajuster les paramètres de la fonction de
transfert entre fréquences d'interaction et distances physiques, et infère un
modèle expliquant au mieux les données.

Nous comparons deux variantes de notre méthode (avec ou sans optimisation des
paramètres de la fonction de transfert) à quatre algorithmes basés sur MDS sur
des données simulées : deux MDS dit "métriques", dont les fonctions objectives
diffèrent, une version non métrique de cet algorithme, et ChromSDE qui propose
une version récente et convexe du MDS. Nous démontrons que notre modèle de
Poisson reconstruit des modèles plus précis que toutes les méthodes MDS, en
particulier lorsque la couverture des données est faible ou que les données
Hi-C sont à haute résolution, soulignant ainsi l'importance du choix de la
fonction objective à optimiser. Sur des données Hi-C publiques de cellules
embryoniques de souris, nous démontrons par ailleurs que les méthodes Poisson
infèrent des structures plus stables, plus reproductibles et plus robustes à
la fois lorsque le protocole biologique diffère, mais aussi à différentes
résolutions.

Une implémentation Python de notre méthode est disponible à
\href{http://cbio.ensmp.fr/pastis}{http://cbio.ensmp.fr/pastis}.
\end{abstract}


\begin{abstract}{Abstract}
\textbf{Motivation:} Recent technological advances allow the
measurement, in a single Hi-C experiment, of the frequencies of
physical contacts among pairs of genomic loci at a genome-wide
scale. The next challenge is to infer, from the resulting DNA-DNA
contact maps, accurate three dimensional models of how chromosomes
fold and fit into the nucleus. Many existing
inference methods rely upon {\em multidimensional scaling} (MDS), in
which the pairwise distances of the inferred model are optimized to
resemble pairwise distances derived directly from the contact
counts. These approaches, however, often optimize a heuristic objective
function and require strong assumptions about the biophysics of
DNA to transform interaction frequencies to spatial distance, and thereby
may lead to incorrect structure reconstruction.

\textbf{Methods:}
We propose a novel approach to infer a consensus three-dimensional
structure of a genome from Hi-C data. The method incorporates a
statistical model of the contact counts, assuming that the counts
between two loci follow a Poisson distribution whose intensity decreases
with the physical distances between the loci. The method can automatically
adjust the transfer function relating the spatial distance to the Poisson
intensity and infer a genome structure that best explains the observed data.

\textbf{Results:}
We compare two variants of our Poisson method, with or without
optimization of the transfer function, to four different MDS-based
algorithms---two metric MDS methods using different stress functions,
a nonmetric version of MDS, and ChromSDE, a recently described, advanced
MDS method---on a wide range of simulated datasets. We demonstrate that the
Poisson models reconstruct better structures than all MDS-based methods,
particularly at low coverage and high resolution, and we highlight the
importance of optimizing the transfer function. On publicly available Hi-C data
from mouse embryonic stem cells, we show that the Poisson methods lead to more
reproducible structures than MDS-based methods when we use data generated using
different restriction enzymes, and when we reconstruct structures at different
resolutions.

\textbf{Availability:} A Python implementation of the proposed method
is available at \href{http://cbio.ensmp.fr/pastis}{http://cbio.ensmp.fr/pastis}.
\end{abstract}

\section{Introduction}

Spatial and temporal three-dimensional (3D) genome architecture is
thought to play an important role in many genomic functions, but is
still poorly understood \citep{vansteensel:genomics}. In recent years, the
technique of chromosome conformation capture (3C)
\citep{dekker:capturing}, which identifies physical contacts between
different genomic loci and yields information about their relative
spatial distance in the nucleus, has paved the way for the systematic
analysis of the 3D structure of DNA. Coupled with high-throughput
sequencing, genome-wide conformation capture assays, broadly referred
to as {\em Hi-C} \citep{lieberman-aiden:comprehensive}, have emerged
as promising techniques to investigate the global structure of DNA at
various resolutions. Hi-C has opened new avenues to understanding many
biological processes including gene regulation, DNA replication,
somatic copy number alterations and epigenetic changes
\citep{shen:map,ryba:evolutionarily, de:DNA, dixon:topological}.

A typical Hi-C experiment yields a DNA {\em contact map}, that is, a
matrix indicating the frequency of interactions between all pairs of
loci at a given resolution.  A fundamental question is then to
reconstruct the 3D structure of the genome from this contact map. Two
general approaches have been proposed for that purpose: (i)
\emph{consensus methods} that aim at inferring a unique mean structure
representative of the data and (ii) \emph{ensemble methods} that
yield a population of structures.

Consensus approaches \citep{duan:three,
  tanizawa:mapping, bau:three-dimensional} model each
chromosome by a chain of beads, convert the contact map frequencies
into pairwise distances (which we refer as \emph{wish
  distances}) using various biophysical models of DNA, and infer a 3D
conformation that best matches the pairwise distances by solving a
multidimensional scaling (MDS) problem
\citep{kruskal:multidimensional2}. Converting interaction counts to
physical wish distances requires, however, strong assumptions which
are not always met in practice. For example, this mapping may change
from one organism to another \citep{fudenberg:higher-order}, from one
resolution to another \citep{zhang:inference}, from one genomic distance 
range to another \citep{ay:statistical}, or from one time point to 
another during the cell cycle \citep{le:high-resolution, ay:three-dimensional}.

To alleviate this problem, \citet{zhang:inference} proposed ChromSDE,
a method that jointly optimizes the 3D structure and a parameter of
the function that maps contact frequencies to spatial distances, in
addition to modifying the objective function of MDS. \citet{ben-elazar:spatial}
proposed an approach akin to \emph{nonmetric MDS}
\citep{kruskal:multidimensional}, where the 3D structure and the wish
distances are alternatingly optimized in an attempt to preserve coherence
between the ranking of pairwise distances and the ranking of pairwise
contact frequencies.

As for the ensemble methods, \citet{rousseau:three} and
\citet{hu:bayesian} describe two formal probabilistic models of contact
frequencies and their relationship with physical distances. They then
use a Markov chain Monte Carlo (MCMC) sampling procedure to produce an
ensemble of 3D structures consistent with the observed contact counts.
\citet{kalhor:genome} propose an optimization framework that generates
a population of structures by enforcing each contact to define an active
constraint in only a fraction of the inferred structures, thereby mimicking the
heterogeneity of contacts coming from each cell in the Hi-C sample.
Applying a similar method to budding yeast, \citet{tjong:physical}
demonstrate that a large population of structures inferred using known
physical constraints of yeast genome architecture can recapitulate, to a
large extent, the consensus contact map observed from Hi-C experiments.

Both consensus and ensemble models have benefits and
limitations. Ensemble approaches are biologically more accurate,
because Hi-C data is derived from a population of cells, each with
potentially a unique 3D architecture. An inferred population of 3D
structures may therefore better reflect the diversity of structures
than a single consensus structure.  In concordance with such ensemble
methods, a recent development in Hi-C
technology, assaying chromatin conformation at a single cell level,
demonstrates that chromatin structure varies highly from cell to cell
by modeling the single-copy X chromosomes of a male mouse cell line
\citep{nagano:single-cell}.

However, an ensemble approach
raises the question of interpretability: one often has to fall back to
interpreting a mean signal from the population structure
\citep{kalhor:genome} or to selecting a few structures, representative
in some way of the diversity of the population
\citep{rousseau:three}.  Consensus
methods, in contrast, provide a single structure more amenable to
visual inspection and analysis.  This structure can be seen as a useful
\emph{model} to recapitulate the rich information captured in Hi-C data
and to allow easy integration with other sources of data, such as
RNA-seq, which are usually also population based. In addition, despite
the stochasticity of cell-to-cell variations, certain hallmarks
of genome organization observed by consensus methods, such as chromosome
territories or topological domain organization, are conserved across
different cells \citep{nagano:single-cell,hu:bayesian}.
Computationally, ensemble methods are more demanding than consensus methods
since they need to sample from a very large dimensional space of possible
structures with complicated likelihood landscapes. Optimization-based
consensus methods are usually faster to converge to a local optimum,
but may miss the global optimum corresponding to the best structure when
the objective function is non-convex.


In this work, we focus on the consensus approach, and we propose a new
method to infer a 3D structure from Hi-C data. We propose to
replace the arbitrary loss function minimized by existing MDS-based
approaches by a better-motivated likelihood function derived from a
statistical model, similar to the one use by a previous ensemble method \citep{hu:bayesian}.
Specifically, our proposed method models the interaction frequency between two
loci by a Poisson model (PM), the intensity of which decreases with the
increasing spatial distance between the pair of loci.
Similar to the problem of inferring the wish distances
from interaction frequencies faced by MDS-based approaches, our model
faces the difficulty of transforming spatial distances into
intensities of the Poisson distribution. To solve this problem,
we propose two variant methods. The first method (PM1) uses a default
transfer function motivated by a biophysical model, whereas the second
method (PM2) uses a parametric family of transfer functions, the parameters
of which are automatically optimized together with the 3D structure to best
explain the observed data.

We compare both PM variants to four MDS-based methods, including metric
MDS with two stress functions, nonmetric MDS and ChromSDE. We demonstrate
on simulated data that the new models reconstruct more accurate 3D structures
than all MDS-based methods, especially at low coverage and high resolution.
We also assess the negative effect of using an incorrect transfer function,
and we show that PM2 is able to overcome this difficulty. On real data, we show
that, compared to MDS-based methods, PM1 and PM2 generate more similar models
when applied to replicate experiments performed with different restriction enzymes
or when applied to the same data at varying resolutions. The results suggest that
the Poisson model methods we describe here provide promising alternatives to
current methods for consensus DNA structure inference.

\section{Approach}

We model chromosomes as series of beads in 3D, each bead representing
a genomic window of a given length, and we denote by $\Xb =
(x_1,\ldots,x_n) \in \RR^{3 \times n}$ the coordinate matrix of the
structure, where $n$ denotes the total number of beads in the genome
(for example, $n=1216$ at 10kb resolution for the yeast genome) and
$x_i\in\RR^3$ represents the 3D coordinate of the $i$-th bead. The Hi-C
data can be summarized as an $n$-by-$n$ matrix $\mathbf{c}$ in which
each row and column corresponds to a genomic locus, and each matrix
entry $c_{ij}$ is a number, called the {\em contact frequency} or {\em
  contact count}, indicating the number of times locus $i$ and $j$
were observed to contact one another. The matrix is by construction
square and symmetric.

\subsection{Data normalization}
The raw contact count matrix suffers from many biases, some technical (from
the sequencing and mapping) and others biological (inherent to the physical
properties of chromatin) \citep{yaffe:probabilistic, imakaev:iterative}.
Therefore, before inferring the 3D structure of the genome, we normalize each
raw contact matrix using iterative correction and eigenvalue decomposition
(ICE) \citep{imakaev:iterative}, a method based on the assumption that all
loci should interact equally. Due to mappability issues, some beads have zero
contact counts. We remove these beads from the optimization and only try to
infer the positions of beads with nonzero contact counts.


\subsection{MDS-based methods}

\subsubsection{Metric MDS}

Metric MDS is a classical method to infer coordinates of points given their
approximate pairwise Euclidean distances~\citep{kruskal:multidimensional2}. To
use MDS in the context of DNA structure inference from Hi-C data, we need to
assign each pair of beads ($i$, $j$) a physical wish distance
$\delta_{ij}$---i.e., the distance that we aim to capture with our 3D
model---derived from the bead pair's contact count $c_{ij}$. Performing this
assignment requires us to decide how contact counts are transformed into physical
distances. In Section~\ref{sec:evaluating_parameters} we discuss a commonly used
transformation of the form $\delta_{ij} = \gamma
c_{ij}^{-3}$ if $c_{ij}>0$ motivated by polymer physics.
Metric MDS then places all the beads in 3D space such that
the Euclidean distance $d_{ij}(\Xb) = \|x_i - x_j\|$ between the beads $i$ and
$j$ is as close as possible to the wish distance $\delta_{ij}$. Denoting by
$\Dcal$ the subset of indices whose distances we wish to constrain (typically,
the set of pairs $(i,j)$ with non-zero contact counts $c_{ij}>0$), metric MDS
attempts to minimize the following objective function, usually called the \emph{raw
stress}:
\begin{equation}\label{eq:mds1}
\renewcommand{\arraystretch}{2}
\begin{array}{ccll}
\underset{\mathbf{X}}{\text{minimize}} & &
\underset{(i,j) \in \mathcal{D}}{\sum} \big(d_{ij}(\Xb) - \delta_{ij}\big)^2 \,.&\\
\end{array}
\end{equation}

In two previous studies that use metric MDS,
\citet{duan:three} and \citet{tanizawa:mapping} infer the 3D structure of DNA
from Hi-C data by solving Equation \ref{eq:mds1}, limiting $\Dcal$ to pairs of
indices with statistically significant contact counts (FDR 0.01\%). Both methods use
additional constraints such as
minimum and maximum distances between adjacent beads, minimum pairwise
distances between arbitrary beads to avoid clashes, and organism-specific constraints that concern the
positioning of centromeres, telomeres and ribosomal RNA coding regions. In
the experiments we present here, we simply solve Equation~\ref{eq:mds1} without
any constraints but including all pairs of beads with positive counts in $\Dcal$, and we call the
resulting method MDS1. In general, we have observed that
adding constraints related to minimal and maximal distances between beads
is unnecessary, because the structures found by MDS1 typically fulfill all of these constraints (data not shown).

A drawback of the raw stress of Equation~\ref{eq:mds1} in our context is that the
quadratic form is dominated by large values, corresponding to pairs of loci
with large wish distances (i.e., small contact counts). Because these counts are
less reliable than large contact counts, we propose a variant of MDS1, which
we call MDS2, where we weight the contribution of a pair $(i,j)$ in the stress
by a factor inversely proportional to the square wish distance between the corresponding
beads:
\begin{equation}\label{eq:mds2}
\renewcommand{\arraystretch}{2}
\begin{array}{ccll}
\underset{\mathbf{X}}{\text{minimize}} & &
\underset{(i,j) \in \mathcal{D}}{\sum} \delta_{ij}^{-2} \big(d_{ij}(\Xb) - \delta_{ij}\big)^2 \,.&\\
\end{array}
\end{equation}
While other weighting schemes could be proposed to decrease the influence of
pairs with large wish distances, we found this formulation to be quite robust
in practice. Notice that MDS2 can be thought of as a quadratic approximation of
the raw stress (minimized by MDS1) applied to log-transformed distances, because
in the setting $d_{ij}(\Xb) \approx \delta_{ij}$ it holds that:
\begin{equation*}
\begin{split}
\sum_{(i,j) \in \mathcal{D}} \br{\log d_{ij}(\Xb) - \log \delta_{ij}}^2
& = \sum_{(i,j) \in \mathcal{D}} \log\br{\frac{d_{ij}(\Xb)}{\delta_{ij}}}^2 \\
& \approx \sum_{(i,j) \in \mathcal{D}} \br{ \frac{d_{ij}(\Xb)}{\delta_{ij}}-1}^2 \,.
\end{split}
\end{equation*}
Both MDS1 and MDS2 implicitly
ignore non-interacting pairs of beads (i.e., pairs with zero contact
counts).

In addition to MDS1 and MDS2, we include in our benchmark ChromSDE
\citep{zhang:inference}, a recently proposed method which also attempts to
minimize a weighted stress function penalized by an additional term to push
non-interacting pairs far from each other. In addition, ChromSDE optimizes the
exponent of the transfer function that maps from contact counts to
wish distances. However, it
does not infer the relative positions of chromosomes. Accordingly, we compare
only the reconstruction of each individual chromosome produced by each method. Note
that, because intra-chromosomal counts are more reliable than inter-chromosomal
counts, ChromSDE should not be penalized compared to the other methods by
only considering intra-chromosomal counts.

\subsubsection{Nonmetric MDS (NMDS)}
\label{sec:nmds}

The derivation of the transfer function from contact counts to 3D wish distances, 
needed by metric MDS-based methods, relies on strong
assumptions about the physics of DNA
(Section~\ref{sec:evaluating_parameters}). NMDS \citep{shepard:analysis,kruskal:multidimensional} offers an alternative 
way to proceed, which was proposed in the context of DNA structure inference 
from Hi-C data by \citet{ben-elazar:spatial}. Instead of inferring physical
distances from the contact matrices, NMDS relies on the sole
hypothesis that if two loci $i$ and $j$ are observed to be in contact more
often than loci $k$ and $\ell$, then $i$ and $j$ should be closer in 3D space
than $k$ and $\ell$.  Using this hypothesis, NMDS attempts to solve the
following problem:
\begin{problem}
Given a set of similarities $c_{ij}$ (e.g., the contact frequency between $i$ and $j$), find 
$\mathbf{X} \in R^{3 \times n}$ such that:
\begin{equation}
\label{eq:ordinal_constraint}
c_{ij} \geq c_{k\ell} \Leftrightarrow \|x_i - x_j\|_2 \leq \|x_k -
x_\ell\|_2 \,.
\end{equation}
\end{problem}
Equation~\ref{eq:ordinal_constraint} is known as the nonmetric
constraint, or the ordinal constraint. This problem was first introduced
by \citet{shepard:analysis} and formalized as an optimization problem
by \citet{kruskal:multidimensional}. It can be solved by minimizing the
cost function:
\begin{equation}
\underset{\mathbf{X,\Theta}}{\text{minimize}}
\  \sum_{i, j} \frac{\br{\|x_i - x_j\|_2 -
\Theta(c_{ij})}^2}{\Theta(c_{ij})^2},
\end{equation}
with respect to the embedding $\mathbf{X}$ and the function $\Theta$,
where $\Theta$ is a decreasing function. Algorithms to solve this
optimization problem involve iterating over two steps: (1) fixing
$\Theta$ and minimizing the objective function with respect to $\mathbf{X}$ (hence falling
back to solve MDS2), and (2) fitting $\Theta$ to the
new configuration $\mathbf{X}$ subject to the ordinal
constraints. This second step of the algorithm can be performed using an
isotonic regression method, such as the pool adjacent violator
algorithm \citep{best:minimizing}.

A trivial solution of this problem is to set $\Theta$ equal to
$0$. In this case all points will collapse on the origin. To avoid this collapse,
we add additional constraints on $\mathbf{X}$ or on
$\Theta$, such as $\sum_{i, j} \|x_i - x_j\|_2 = K$ for some constant
value of $K$.

\subsection{Poisson model}
\label{sec:pm}

Instead of metric or non metric MDS-based methods, which attempt to minimize a stress function that measures a discrepancy between the wish distances and the 3D distances of the structure, we propose to cast the problem of structure inference as a maximum likelihood problem. For that purpose, we need to define a probabilistic model of contact counts parametrized by the 3D structure that we want to infer from contact count observations.

For that purpose, we take a model similar to the one used in the BACH algorithm~\citep{hu:bayesian} and model the contact frequencies $(c_{ij})_{(i,j)\in\Dcal}$ as independent Poisson random variables, where the Poisson parameter of $c_{ij}$ is a decreasing function of $d_{ij}(\Xb)$ of the form $\beta d_{ij}(\Xb)^\alpha$, for some parameters $\beta>0$ and $\alpha<0$.
We can then express
the likelihood as
\[
\ell(\mathbf{X}, \alpha, \beta) =
  \prod_{i, j} \frac{(\beta d_{ij}^{\alpha})^{c_{ij}}}{c_{ij}!}
	\exp (- \beta d_{ij}^{\alpha})\,.
\]
By maximizing the log likelihood, a new optimization problem naturally
emerges from this formulation:
\begin{equation}
\renewcommand{\arraystretch}{2}
\begin{array}{cll}
\underset{\alpha, \beta, \textbf{X}}{\text{max}} &
\mathcal{L}(\mathbf{X}, \alpha, \beta) = \underset{i<j\leq n}{\sum}  c_{ij}
\alpha \log d_{ij} + c_{ij} \log \beta - \beta d_{ij}^\alpha &\\
\end{array}
\end{equation}
With this new formulation, we can either provide the parameter $\alpha$, using
prior knowledge, and only optimize the structure and $\beta$ (which depends on
the dataset), or we can use a nonmetric approach, by inferring $\alpha$.
We refer to the former as PM1 and to the latter as PM2.

PM2 is solved using a coordinate-descent algorithm:
first choose randomly an $\mathbf{X}$ configuration, then iterate
between maximizing $\mathcal{L}$ with respect to $\alpha$ and $\beta$
 and, fixing $\alpha$ and $\beta$ and maximizing $\mathcal{L}$
with respect to $\mathbf{X}$.  In this work, we try to
initialize $\textbf{X}$ with a good approximation of the solution by
first evaluating the parameters $\alpha$ and $\beta$ using some prior
knowledge and initialize $\textbf{X}$ with the inferred structure from
the MDS.

%\begin{equation}
%\frac{\partial \mathcal{L}}{\partial{\alpha}}(\alpha, \beta, \textbf{X}) =
%  \sum_{i, j \in \mathcal{D}} c_{ij} \log d_{ij} - \beta  d_{ij}^{\alpha} \log d_{ij}
%\end{equation}

%\todo{If space allows, we could write some more equations, like the minimum in $\beta$ or the gradients.}
%\todo{We should change the name EM (expectation-minimization) because what we do is quite different from standard EM. In standard EM, one would model observed data (count matrix C) and hidden variable (X) through a parametric model $P_a(C,X)$. To optimize the parameter a, one would first perform a E step where we would compute $E_a(X|C)$ , and then adjust the parameters. What we do is not computing the expectation of the structure, but instead the best structure (that maximizes $P_a(X|C)$, assuming a uniform prior on X) before reoptimizing the weights. To make a comparison with a more classical model, it is as if for hidden markov models we would run a Viterbi to find the most likely hidden variables, then optimize the parameters to maximize the likelihood of the hidden variables, as opposed to the EM approach (Baum-Welsh) where instead of Viterbi we need to estimate the expectation of the hidden variables. Mathematically, the EM procedure ensures that the likelihood of the observations observe, while what we do does not. By the way, a real EM may be relevant, in fact it would probably involve sampling a family of structures (as in the population methods), then do the average distances over the populations and optimize the parameters on that (details to be checked...).}

All optimization problems (MDS1, MDS2, NMDS, PM1 and PM2) were solved using IPOPT, an interior point filter
algorithm \citep{wachter:on} and the isotonic regression implementation from
the Python toolbox Scikit-Learn for NMDS \citep{pedegrosa:scikit}.


\subsection{Default contact-to-distance transfer function}
\label{sec:evaluating_parameters}

A prerequisite for both the MDS and the PM1 model (and for good initialization
of the NMDS and PM2 methods) is a function that converts from contact counts
to wish distances.  Extensive previous studies of the behaviour of polymers in
general and DNA in particular have yielded proposed relationships between, on
the one hand, the genomic distance $s$ and contact counts $c$ and, on the
other hand, genomic distance $s$ and physical distances $d$ for several
classes of polymers \citep{grosberg:role, lieberman-aiden:comprehensive,
fudenberg:higher-order}. For a fractal globule polymer, representative of
mammalian DNA, the contact count is inversely proportional to the genomic
distance ($c \sim s ^{-1}$), whereas the volume scales linearly with the
subchain length ($d^3 \sim s$), from which we deduce a relationship between
$d$ and $c$ of the form $d \sim c ^{-1/3}$. For an equilibrium globule,
representative of a smaller genome such as {\em S.\ cerevisae}, the
relationships differ: $c \sim s ^{-3/2}$ and $d \sim s ^ {1 / 2}$ up to a
maximum distance, corresponding to the size of the nucleus in which the DNA is
confined. Conveniently, coupling those two relationships for either type of
polymer yields the same mapping between contact counts and physical distances:
\begin{equation}
d \sim c ^{-1/3}.
\end{equation}
Thus, by default we convert contact counts $c_{ij}$ into 3D wish distances
$\delta_{ij}$ using the following relationship:
\begin{equation}
\delta_{ij} = \gamma c_{ij} ^{-1/3},
\end{equation}
where $\gamma$ defines the scale of the structure. It is important to note that 
this relationship holds true for only a subset of the full genomic distance range and
that this range varies for different genomes. In practice, we will not
infer $\gamma$ for the MDS and NMDS problem: the structures can easily be
rescaled after convergence to match biological knowledge of the organism
studied.

\subsection{Data}
\label{sec:data}

In order to test various 3D architecture inference methods, we
conducted experiments on both simulated datasets and publicly
available genome-wide Hi-C datasets.

For the simulation, we generated 170 data sets using the yeast genome
architecture proposed by \citet{duan:three}. Because the repetitive
rDNA on yeast chromosome XII cannot be observed in practice, we
discard all contacts involving these loci, and we do not infer the
position of the corresponding rDNA. We generate these 170 datasets
using the following model:
\begin{equation}\label{eq:simu}
c_{ij} = P(\beta d_{ij}^\alpha),
\end{equation}
where $\alpha = -3$ (corresponding to the theoretical exponent discussed in Section~\ref{sec:evaluating_parameters})
and $\beta$ varies between $0.01$ and $0.7$ ($0.01$, $0.01$, $0.02$, $0.03$,
$0.04$, $0.05$,
$0.06$, $0.07$, $0.08$, $0.09$, $0.1$, $0.15$, $0.2$, $0.3$, $0.4$, $0.5$, $0.6$,
$0.7$) with 10
different random generator seeds, thus obtaining 10 different datasets per
parameter.
The $\beta$ parameter controls the number of contact counts in the
datasets. A low $\beta$ will yield a dataset with few counts; hence, the
corresponding wish distance matrix will be less likely to be close to the true
distance matrix. To estimate how noisy the generated data is, we compute the
following measure of signal-to-noise ratio (SNR):
\begin{equation}
SNR = \frac{\sum{c_{ij}}}{\sqrt{\sum (\beta d_{ij} ^{\alpha} - c_{ij})^2}}\,.
\end{equation}
The numerator (the signal) corresponds to the number of counts, and
the denominator (the noise) corresponds to the sum of deviation
between each count and its expected value.  We use this first ensemble
of simulated datasets to assess the robustness to noise of the
different methods. Note that in actual data, the SNR gets smaller when we sequence fewer reads or when we infer a structure at a higher resolution.

We simulated another ensemble of datasets to compare nonmetric and metric
methods when the parameters provided to the different algorithms are not
the correct ones. We generate 20 datasets according to
Equation~\ref{eq:simu}, with $\alpha$ between $-4$ and $-2$ ($-4$, $-3.5$,
$-3$, $-2.5$, $-2$) and $\beta$ between
$0.4$ and $0.7$ ($0.4$, $0.5$, $0.6$, $0.7$).

We also applied our methods to publicly available Hi-C data from mouse
embryonic stem cells (mESC) \citep{dixon:topological}.  We started with the
data at 20~kb resolution and considered only chromosomes 1 to 19, with both
available restriction enzymes (HindIII and NcoI).  We then subsampled the data
at resolutions of 100~kb, 200~kb, 500~kb and 1~Mb.
Note that the methods studied here infer a single copy per
chromosomes, thus yielding a consensus model for both homologous chromosomes.


\subsection{Structure similarity measures}
In order to assess the ability of a method to reconstruct a known
structure from simulated data, or the stability of the reconstructed structure with respect to change in resolution or library preparation, we need quantitative measures of similarity between 3D structure. We use two such measures: the root mean square deviation (RMSD) and the distance error, which we now explain.

The RMSD is a
standard way to compare two sets of structures described by their
coordinates $\mathbf{X}, \mathbf{X}' \in R^{3 \times n}$, widely used for example to compare protein 3D structures. It is given by:
\begin{equation*}
RMSD = \min_{\mathbf{X}^*} \sqrt{\sum_{i = 1}^n (\mathbf{X}_{i} -
  \mathbf{X}_{i}^*)^2} \,,
\end{equation*}
where the structure $\mathbf{X}^*$ is obtained by translating,
rotating and rescaling $\mathbf{X}'$ ($\mathbf{X}^* = s \mathbf{R X'} -
\mathbf{t}$, where $\mathbf{R} \in R^{3 \times 3}$ is a rotation
matrix, $\mathbf{t} \in R^{3}$ is a translation vector, and $s$ is a
scaling factor). Because ChromSDE
does not infer the relative position of chromosomes, the RMSD values we report below
are sums of RMSDs computed independently on each chromosome.

We also directly compare the 3D distance matrices corresponding to the two structures with the distance error:
\begin{equation*}
\text{distanceError} = \sqrt{\sum_{i, j = 0}^n (d_{ij}(\Xb) - d_{i, j}(\Xb'))^2} \,.
\end{equation*}


The main difference between the optimization formulated by ChromSDE
and those of the other methods is the penalty assigned to
non-interacting beads.  Due to this penalty, ChromSDE should recover
better long distances than other MDS-based methods.  This property is
not well captured by the RMSD measure, therefore, we also compute how
well the distance matrix is recovered with the distance error, which assigns most of the weight to long distances. We
expect that methods based on MDS, which optimize an objective function
based on the distance matrix, should perform better on this measure
than others.

\section{Results}

To assess the relative strength of our new Poisson model-based methods, PM1 and PM2, we compare them to a panel of four MDS-based methods: MDS1, MDS2, NMDS and ChromSDE on simulated and real data.

\subsection{Simulated Hi-C data}

We first tested the six methods on data simulated as explained in Section~\ref{sec:data}.

\begin{figure}
\includegraphics[width=\linewidth]{{varoquaux.78.fig1}.pdf}
\caption{\textbf{Performance evaluation on simulated data, varying the parameter $\beta$.}
\textbf{A} RMSD of each experiment for varying values of the parameter $\beta$. ChromSDE
failed to yield consistent results for 14 experiments (It reported the wrong number of beads
in the results file.), and the PM2 algorithm
failed to converge at the desired precision for one experiment (It exceeded the
maximum number of iterations.).
\textbf{B} Distance error of each experiment for varying values of $\beta$.
\textbf{C} Average SNR for each $\beta$. Higher SNR corresponds to better quality data.
}
\label{fig:generated_data}
\end{figure}

\subsubsection{Performance as a function of SNR}

We ran all six methods---MDS1, MDS2, NMDS, PM1, PM2 and ChromSDE---on the 170
simulated datasets with varying SNR levels. Our goal here is to assess how
well the different methods manage to reconstruct a known 3D structure from
simulated data at different SNR levels. Remember that SNR estimates how far
the empirical counts differ from their expectations; in real Hi-C data, SNR
typically decreases when we have fewer reads in total, or when we want to
increase the resolution of the structure. In this first series of experiments,
we provide the correct count-to-distance or distance-to-count transfer
functions to the methods that need them (MDS1, MDS2, PM1). In this setting,
for infinite SNR, all methods should consistently estimate the correct
structure.

Figure~\ref{fig:generated_data} shows the performance of the different methods
in terms of RMSD (top) and distance error (middle) as a function of the
$\beta$ parameter, which controls the SNR (bottom). As expected, all methods
perform well when the SNR is high, but exhibit marked differences in
performance for finite SNR. In the low SNR setting (SNR $< 2$), both PM1 and
PM2 significantly outperform all MDS-based methods, in both RMSD and distance
error. Interestingly, we observe no significant difference between PM1 and
PM2, which shows that there is no price to pay in terms of inferred structure
if we don't specify the exponent of the distance-to-count transfer function.
In this setting, PM2 is able to estimate the structure accurately enough to
produce a structure of the same quality as PM1. Among MDS-based methods, we
see that NMDS generally outperforms MDS2, which itself outperforms MDS1. This
observation highlights that in the non-asymptotic, low SNR setting, the choice
of stress function influences the performance of MDS. ChromSDE performs better
than other MDS-based methods on datasets with a low SNR, corresponding to
datasets with low coverage and, consequently, many non-interacting pairs of
beads. This may be due to the way ChromSDE explicitly handles such pairs. On
the other hand, in a more favorable setting (SNR $>2$), ChromSDE does not
perform as well as other MDS-based method; we hypothesize that when the
coverage is high enough, taking into account non-interacting pairs of beads
does not add any additional information. Since ChromSDE is not better than
other MDS-based methods, and requires much longer to run, we do not report its
performance on the next experiments and instead focus on the differences
between the other MDS-based methods and the PM methods.

\subsubsection{Metric versus nonmetric methods: robustness to incorrect parameter estimation}

\begin{figure}
\includegraphics[width=\linewidth]{{varoquaux.78.fig2}.pdf}
\caption{\textbf{Performance evaluation for simulated data, varying
    the parameter $\alpha$.} The figure plots the average RMSD of the
  inferred structures for a range of $\alpha$ values.  As $\alpha$
  increases, the SNR of the dataset also increases.}
\label{fig:rmsd_alpha}
\end{figure}

Three of the methods tested, which we collectively refer to as \emph{metric}
methods, require as input a count-to-distance or distance-to-count transfer
function: MDS1, MDS2 and PM1. In reality, however, the DNA may not follow the
ideal physical laws underlying the default transfer function discussed in
Section~\ref{sec:evaluating_parameters}, and the structures inferred from
these methods may diverge from the correct one because of miss-specification
of the transfer function.

To assess this phenomenon, and evaluate the robustness of the different
methods (including NMDS and PM2, which automatically infer a transfer
function), we now study the performance of the methods on datasets generated
with varying $\alpha$ parameters. We therefore run the MDS1, MDS2, NMDS, PM1
and PM2 methods on the second ensemble of simulated datasets. We provide the
default transfer function to all metric methods, thus inducing a
miss-specification for all simulated datasets with $\alpha\neq -3$.

Figure~\ref{fig:rmsd_alpha} shows the RMSD of each method, averaged
over the datasets with different $\beta$, as a function of $\alpha$.
The performance curve of PM1, which is the best method when the data
are simulated with the correct parameter $\alpha=-3$, exhibits a
characteristic U-shape centered around $\alpha = -3$. This curve
confirms that PM1 performs better when given the true parameter and
performs worse as $\alpha$ moves away from $-3$. On the other hand,
the performance curves of the two other metric methods, MDS1 and MDS2,
do not exactly follow this trend: MDS1 and NMDS perform increasingly
better when $\alpha$ decreases, and MDS2 achieves the best performance
when $\alpha = -3.5$. This phenomenon occurs because in our
simulation, when $\alpha$ decreases, the SNR for a given $\beta$
increases, counterbalancing the negative effect of the transfer
function miss-specification.  Thus, for MDS-based methods, it is
apparently more important to have more data than to have a correct
$\alpha$ parameter. Finally, we see that, as expected, the non-metric
approaches, NMDS and PM2, are more robust to transfer function
misspecification than the metric approaches, because they
automatically estimate it. When the parameter is wrong, PM2
outperforms the other methods for low SNR, whereas for high SNR, NMDS
performs better.

\subsection{Real Hi-C data}

We now test the different methods on real Hi-C data. Since in this case the
true consensus structure is unknown, we investigate the behaviors of the
different methods in terms of their ability to infer consistent structures
from different datasets and across resolutions.

\begin{table*}
\begin{center}
\footnotesize
\begin{tabular}{ccccccccccccc}
\hline
{\tiny Resolution} &  Corr &
\multicolumn{2}{c}{MDS1} &
\multicolumn{2}{c}{MDS2} &
\multicolumn{2}{c}{NMDS} &
\multicolumn{2}{c}{PM1} &
\multicolumn{2}{c}{PM2} \\
&& {\tiny RMSD} & Corr & {\tiny RMSD} & Corr & {\tiny RMSD} &Corr & {\tiny
RMSD} & Corr & {\tiny RMSD} & Corr \\
\hline
1~Mb  & 0.981 & 13.13 & 0.945 & 5.54 & 0.964 &  5.80 & 0.965 & 7.28  & 0.931 & \textbf{4.92} & \textbf{0.976} \\
500~kb & 0.959 & 10.00 & 0.942 & 5.68 & 0.959 & 5.67 & 0.959 & 7.14 & 0.913 & \textbf{4.66} & \textbf{0.968} \\
200~kb & 0.845 & 5.64 & 0.940 & 3.74 & 0.945 & 3.73 & 0.946 & 4.01 & 0.891 & \textbf{3.42} & \textbf{0.958} \\
100~kb &  0.605 & 5.07 & 0.736 & 2.53  & 0.676 & 2.52 & 0.666 & \textbf{2.51} & 0.664 & 2.76 & \textbf{0.771}\\
\hline
\end{tabular}
\end{center}

\caption{\textbf{Stability across enzyme replicates.} For each resolution, the
table lists the Spearman
correlation the two enzyme replicate datasets, and, for each inference
method, the average RMSD and Spearman correlation between pairs of structures
inferred from the two datasets.  Boldface values correspond to the best RMSD
or correlation values among all five methods.  In general, higher resolution
leads to a lower correlation between pairs of inferred structures.}

\label{table:results_real_data}
\end{table*}
\subsubsection{Stability to enzyme replicates}

The Hi-C assay depends upon a restriction enzyme to cleave the DNA after
cross-linking, and the same sequence library can be analyzed multiple times
using different enzymes.  Although the resulting restriction fragments will
differ, we expect {\em a priori} that the overall genome architecture should
be the same from such replicate experiments.  We therefore evaluate each
genome architecture inference method with respect to the similarity of the
structures inferred from two replicate Hi-C experiments that differ only in
the choice of restriction enzyme.  Specifically, we apply each method to two
enzyme replicates, HindIII and NcoI, carried out in mouse ES cells
\citep{dixon:topological} for chromosomes 1--19.

To measure the stability of the methods, we compute (1) the Spearman
correlation between the two pairwise Euclidean distance matrices of
the pairs of predicted structures and (2) the RMSD between the
rescaled predicted structures.  Note that, before computing our two
error measures, we filter out from the pair of structures any beads
for which the inference hasn't been done on either dataset, i.e.,
beads that have zero contact counts in either data set.

To give a sense of how similar the two replicate datasets are, we also
compute the Spearman correlation directly on the data, rather than on
the inferred structures.  As expected
(Table~\ref{table:results_real_data}), the higher the resolution is,
the lower the correlation between the pairs of datasets is and the
more different the inferred structures are.
Across different enzyme replicates, the PM2 method yielded significantly
higher correlation than all of the other methods ($p < 0.05$,
signed-rank test
adjusted for multiple tests with a Bonferroni correction).



\subsubsection{Stability to resolution}

\begin{table}
\begin{center}
\begin{tabular}{clllll}
\hline
   & MDS1 & MDS2 & NMDS & PM1 & PM2 \\
\hline
{\em RMSD }        & 14.86 & 12.92 & 12.98 & 13.03 & \textbf{11.48} \\
{\em Correlation } &  0.781 & 0.754 & 0.738 & 0.737 & \textbf{0.807} \\
\hline
\end{tabular}
\end{center}
\caption{\textbf{Stability across resolution.} The table lists the
  average RMSD and Spearman correlation between pairs of structures of
  different resolutions. In bold are the lowest average RMSD and
  highest average Spearman correlation. These values were computed on
  mouse ESC HindIII libraries \cite{dixon:topological})}
\label{table:real_hic_stability}
\end{table}

\begin{figure*}
\begin{center}
\includegraphics[width=0.9\linewidth]{{varoquaux.78.fig3}.png}
\end{center}
\caption{\textbf{Predicted structures for chromosome 1 at different resolution}
Contact counts matrices and predicted
structures for the MDS2, NMDS, PM1 and PM2 methods at 1~Mb (\textbf{A}),
 500~kb (\textbf{B}), 200~kb (\textbf{C}),
100~kb (\textbf{D})}
\label{fig:real_data}
\end{figure*}

\citet{zhang:spatial} show that the mapping from contact counts to
physical distance differs from one resolution to another, underscoring
the importance of good parameter estimation. To study the stability of
the structure inference methods to changes in resolution, we compute
the RMSD between pairs of structures inferred at different
resolutions.  Let $(\textbf{X}, \textbf{Y}) \in (R^{3 \times n}, R^{3
  \times m})$ be a pair of predicted structures such that $n < m$
(i.e., $\textbf{X}$ is a structure at a lower resolution than
$\textbf{Y}$). We compute a downsampled structure $\textbf{Y}^* \in
X^{3 \times n}$ at the same resolution as $\textbf{X}$ by averaging
the coordinates of beads.  We then compute the RMSD between this new
structure $\textbf{Y}^*$ and $\textbf{X}$, as well as a corresponding
Spearman correlation of the distance matrices.

Results are shown in Figure~\ref{fig:real_data} and
Table~\ref{table:real_hic_stability}. PM2 is significantly ($p < 0.05$)
more stable to resolution changes, both in terms of RMSD and correlation of
distances.

\section{Discussion and conclusion}

In this work, we present a novel method for inferring a consensus genomic 3D
structure from Hi-C data. The method maximizes a likelihood derived from a
statistical model of the relationship between the contact counts and physical
distances, and includes an automatic tuning of the parameters defining the
link between a 3D distance and the Poisson parameter of the corresponding
contact count. We showed in simulations that the new method outperforms a
panel of MDS-based approaches, including ChromSDE, which optimize an often
ad-hoc stress function. The improvement is particularly important at low SNR,
corresponding to more difficult problems where we want to increase the
resolution of the model with a fixed total number of reads; this is typically
the situation where one expects a correct maximum likelihood estimator to
outperform more {\em ad hoc} estimators. We also showed that misspecification
in the count-to-distance transfer function can harm the performance of metric
methods, while our model can adapt to unknown distributions within a
parametric family. Finally, we also demonstrated, on real Hi-C data, the
robustness of our methods to resolution change and enzyme duplicated datasets.

Our probabilistic model of reads is similar to the model proposed
by~\citet{hu:bayesian}; however, instead of generating a family of structures
by MCMC we use the model for direct maximum likelihood estimation of a
consensus structure. Although the consensus structure might not be a
definitive structure {\em in vivo}, it provides us with a rich model for
further analysis, conserving hallmarks of genome organization such as the
water lily form of the budding yeast \citep{duan:three} or topological domains
\citep{kalhor:genome}.

The Poisson model underlying our approach remains very basic and could be
subject to many improvements.  For example, physical constraints, such as the
size of the nucleus, could be incorporated into the model. Better models for
zero entries may be possible, because those can either come either from
non-interacting loci or from measurement errors due to, e.g., mappability
problems. Overall, expressing the structure inference problem as a maximum
likelihood problem offers a principled way to improve the method by improving
the probabilistic model of measured dat.
