\section*{Methods}
\subsection*{Experimental protocols}

\subsubsection*{{\em P.\ falciparum} strain and culture conditions}
{\em P.\ falciparum} strain 3D7 was maintained in human O+ erythrocytes in
5\% haematocrit according to a previously described protocol \citep{trager:human}.
Cultures were synchronized twice at ring stage with 5\% D-sorbitol treatments
performed eight hours apart \citep{lambros:synchronization}. Parasites were
harvested 48 hours after the first sorbitol treatment (0h; ring stage), and
then 18 hours (early trophozoite stage) and 36 hours (late schizont stage)
thereafter. The developmental stage of the parasites was verified by microscopy
using Giemsa-stained blood smears prior to harvesting.

\subsubsection*{Cross-linking}
Aspirated {\em P.\ falciparum} cultures were pooled into 50 ml centrifuge
tubes and filled up to 35 ml with phosphate buffered saline (PBS) warmed
to 37$^\circ$C. Cultures were treated with 3 ml 16\% formaldehyde
(1.25\% final concentration) and incubated for 25 min at 37$^\circ$C while rocking.
Formaldehyde was quenched with 5.2 ml 1.25 M glycine (final concentration 150 mM)
for 15 min at 37$^\circ$C while rocking, followed by 15 min at 4$^\circ$C while
rocking. PBS was used instead of formaldehyde and glycine for the not cross-linked
control. Cultures were spun at 660 $\times$ g for 20 min at 4$^\circ$C. Not
cross-linked control parasites were treated with 5 volumes 0.15\% saponin in water
and incubated 10 min at 4$^\circ$C while rocking. PBS was used instead of saponin
for the cross-linked parasites. Parasites were spun at 660 $\times$ g for 15 min
at 4$^\circ$C. Pellets were washed multiple times until clean and stored at -80$^\circ$C.

\subsubsection*{Tethered conformation capture procedure}
We applied an adapted Hi-C method referred to as tethered conformation capture
(TCC) \citep{kalhor:genome} to map the intra and interchromosomal contacts in
{\em Plasmodium falciparum}. For a detailed description of the overall protocol
see Supplementary Note 1.

\subsubsection*{DNA-FISH}
For each 10 kb locus of interest, we determined the location for which on average
the highest number of contact counts were observed and designed DNA probes targeting
the 2 kb region surrounding this location. Probes were prepared using Fluorescein-High
Prime and Biotin-High Prime kits (Roche) according to manufacturer's instructions.
Template DNA was prepared by PCR (5 min at 95$^\circ$C, 35 cycles of 30 sec at
98$^\circ$C followed by 150 sec at 62$^\circ$C, and 5 min at 62$^\circ$C) using
the KAPA HiFi DNA Polymerase HotStart ReadyMix. Sequences of primers used for probe
generation are shown in Supplementary Table~\ref*{table:FISHprimers}. For a detailed
description of the DNA-FISH protocol see Supplementary Note 3. The percentage of
colocalization was determined by visual inspection of $>$100 cells per condition.

\subsection*{Computational methods}

\subsubsection*{Mapping and filtering of sequence data}
\label{met:mapping}
We first trimmed each end of the paired-end reads from all samples to 40~bp.
%as our read lengths varied between 47 to 101~bp.
We used FastQC~\citep{andrews:fastqc}
% (\url{http://www.bioinformatics.babraham.ac.uk/projects/fastqc})
reports of aggregate read qualities for each sample to determine the amount
of trimming required from each end of the read to keep the highest quality
$40$-bp region.
%For short reads we only trimmed the reads from the $5'$ end, whereas for long reads we trimmed from both the $5'$ and the $3'$ ends.
% read lengths = 51 51 51 51 51 51 47 51 47 51 47 51 97 101 47 51 97 101
% trim 5' = 8 8 8 8 8 8 7 8 7 8 7 8 27 27 7 8 27 27
% trim 3' = 3 3 3 3 3 3 0 3 0 3 0 3 30 34 0 3 30 34

To filter out reads from human DNA, we mapped the trimmed paired-end reads to
the human genome (UCSC hg19) using the short read alignment mode of BWA (v0.5.9)~\citep{li:fast} 
with default parameter settings. Each end of the paired reads was mapped
individually. We post-processed the alignment results to extract reads that
mapped with an edit distance of at most 3. We then eliminated all pairs for
which at least one of the ends mapped to the human genome without any filtering
on the mapping quality or uniqueness. This loose mapping criteria is used to
assure that any read pair that is likely to come from human blood contamination
in the parasite samples is filtered out from our further analysis of
{\em Plasmodium} genome architecture.

We mapped the remaining paired-end reads to the \emph{Plasmodium falciparum 3D7}
reference genome (PlasmoDB v9.0). We post-processed the alignment results further
to extract the reads that mapped (i) uniquely to one location in the reference
genome, (ii) with an alignment quality score of at least 30 (which corresponds to
a 1 in 1000 chance that the mapping is incorrect), and (iii) with an edit distance
of at most 2. We extracted the paired-end reads with both ends mapping to the
{\em Plasmodium} genome. We then identified potential PCR duplicates, i.e., pairs
of read-pairs with identical genomic coordinates, and retained only one copy of
each. We also filtered out reads that map to intrachromosomal loci that are
$\le$1~kb apart. We refer to the remaining reads as \emph{informative reads}. We
computed chromosomal contact maps using only these informative reads. Supplementary
File 1 summarizes the results of applying this pipeline to our sequencing libraries.


\subsubsection*{Calculating noise level and percentage of long range contacts}
\label{met:ICP}
We calculated two measures that provide estimates of the noise level and efficiency
of the assay. The first is the interchromosomal contact probability (ICP)
index \citep{kalhor:genome}:
\[
ICP=\frac{\sum{\text{interchr contact counts}}} {\sum{\text{intrachr contact counts ($>$1 kb)}}}
\]
In the denominator, the intrachromosomal contact counts exclude contacts between
pairs of loci $\le$1~kb apart. Smaller ICP values indicate a better signal-to-noise
ratio, assuming that the real data (signal) will be enriched for intrachromosomal
contacts, whereas noise will be dominated by interchromosomal contacts. The second
number is the percent of long-range contacts (PLRC) extracted from the initial set
of paired-end reads that remain after filtering the reads that mapped to human genome:
\[
PLRC=\frac{\sum{\text{interchr contact counts}} + \sum{\text{intrachr contact counts ($>$20 kb)}}} {\text{Number of raw reads after human DNA filtering}}
\]
The bigger this percentage is, the more information the dataset provides about
non-adjacent chromatin contacts for the amount of sequencing in hand.


\subsubsection*{Aggregating data relative to 10~kb windows}
\label{met:10kb}
Digesting the DNA with a frequently cutting restriction enzyme yields a very large
number of possible pairs of restriction fragments (i.e., locus pairs). In our case,
digesting the {\em Plasmodium} genome with MboI, which cuts at the 4~bp recognition
site ``GATC'', yielded 28,784 fragments (mean length 810~bp) corresponding to
33,114,193 intrachromosomal and 336,629,028 interchromosomal locus pairs. For 3D
modeling, we partitioned the {\em Plasmodium} genome into a collection of
non-overlapping 10~kb windows, and we assigned each restriction fragment to the
10~kb window that covers the majority of the bases in the fragment. This operation
reduced the number of possible fragments from 28,784 to 2,337 and the number of
possible locus pairs from $3.7 \times 10^8$ to 2,715,615 (228,539 intrachromosomal
and 2,487,076 interchromosomal).

\subsubsection*{Normalizing raw contact maps}
\label{met:normalization}
For each possible pair of 10~kb loci, we refer to the total number of informative
read pairs that link the two loci as the {\em contact count}, and we refer to the
two-dimensional matrix containing these contact counts as the {\em raw contact map}.
We normalized the raw contact maps in two steps.  First, we ranked loci by their
percentage of intrachromosomal contacts with zero counts, and we filtered out the
top 2\% of this list. This removes all loci for which the signal to noise ratio
is too low (typically, regions of low mappability). Second, we applied an iterative
correction and eigenvector decomposition (ICE) method~\citep{imakaev:iterative}
that attempts to eliminate systematic biases in Hi-C data. The method estimates a
bias vector with one entry per locus. The tensor product of the bias vector with
itself generates a bias matrix $B$ that can be used to convert the raw contact map
into a normalized contact map.

\subsubsection*{Estimating power-law fits to intrachromosomal contact probabilities}
\label{met:power-law}
It has been observed in the literature that for a pair of intrachromosomal loci,
the relationship between genomic distance and the expected contact count can be
estimated by a log-linear model \citep{lieberman-aiden:comprehensive, fudenberg:higher-order}.
This log-linear model is captured by a power-law fit of the form $P(s) \sim s^\alpha$
where $s$ denotes the genomic distance, $P(s)$ denotes the expected contact
probability at distance $s$ and $\alpha$ is the gradient of the log-linear fit.
For each stage, we first calculated $P(s)$ by segregating all intrachromosomal locus
pairs into $b=50$ equal-occupancy bins. This procedure involves enumerating all
possible intrachromosomal locus pairs (including pairs that have a contact count of zero),
sorting the pairs in increasing order according to their genomic distances, and then
segregating the resulting list into $b$ quantiles. For each bin $i$, we computed the
average number of contact counts per locus pair $\hat{c}_i$, and the average contact
distance $\hat{s}_i$ over all locus pairs in the bin. Then, for each bin $i$,
$P(\hat{s}_i)= \frac{\hat{c}_i}{N}$ where $N$ is the sum of all observed intrachromosomal
contact counts. We then found the best linear fit to $\log P(s)$ versus $\log s$ in
a given genomic distance range. Note that the control library  ``TROPH.-cont.''
was not subjected to normalization.


\subsubsection*{Assigning statistical significance to normalized contact maps}
\label{met:fithic}
To obtain a set of high confidence contacts for each stage, we subjected the contact
maps at 10 kb resolution to a statistical confidence estimation procedure
\citep{ay:statistical}. We first accounted for the effect of genomic distance on the
intrachromosomal contact probability by fitting a smoothing spline to capture this
effect. We then accounted for biases using the normalization procedure described
above. Finally, we calculated p-values for intra and interchromosomal contacts and
corrected them jointly for multiple hypothesis testing to compute q-values, which
are used to filter contacts at a desired false discovery rate. For a detailed
description of the statistical significance estimation procedure see Supplementary Note 2.

\subsubsection*{Identifying stage-specific contacts}
\label{met:spageSpecific}
We determined the contacts that are specific to only one stage or to two out of three stages
as follows. First, we sorted the lists of contacts at 10~kb resolution according to increasing
p-values computed as described above for each stage. Then, we extracted contacts that are
ranked in top 1,000 in each stage and checked to see whether they appear among top 10,000
contacts for the other two stages. We labeled these contacts as stage-specific because they
are among the strongest contacts for one stage but not among moderately-strong contacts for
the other two stages. Similarly, we labeled contacts that are in top 1,000 in two out of
three stages but not in top 10,000 for the third stage. To perform gene set enrichment
analysis (GSEA), we extracted the lists of genes that are involved in stage-specific contacts
(only ring, only trophozoite or only schizont) as well as contacts common to two stages
(common to ring and trophozoite, common to ring and schizont or common to trophozoite and schizont).


\subsubsection*{Inferring the 3D structures}
\label{met:inferring}
Our method for inferring the 3D structures is based on the method of \cite{duan:three}.
Each chromosome is modeled as a series of beads on a string, spaced approximately
10~kb apart. We associated with each pair of beads $x_i$ and $x_j$ a physical
{\em wish distance} $\delta_{ij}$---i.e., the distance that we aim to capture with
our 3D model---derived from the bead pair's contact count $c_{ij}$. We then
placed all the beads in 3D space such that the distance $d_{ij}$ between the beads
$i$ and $j$ is as close as possible to the wish distance $\delta_{ij}$.

\paragraph{Wish distances: }

To obtain the wish distances, we note that two proximal intrachromosomal loci
are likely to come into contact due to random looping of the DNA, and that
this ``polymer packing'' contact likelihood can be expressed as a function of
the genomic distance $s$ between the loci. We then assumed that two loci with
observed contact count $c_{ij}$ will have the same physical distance
$\delta^{ij}$ as two intrachromosomal loci with expected contact count
$c_{ij}$ by polymer packing. The relationship between the expected contact
frequencies and the genomic distances $s$ suggests that \textit{P.\
falciparum}'s DNA behaves like a fractal globule polymer
\citep{lieberman-aiden:comprehensive} (Supplementary Fig.
\ref*{suppfig:power-law}). Any crumpled polymer exhibits a well-defined
relationship between its genomic length $s$ and the physical distance $d$
\citep{grosberg:role}:
\begin{equation}
d \sim s^{1 / 3}
\label{eq:fractal_globule}
\end{equation}
Therefore, using the relationship between genomic distances $s$ and
contact frequencies $c$, obtained by the fitting of the linear
model, and the relationship between physical distances $d$ and genomic
distances $s$ (Equation~\ref{eq:fractal_globule}), we inferred a
mapping between contact frequencies $c$ and physical distances $d$
up to a factor. We arbitrarily set the distance of the two beads with
the smallest non-zero contact count $c_\text{min}$ to be at a
certain percentage $\beta$ of the nucleus diameter. Note that
$c_\text{min}$ is not necessarily equal 1 since the contact counts are
normalized. The $\beta$ parameter hence sets the scaling of the physical
distances. We then obtain:
\begin{equation}
\delta_{ij} = \frac{\beta 2 r}{c_\text{min}^{\alpha / 3}} c_{ij}^{\alpha / 3}
\end{equation}
where $r$ is the nucleus radius, and $\alpha$ the coefficient obtained
in the linear model fitting (range: 30--500~kb, $\alpha=-0.963$ for rings,
$\alpha = -1.124$ for trophozoites, $\alpha = -1.013$ for schizonts).
We set all distances larger than the nucleus diameter to this value.

\paragraph{Optimization: } Given the resulting physical wish distances, we
defined the following optimization problem to find a structure $\mathbf{X}
\in R^{3 \times n}$, where $n$ is the number of beads:
\begin{equation*}
\renewcommand{\arraystretch}{2}
\begin{array}{ccll}
\underset{\mathbf{X}}{\text{minimize}} & &
\underset{\delta_{ij} \in \mathcal{D}}{\sum} \frac{1}{\delta_{ij}^2}\big(d_{ij} - \delta_{ij}\big)^2 &\\
\text{subject to}
& & x_i^Tx_i \leq r_{\rm max}^2, \quad
& i = 1:n\\
& & d_{i, i+1} \leq b^{\rm max} , \quad
& i = \{1:n \;|\; {\rm chr}_i = {\rm chr}_{i+1}\}\\
\end{array}
\end{equation*}
where $d_{ij}$ is the Euclidean distance between beads $x_i$ and
$x_j$, $\mathcal{D} = \{ \delta_{ij} | \delta_{ij} \neq 0\}$ is
the set of non-zero wish distances, and $b^{\rm max}$ is defined below.

The constraints are as follows:
\begin{enumerate}
\item \emph{All loci must lie within a spherical nucleus centered on the origin. }
  Electron microscopy experiments show that the nucleus roughly resembles a
  sphere, with the radius depending on the stage of the organism.  In
  this work, we use a nuclear radius of $r = 350$~nm for the ring
  stage, $r=850$~nm for the trophozoite stage and $r=425$~nm for the
  schizont stage \citep{bannister:making, weiner:3d}.
\item \emph{Two adjacent loci must not to be too far apart.}
  1000~bp of chromatin occupies a distance between 6.6--9.1~nm
  \citep{berger:high}. Because we use 10~kb resolution, we set $b^\text{max} = 91$~nm.
\end{enumerate}

\paragraph{Initialization:}
We create a population of 100 independently optimized structures by initializing
$\textbf{X}$ randomly from a standard normal distribution.

\paragraph{Measuring similarities between structures:}

To compare pairs of structures ($X$, $Y$) we used the standard RMSD measure:

\begin{equation}
\text{RMSD} = \text{min}_{X^*} \sqrt{\frac{1}{n} \sum_{i = 0}^n (x^*_i - y_i) ^ 2}
\end{equation}
where $X^*$ is obtained by translating and rotating $X$. To compare structures
of different scale (e.g., different $\beta$ values), we seek, in addition of the
translation and rotation factor, the scaling factor that minimizes the RMSD
between structures.

Another similarity measure we use to compare two structures is the average difference
of their pairwise distance matrices (at 10 kb resolution), which we denote by
\textit{distance difference}:
\begin{equation}
\text{distance difference} = \frac{1}{n(n - 1)/2} \sum_{i > j} |d^X_{ij} - d^Y_{ij}|
\end{equation}
where $d^X$ and $d^Y$ are the Euclidean distance matrices of the structures
$X$ and $Y$.

\paragraph{Clustering the population of structures:}

In order to see whether the structures fall into discrete groups, we computed
the RMSD between pairs of structures and performed hierarchical clustering on 
the resulting $100\times100$ distance matrix for each stage 
(Supplementary Fig.~\ref*{suppfig:CHindices}).

\paragraph{Choosing the parameter $\beta$:}
As noted above, the parameter $\beta$ controls the scaling of the inferred 3D
structure. A small value of $\beta$ will yield a structure with a very dense
center, and a large value of $\beta$ will push all beads against the nuclear
envelope. The literature suggests that chromatin should abut the nuclear
envelope \citep{weiner:3d}. Assuming the chromatin should also occupy the
center of the nucleus, we ran the entire optimization multiple times, and we
selected a value of $\beta$ that yields a chromatin density as close as
possible to a uniform distribution.

This procedure required that we estimate the density of chromatin at a
distance $\ell$ from the center of the nucleus.  To do so, we first
created an intermediate function
\[
f(\ell) = \sum_{i = 1}^N g\left(\ell - \sqrt{x_i ^2 + y_i^2 + z_i^2}\right),
\]
where $g(\cdot)$ is a Gaussian ($\mu = 0$, $\sigma = 10$~nm).  The
standard deviation $\sigma$ of the Gaussian corresponds to the
uncertainty of the position of each bead. The estimated density $D(\ell)$
was then computed as a generalized histogram, using discretized distance bins
$\ell_i$. To ensure that the volume was constant for each bin, the bin spacings
were defined as $\ell_i = i^{1 / 3} \ell_1$, where we chose $\ell_1 = \frac{r}{3}$.
We then normalized the histogram to sum to one.

Let $D_i$ be the density of bin $i$ and let $n_{\text{bins}}$ be the
number of bins. To select $\beta$, we defined the scoring function

\begin{equation}
\text{score} = \sqrt{\sum_{i = 1}^{n_\text{bins}} \left(D_i -
\frac{1}{n_\text{bins}}\right)^2},
\end{equation}
which corresponds to the mean squared error between the estimated density and
the expected density. The resulting density scores are shown in Supplementary
Table~\ref*{table:density}, with the minimal value for each stage in boldface.

\subsubsection*{Eigenvalue decomposition and chromatin compartments}
\label{met:compartments}
To identify chromatin compartments, for each stage, we
carried out eigenvalue decomposition on the matrix of Euclidean
distances between locus pairs. For each chromosome we used the
intrachromosomal 3D distance matrix at a resolution of 10~kb, where
each 10~kb locus is represented by the 3D coordinate of its
midpoint. We then calculated the Spearman correlation between each
pair of rows of the 3D distance matrix and applied eigenvalue
decomposition (using the \emph{eig} function in MATLAB) to this
correlation matrix. The sign of the first eigenvector defined a
compartment assignment for each 10~kb locus at each stage. We also
aggregated all three stages and calculated a set of aggregate compartments
(Supplementary Fig.~\ref*{suppfig:perChrFigs}, fourth row of figures on each page)
which divided each chromosome into three main compartments (i.e.,
telomeric-centromeric-telomeric or left(L)-mid(M)-right(R)).

\subsubsection*{Kernel canonical correlation analysis}
\label{met:kCCA}

We used an approach based on kernel canonical correlation analysis (kCCA)
\citep{bach:kernel,vert:graph,vert:extracting} to extract gene expression profiles that
simultaneously capture the variance of the gene expression data and exhibit
coherence with respect to the 3D structure.

Let $\mathcal{G}$ be the set of $n$ genes. Each gene $g\in\mathcal{G}$ is
characterized by its log expression profile $e(g) = \left( e_1(g), \ldots,
e_p(g)\right) \in \mathbb{R}^p$ at $p$ timepoints
and by its position $x(g) \in \mathbb{R}^3$
in 3D
space. We assume that the set of gene expression profiles is mean centered and
unit variance scaled, i.e., $\sum_{g\in\mathcal{G}}e_i(g) = 0$ and
$\frac{1}{|\mathcal{G}|}\sum_{g\in\mathcal{G}} e_i(g)^2 = 1$ for
$i=1,\ldots,p$.

Let $v\in\mathbb{R}^p$ be a direction in the expression profile space. To
assess whether $v$ is representative of the observed expression profiles, we computed
the percentage of variance explained among the gene expression profiles once
they are projected onto $v$, defined by
\begin{equation}\label{eq:variance}
V(v) = \frac{\sum_{g \in \mathcal{G}} \left(v^T e(g)\right)^2}{\|v\|^2}\,.
\end{equation}
The larger $V(v)$ is, the more $v$ explains the differences between gene
expression profiles, and the more likely $v$ is to correspond to some
biological event which influences the expression of many genes. $V(v)$ is, for
example, maximized by principal component analysis.

Instead of just asking the profile $v$ to capture variance among gene expression, we
simultaneously asked it to exhibit coherence with respect to the 3D structure.
For that purpose, we defined for every $f\in\mathbb{R}^n$ a function $S(f)$
that quantifies how smoothly $f$ varies in 3D. $f$ can be thought of as a
vector of scores, one score being assigned to each gene.  Because we know the
3D coordinates of each gene we can imagine $f$ as a set of scores in 3D.
Following a standard approach in kernel methods \citep{scholkopf:learning}, we
quantified the smoothness of $f$ with the function
\begin{equation}\label{eq:smoothness}
S(f) = \frac{f^\top K^{-1}_{3D} f}{||f||^2}\,,
\end{equation}
where $K_{3D}$ is the $n\times n$ matrix whose $(i,j)$ entry is the Gaussian
kernel between genes $i$ and $j$, namely, $\exp\left(-||x(i) - x(j)||^2 /
2\sigma^2\right)$. The smaller $S(f)$ is, the more smoothly $f$ is distributed
in 3D.

We then combined the ideas of capturing variance (Equation~\ref{eq:variance}) and being
smooth in 3D (Equation~\ref{eq:smoothness}) by designing a joint objective function
over $v$ and $f$ to ensure that (i) $v$ captures a lot of variance, (ii) $f$
is smooth in 3D, and (iii) $f$ is maximally correlated with the vector
$\left(v^\top e(g)\right)_{g\in\mathcal{G}}$. In words, we aimed to ensure that
genes which are positively correlated with $v$ (and those which are negatively
correlated) tend to be co-localized in 3D. We designed the function by following
the approach of \cite{bach:kernel}, who show that $v$ and $f$
can be found by solving a kCCA problem equivalent to the following generalized
eigenvalue problem:
\[
\left(\begin{array}{cc}0 & K_E K_{3D} \\ K_{3D} K_E & 0\end{array}\right)
\left(\begin{array}{c}\alpha \\ \beta\end{array}\right)
= \rho
\left(\begin{array}{cc}(K_E+\delta I)^2 & 0 \\ 0 & (K_{3D}+\delta I)^2\end{array}\right)
\left(\begin{array}{c}\alpha \\ \beta\end{array}\right)\,,
\]
where $K_{E}$ is the $n \times n$ matrix whose $(i,j)$-th entry is $e(i)^\top
e(j)$, and $\delta$ is a small regularization parameter. Once we found the
generalized eigenvectors $(\alpha,\beta)^\top$, ranked by decreasing
eigenvalue $\rho$, we recovered a pair $(v,f)$ by $v = \sum_{g\in\mathcal{G}}
\alpha_g e(g)$ and $f=K_{3D} \beta$.

We computed the profiles for several values of $\sigma$ (0.01, 0.02, 0.05,
0.1) and $\delta$ (0.01, 0.02, 0.04, 0.06) and obtained highly correlated results
(correlation $>0.99$ for all pairs of profiles). Therefore, we chose
$\sigma = 0.01$ and $\delta = 0.02$ for the rest of the analysis.

\subsubsection*{Gene set enrichment analysis}
\label{met:GSEA}
To detect set of genes highly or poorly correlated kCCA profiles, we apply gene
set enrichment analysis (GSEA) \citep{subramanian:gene}. Unlike a traditional
GO term enrichment analysis, this method takes as input a ranked list of genes
rather than a set of genes; hence, GSEA takes full advantage of the results of
the kCCA.  The procedure detects sets of genes enriched at the top or at the
bottom of the ranked list of genes. We applied GSEA to the ranked list of projections
of expression profiles on the first and second extracted profile. Corresponding
p-values were computed using $4,000$ permutations. We also used GSEA in our
comparison gene sets that are involved in contacts that are specific to either
one stage or two out of three stages.


\subsubsection*{Volume exclusion model}
\label{met:volume-exclusion}

Following the methodology of \cite{tjong:physical}, we
constructed a population of three-dimensional structures by modeling
chromosomes as random configurations subject to the following
constraints:
\begin{enumerate}
\item Each chromosome is modeled as a series of $N$ beads spaced 3.2~kb apart,
  with consecutive beads restrained to be 30~nm apart.
\item Overlaps between beads are prevented by imposing a volume
  exclusion constraint for all pairs of beads.
\item All chromosomes lie within a spherical nucleus of a specified
  radius.
\item All centromeres are colocalized in a small sphere of radius
  50~nm abutting the nuclear envelope.
\item All telomeres are located within 50~nm of the nuclear envelope.
\end{enumerate}
We formulated an optimization problem that includes, in addition to the
constraints, a penalty term that accounts for chromatin stiffness by
placing an angular restraint between three consecutive beads:
\begin{equation}
\frac{1}{2} k_{\text{angle}} \sum^{N - 2}_{i = 1} \left( 1 - \frac{x_{i + 1} -
x_i}{\|x_{i + 1} - x_i\|} \cdot \frac{x_{i + 2} - x_{i + 1}}{\|x_{i + 2} -
  x_{i + 1}\|} \right)^2\,,
\end{equation}
where $x_i \in \mathbb{R}^3$ is the coordinate vector of bead $i$.
We used the Integrated Modeling Platform (IMP) \citep{bau:three-dimensional}
to generate 5,000 budding yeast structures with a nuclear radius of
1000~nm, and 5,000 {\em Plasmodium} structures for each of the three
stages with nuclear radii of 350~nm, 850~nm and 425~nm, respectively.

Following \cite{tjong:physical}, we used the population of structures to
generate a volume exclusion (VE) contact frequency matrix $C$,
considering that two beads are in contact when they are $\leq$45~nm apart.
The contact frequency matrix was then aggregated to a resolution of
32~kb and normalized following the ICE procedure as described above,
resulting in a contact frequency matrix
$c_{ij}^{VE}$ for $i,j=1,\ldots,N$ according to the VE model.

In order to compare the VE contact matrix to experimental Hi-C data,
we similarly computed the Hi-C contact count matrix at a resolution of
3.2~kb, aggregated it at 32~kb, and normalized the same way as the VE
contact frequency matrix to get a Hi-C contact matrix
$c_{ij}^{HIC}$ for $i,j=1,\ldots,N$.

We then compared both matrices by computing the row-based Pearson correlation
\citep{tjong:physical} defined as the average Pearson correlation between their rows.
\begin{equation}\label{eq:pearson}
\frac{1}{N} \sum_{i=1}^{N} \frac{
  N \sum_{j \neq i}^N c_{ij}^{\text{HIC}}{c}_{ij}^{\text{VE}} -
  \sum_{j \neq i}^N {c}_{ij}^{\text{HIC}} \sum_{j \neq i}^N {c}_{ij}^{\text{VE}}
}{
\sqrt{N \sum_{j \neq i }^N({c}_{ij}^{\text{HIC}})^2 - (\sum_{j \neq i}^N
{c}_{ij}^{\text{HIC}})^2}
\sqrt{N \sum_{j \neq i }^N({c}_{ij}^{\text{VE}})^2 - (\sum_{j \neq i}^N
{c}_{ij}^{\text{VE}})^2}}\,.
\end{equation}
Furthermore, we also computed a \emph{normalized} row-based Pearson correlation
between the matrices by replacing the counts $c^{VE}_{ij}$ and $c^{HIC}_{ij}$
in~(Equation~\ref{eq:pearson}) by their ratio to an expected count $c^E_{ij}$ that we
would expect if there was no structural information in the matrix, besides
the obvious decrease of contacts between loci at increasing genomic
distance. To estimate the
expected frequencies $c^E_{ij}$ used to define the ratios, we fit an
isotonic regression to the mapping between genomic distance and the average
contact frequency at this genomic distance. The isotonic regression
allows us
to fit a non-increasing mapping between genomic distance and contact
frequency, thus correcting the effect of enrichment of contact frequencies
at chromosome ends. This mapping allowed us to define $e_{ij}^E$ as the expected
count corresponding to the genomic distance between loci $i$ and $j$ in the
case of intrachromosomal contacts, and to the genome-wide average of
inter chromosomal counts in case of interchromosomal contacts.



