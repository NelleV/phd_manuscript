\documentclass{article}
\usepackage[margin=1in]{geometry}

\begin{document}

\section{RPKM}

We first briefly recall the definition of RPKM (reads per kilobase per
million mapped reads), widely used to quantify the expression level
of genes in RNA-seq experiments. Let
\begin{itemize}
\item $l$ be the length (in bp) of the coding region of a gene,
\item $c$ be the number of mappable reads that fall onto the gene's
  exon, and
\item $m$ be the total number of mapped reads.
\end{itemize}
Then the RPKM of the gene is:
$$
RPKM = \frac{c}{ml}\times \alpha\,,
$$
with $\alpha=10^9$. Note that when when $l$ is expressed in kilobases, and $m$ in millions, then the constant $10^9$ disappears since the RPKM is then just:
$$
RPKM = \frac{c}{(m \times 10^{-6})\times (l \times 10^{-3})}\,.
$$

Why do we need the constant $\alpha=10^9$? It seems to be a happy
coincidence that (i) it leads to simple computation, since genomic
lengths are naturally expressed in kb and read counts in millions, and
(ii) it leads to values easy to manipulate, typically between $0$ and
$100$ for human genes. Some ``rules of thumb'' are even sometimes
proposed, such as $RPKM<10$ meaning low expression, $10<RPKM<30$
meaning normal expression, and $RPKM>100$ meaning highly expressed.

In order to understand why $\alpha=10^9$ leads to such ``normal''
values, we can for example compute the average RPKM over the
genes. Assume for simplicity that the genome is made of $n$ genes of
length $l$, thus corresponding to a total coding size on the genome of
$L_c=nl$. The average RPKM is then
\[
<RPKM> = \frac{1}{n} \sum_{g} \frac{c_g}{ml} \times \alpha = \frac{\alpha}{nl} = \frac{\alpha}{L_c}\,,
\]
where $c_g$ denotes the number of reads falling onto gene $g$, and
where we used the fact that the total read count is $m=\sum_g
c_g$. For example, for the human genome, where coding sequences
represent about $2\%$ of the $3\times 10^9$ genome, we have $L_c = 6\times 10^7$
and therefore an average RPKM of order $15$ when we take
$\alpha=10^9$. This is observed empirically, and justifies the ``rule
of thumb'' that $RPKM\sim 15$ means normal expression.

However, this observation also shows that RPKM is not normalized for
genome length, but scales inversely proportional to the total coding
length for an organism. This means in particular that any ``rule of
thumb'' for RPKM values is limited to a given genome, and should be
rescaled to be adapted to a different species. Of course, an easy fix
would be to divide by the total coding length in the definition of
RPKM.

In summary:
\begin{itemize}
\item RPKM normalizes the counts by the total number of mapped reads
  (to be able to compare different experiments) and by the gene length
  (to be able to compare different genes within an experiment).
\item The $\alpha=10^9$ constant in the definition of RPKM is
  convenient because (i) it makes the computation easy to perform
  (express the length in kb and the total number of mapped reads in
  millions, then divide), and (ii) it leads to average values of about
  $15$ for the human genome, which is easy to remember.
\item Unfortunately, RPKM is not normalized across species: for a
  fixed constant $\alpha$, RPKM scales as $1/L_c$, the inverse of the
  total coding length. This means that for a genome with half as many
  coding bases as the human genome, the ``normal'' RPKM is $30$
  instead of $15$. A normalization by $L_c$ in the definition of RPKM
  would have been useful to address this issue.
\end{itemize}

\section{$RPK^2M$}

For Hi-C experiments, we need to normalize the number of reads linking
two loci by two factors: (1) the total number of mapped reads (to be
able to compare different experiments), and (2) the ``surface" of the
mapped region, i.e., the product of the loci lengths. Indeed, read
count is expected to increase linearly with respect to each of the
loci's length. More precisely, denoting by $l_1$ and $l_2$ the lengths
(in bp) of the two interacting loci, we define:
\[
RPK^2M = \frac{c}{ml_1l_2}\times \beta\,,
\]
where $\beta$ is a scaling factor.
Note that the name $RPK^2M$ suggests taking $\beta=10^{12}$, corresponding to expressing $m$ in millions and each length in kilobases, and leading to the simple division:
\[
RPK^2M = \frac{c}{(m \times 10^{-6})\times (l_1 \times 10^{-3}) \times (l_2 \times 10^{-3})}\,.
\]

Is $\beta=10^{12}$ a good choice? In terms of usability, the answer is
yes, because it is common to express length in kb and read counts in
millions, leading to a simple division. Contrary to RPKM, however, it
is not obvious that the resulting $RPK^2 M$ values will fall in the
interval $[0,100]$. To quantify this, we can compute the average
$RPK^2M$ of all pairs of loci. We suppose that the genome is made of
$n$ loci of length $l$, corresponding to a genome size $L_G = nl$. The
average $RPK^2M$ over all pairs of loci is then
\[
<RPK^2M> = \frac{1}{n^2} \sum_{i,j=1}^n \frac{c_{i,j}}{ml^2}\times \beta = \frac{\beta}{n^2 l^2} = \frac{\beta}{L_G^2}\,,
\]
where we used the fact that the total count is $m=\sum_{i,j=1}^n
c_{i,j}$. We now see that the mean $RPK^2M$ scales as the inverse of
the square of the genome size. For example, for {\em P.\ falciparum} with
$L_G = 25 \times 10^6$, keeping the default $\beta=10^{12}$ leads to an
average $RPK^2M$ of about $1.6\times10^{-3}$; for the human genome
($L_G=3\times 10^9$) we get $<RPK^2M> \sim 10^{-7}$. These values look a bit
small, so one may decide to choose another, larger constant $\beta$
(but then the name $RPK^2M$ loses its relevance).

However, a caveat with this computation is that most interactions take
place within nearby loci along the genome; hence, the global average
over all pairs of loci may not be a very relevant value to define the
scaling. In other words, the interaction matrix is rather sparse,
except near the diagonal. An alternative quantity of interest could
therefore be the mean value taken near the diagonal. Suppose for
example that we know that ``most'' interactions (e.g., a fraction
$0<\gamma<1$ of all reads) fall between loci at distance at most $D$
(in bp) of each other. Then a locus mostly interacts with its $D/l$
nearest loci.  We can estimate the average $RPK^2M$ among the loci
involved in this interactions as follows:
\[
<RPK^2M> = \frac{1}{n\times D/l} \sum_{i=1}^n \sum_{j=i}^{i+D/l}  \frac{c_{i,j}}{ml^2}\times \beta = \frac{\gamma \beta}{nDl} = \frac{\gamma\beta}{D L_G}\,,
\]
where we used the fact that $ \sum_{i=1}^n \sum_{j=i}^{i+D/l} c_{i,j}
= \gamma m$.  Say, for example, that $50\%$ of counts are found
between loci less than $10^5$ bp apart; then the average $RPK^2M$ ``near the
diagonal'' for {\em P.\ falciparum} becomes $0.5\times 10^{12} / (10^5
\times 2.5 \times 10^7) = 0.2$, which starts looking easier to plot on
a graph.

In summary:
\begin{itemize}
\item The name $RPK^2M$, which is good for ``marketing,'' suggests
  taking implicitly the constant $\beta=10^{12}$. Similar to $RPKM$,
  this approach leads to a computation that is easy to perform (simple
  division without scaling constants when distances are expressed in
  kilobasepairs and the total number of mappable reads in
  millions). However, contrary to RPKM, one should expect ``small''
  values.
\item Like RPKM, $RPK^2M$ is not normalized for genome length (its
  average value scales inversely proportional to the square of the
  genome length), meaning that comparison of values across species
  with different genome lengths are meaning less. As easy fix would be
  to divide by the genome length.
\item The important aspect of $RPK^2M$ is the square in $K^2$, meaning
  that the unit of $RPK^2M$ is the inverse of a square length (in
  contrast to RPKM, which is the inverse of a length).
\item At some point there was a discussion about whether we should not
  consider some "$RPK^2M^2$" to set the scaling constant to
  $\beta=10^{18}$ instead of $10^9$.  Although changing the constant
  could make sense to have values in a better range, I think
  $RPK^2M^2$ is definitively not a good name because it suggests that
  the unit is the inverse of a square distance times the inverse of a
  count, which is wrong. If we divide by the square of the total read
  count, then we are not invariant anymore to changes in total read
  counts. To change the constant, we should instead change the $M$
  (which means millions) by something else, typically $RPK^2G$ for
  $\beta=10^{15}$, or $RPK^2T$ for $\beta=10^{18}$.
\item The choice of the scaling factor $\beta$ remains arbitrary
  anyway, so choosing between $RPK^2M$, $RPK^2G$ or $RPK^2T$ is an
  aesthetic question. I'd rather stay with $RPK^2M$ for its similarity
  with $RPKM$ and its simplicity to compute, with the caveat that it
  could take on small values.
\end{itemize}


\end{document}
