
% Thesis Abstract -----------------------------------------------------


%\begin{abstractslong}    %uncommenting this line, gives a different abstract heading
\begin{abstracts}        %this creates the heading for the abstract page

The structure of DNA, chromosomes and genome organization is a topic that has
fascinated the field of biology for many years. Most research focused on the
one-dimensional structure of the genome, studying the linear organizations of
genes and genomes and their link with gene expression and regulation,
splicing, DNA methylation, \dots. Yet, spatial and temporal three-dimensional
(3D) genome architecture is also thought to play an important role in many
genomic functions.

Chromosome conformation capture (3C) based methods, coupled with next
generation sequencing (NGS), allow the measurement, in a single experiment, of
genome wide physical interactions between pairs of loci, thus enabling to
unravel the secrets behind 3D organization of genomes. These new technologies
have paved the way towards a systematic and genome wide analysis of how DNA
folds into the nucleus and opened new avenues to understanding many biological
process, such as gene regulation, DNA replication and repair, somatic copy
number alterations and epigenetic changes. Yet, 3C technologies, as any new
biotechnology, now poses important computational and theoretical challenges
for which mathematically well grounded methods need to be developped.



\end{abstracts}
%\end{abstractlongs}


% ---------------------------------------------------------------------- 
