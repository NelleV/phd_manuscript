
% Thesis Abstract -----------------------------------------------------


%\begin{abstractslong}    %uncommenting this line, gives a different abstract heading
\begin{abstracts}        %this creates the heading for the abstract page

The structure of DNA, chromosomes and genome organization is a topic that has
fascinated the field of biology for many years. Most research focused on the
one-dimensional structure of the genome, studying the linear organizations of
genes and genomes and their link with gene expression and regulation,
splicing, DNA methylation, \dots. Yet, spatial and temporal three-dimensional
(3D) genome architecture is also thought to play an important role in many
genomic functions.

Chromosome conformation capture (3C) based methods, coupled with next
generation sequencing (NGS), allow the measurement, in a single experiment, of
genome wide physical interactions between pairs of loci, thus enabling to
unravel the secrets behind 3D organization of genomes. These new technologies
have paved the way towards a systematic and genome wide analysis of how DNA
folds into the nucleus and opened new avenues to understanding many biological
process, such as gene regulation, DNA replication and repair, somatic copy
number alterations and epigenetic changes. Yet, 3C technologies, as any new
biotechnology, now poses important computational and theoretical challenges
for which mathematically well grounded methods need to be developped.

In this thesis, we attempt to address some of the challenges faced while
analysing such data.

The first chapter is dedicated to developping a robust and accurate method to
infer a 3D model of the genome from Hi-C data. Previous methods often
formulated the inference as an optimization problem akin to {\em
multidimensional scaling } (MDS) based on an {\em ad hoc} conversion of
contact counts into euclidean {\em wish distances}. Chromosomes are
modeled with a beads
on a string model, and the methods attempt to place the beads in a 3D
euclidean space to fullfill an number of, often non convex, constraints and
such that the pairwise distance between each pairs of beads is as closely
as possible to the corresponding {\em wish distance}. These
approaches rely on dubious hypothesis to convert contact counts into {\em wish
distances}, challenging the accuracy of the final 3D model. Another limitation
is the MDS formulation which is only intuitively motivated, and not grounded
on a clear statistical model. Our method models contact counts as a Poisson
distribution where the intensity is proportional to the spatial distance
between elements interacting. We then formulate the 3D structure inference as
a maximum likelihood problem. We demonstrate that our method infers robust and
stable models across resolution and datasets.

The second chapter focuses on the genome architecture of the {\em P.
falciparum}, a small parasite responsible for the deadliest and most virulent
form of human malaria. This project was biologically driven and aimed at
understanding whether and how the 3D structure of the genome related to gene
expression and regulation at different time points in the complex life cycle
of the parasite. In collaboration with Le Roch lab and Noble lab, we built 3D
models of the genome at three time points which resulted in a complex genome
architecture indicative of a strong association between the spatial genome and
gene expression.

The last chapter tackles a very different question, also based on 3C-based
data. Initially developped to probe the 3D architecture of the chromosomes,
Hi-C and related techniques have recently been re-purposed for diverse
applications: \textit{de novo} genome assembly, deconvolution of metagenomic
samples and genome annotations. We describe in this chapter a novel method,
Centurion, that jointly infers the locations of all centromeres in a single
genome from Hi-C data, using the centromeres' tendency to strongly colocalize
in the nucleus.  Indeed, centromeres are essential for proper chromosome
segregation, yet, despite extensive research, centromere locations are unknown
for many yeast species. We demonstrate the robustness of our approach on
datasets with low and high coverage on well annotated organisms. We then
predict centromere coordinates for 6 yeast species that currently lack those
annotations.

\end{abstracts}
%\end{abstractlongs}


% ---------------------------------------------------------------------- 
