\begin{resumes}

La structure de l'ADN, des chromosomes et l'organisation du genome sont des
sujets fascinants du monde de la biologie. Le gros de la recherche s'est
concentré sur la structure uni dimensionnelle du génome, étudiant comment les
gènes et les chromosomes sont organisés, et le lien entre l'organisation
unidimensionnelle et la régulation des gènes, l'épissage, la méthylation,
\dots Cependant, le génome est avant tout organisé dans un espace euclidéen
tridimensionnelle, et cette structure 3D, bien que moins étudiée, joue, elle
aussi, un rôle important dans la fonction génomique de la cellule.

La capture de la conformation des chromosomes (3C) et les méthodes qui en sont
dérivées, associées avec le séquençage à haut débit (NGS) permettent désormais
la mesure en une seule expérience des interactions physiques entre pair de
loci sur tout le génome, permettant ainsi aux chercheurs de découvrir les
secrets derrière l'organisation des génomes. Ces nouvelles technologies
ouvrent la voie à des études systématiques et globales sur le repliement de
l'ADN dans le noyau et ouvrent la porte pour mieux comprendre un certain
nombre de processus biologique, comme la régulation des gènes, la replication
et la réparation de l'ADN, les altérations du nombre de copies somatiques ainsi
que les changements épigénétiques. Cependant, ces nouvelles méthodes 3C, comme
toute nouvelle technologie, sont accompagnées de nombreux défis
computationnelles et théoriques.

Dans cette thèse, nous cherchons à relever un certain nombre de ces défis.

Le premier chapitre est dédié au développement d'une méthode robuste et
précise pour inférer un modèle tri-dimensionnelle à partir de données Hi-C.
Les méthodes développées précédemment formulent souvent ce problème
d'inférence comme un problème d'optimization basé sur le {\em positionnement
multidimensionel} (en anglais {\em multidimensional scaling}) (MDS),
reposant sur une
dérivation {\em ad hoc} des fréquences d'interaction en distances
euclidéennes. Les chromosomes sont modélisées comme des colliers de perles,
lesquelles doivent être placées dans un espace euclidéen de dimension 3 de tel
sorte à respecter un certain nombre de contraintes (souvent non convexes) et
de manière à positionner les perles telles que les distances entre elles
soient les plus proches des distances dérivées des fréquences d'interaction.
Ces approches reposent sur des hypothèses douteuses pour transformer
fréquences d'interaction en distances euclidéenne, soulevant ainsi un doute
sur la validité du modèle final obtenu.
Une autre limitation de ces méthodes est la
formulation du problème d'inférence sous forme MDS, justifiée non pas par un
modèle statistique, mais par uniquement par l'intuition.
Pour pallier à ces problème, notre méthode modèlise les fréquences
d'interaction comme une
distribution de Poisson dont l'intensité est une fonction de la distance
euclideénne entre paire de loci: nous formulons ainsi l'inférence de la
structure 3D comme un problème de maximum de vraisemblance. Nous montrons que
notre méthode infère des modèles plus robustes et plus stables selon les
données et les résolutions.

Le deuxième chapitre est consacré à l'étude de l'architecture du {\em P.
falciparum}, un petit parasite responsable de la forme la plus virulente et
mortelle de la malaria. Ce projet, dont le but était de répondre à une
question biologique, avait pour but de comprendre comment l'architecture 3D du
génome du {\em P. falciparum} est lié à l'expression et la régulation des
gènes à différent moments du cycle cellulaire du parasite. En collaboration
avec les équipes de Karine Le Roch et de William Noble, spécialisées
respectivement dans l'étude du {\em P. falciparum}, et dans le développement
de méthode computationnelle pour étudier, entre autre, la structure 3D du
génome, nous avons construit des modèles de l'organisation du génome à trois
moments du cycle cellulaire du parasite. Ceux-ci révèlent que le génome est
replié dans le noyau dans une structure complexe, où de nombreux 
nombreux éléments génomiques colocalizent: centromères, télomères, ADN
ribosomal, famille
de gènes, \dots Cette architecture indique une forte association entre
l'organisation spatiale du génome et l'expression des gènes.

Le dernier chapitre réponds à une question très différente, mais aussi lié à
l'étude des données 3C. Celles-ci, initiallement développée pour étudier la
structure tri-dimensionnelle du génome, ont été récemment utilisées pour des
applications très diverses: l'assemblage de génome {\em de novo}, la
déconvolution d'échantillon méta génomique et l'annotation de génome. Nous
décrivons dans ce chapitre une nouvelle méthode, Centurion, qui infère
jointement la position de tous les centromères d'un organisme, en utilisant la
propriété qu'ont les centromères à colocaliser dans le noyau. Cette méthode
est donc une alternative aux méthodes de détection de centromères classiques,
qui, malgré des années de recherche, n'ont pu identifier la position des
centromères dans un certain nombre d'espèces de levure. Nous démontrons dans ce
projet la robustesse et la précision de notre approche sur des jeux de données
à haute comme à basse couverture. Nous prédisons par ailleurs la position des
centromères dans 6 espèces qui n'avaient pour l'instant aucune annotation.

J'ai par ailleurs au cours de ma thèse travaillé sur un certain nombre de
projets pour lesquels ma contribution a été mineure et que je ne décrirai pas
dans ce manuscript, mais dont les papiers peuvent être trouvés en appendice.
Le premier projet consiste au développement d'un nouvel outil permettant le
pre processing des données Hi-C afin de construire et de normaliser les cartes de fréquences
d'interaction des reads au cartes de fréquences à partir des données brutes de
séquençage. Ma contribution a été l'implémentation en python d'une version
optimisée à la fois en ram et en temps de calcul de la normalisation. Cette
implémentation, bien que très simple et non parallelisée, est à
notre connaissance la plus performante existant à l'heure actuelle. Le
deuxième papier est une revue de l'épigénétique du {\em P. falciparum} suite à
notre premier publication sur le sujet. Le troisième papier étends la méthode
Hi-C afin de détecter, en plus des pairs d'interactions, des interactions
entre trois et quatre éléments. Ma contribution à ce dernier projet a été le
développement d'une méthode permettant l'inférence de la structure 3D de
génomes polyploides.

\end{resumes}


