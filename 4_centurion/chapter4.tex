% this file is called up by thesis.tex
% content in this file will be fed into the main document

\chapter[Identification of centromere locations using Hi-C]{
Accurate identification of centromere locations in yeast genomes using Hi-C}
\graphicspath{{4_centurion/figures/}}

\begin{work}

This chapter has been published in a slightly modified form in
\citep{varoquaux:accurate}, as joint work with Ivan Liachko, Josh Burton,
Ferhat Ay, Jay Shendure, Maitreya Dunham, Jean-Philippe Vert and Bill Noble.

\end{work}

% Résumé en francais

\begin{abstract}{Résumé}

Les centromères sont des élèments génomiques permettant une une segrégation
correcte des chromosomes lors de la division cellulaire. Malgré leur
importance dans le développement de la cellule et l'important effort de
recherche qui leur ait dédié, la position des centromères chez la levure est
souvent, même de nos jours, difficile à inférer, et est inconnue chez la
pluparts des espèces. Récemment, le protocol de capture de conformation des
chromosome Hi-C a été reciblé pour diverses applications: séquençage
\textit{de novo} de génome, deconvolution d'échantillon métagénomique, et
inférence de la position des centromères chez la levure. Nous décrivons ici
une méthode, nommée Centurion, permettant l'inférence jointe de la position de
tous les centromères dans un génome à partir de données Hi-C en exploitant la
propriété qu'ont les centromères à colocalizer dans le noyau de certains
organismes. Nous démontrons dans un premier temps la précision de notre
algorithme, en identifiant les centromères dans des données Hi-C à haute
couverture chez la levure de boulanger \textit{S. cerevisiae} et le parasite
responsable de la malaria \textit{P. falciparum}. Nous utilisons ensuite
Centurion pour prédire la position des centromères dans 14 autres espèces de
levure d'un échantillon métagénomique. Parmi tous les organismes que nous
étudions, Centurion prédit 89\% de centromeres à moins de 5~kb de leur
position. Nous démontrons par ailleurs la robustesse de notre approche sur des
jeux de données à faible couverture. Finalement, nous inférons la position
des centromeres dans 6 espèces qui n'ont pour l'instant aucune annotation. Ces
résultats montrent que Centurion peut être utilisé pour l'identification de
centromères pour différentes espèces de levures, ainsi que d'autres
organismes.

\end{abstract}
% Abstract

\begin{abstract}{Abstract}
Centromeres are essential for proper chromosome segregation. Despite extensive
research, centromere locations in yeast genomes remain difficult to infer, and
in most species they are still unknown. Recently, the chromatin conformation
capture assay, Hi-C, has been re-purposed for diverse applications, including
\textit{de novo} genome assembly, deconvolution of metagenomic samples, and
inference of centromere locations. We describe a method, Centurion, that
jointly infers the locations of all centromeres in a single genome from Hi-C
data by exploiting the centromeres' tendency to cluster in 3D space. We first
demonstrate the accuracy of Centurion in identifying known centromere
locations from high coverage Hi-C data of budding yeast and a human malaria
parasite. We then use Centurion to infer centromere locations in 14 yeast
species. Across all microbes that we consider, Centurion predicts 89\% of
centromeres within 5~kb of their known locations. We also demonstrate the
robustness of the approach in datasets with low sequencing depth. Finally, we
predict centromere coordinates for six yeast species that currently lack
centromere annotations. These results show that Centurion can be used for
centromere identification for diverse species of yeast and possibly other
microorganisms.
\end{abstract}

\section{Introduction}

Centromeres are chromosomal regions whose function enables faithful chromosome
segregation via formation of the kinetochore \citep{Bloom:centromeric}. These
elements are also key regulators of genome stability \citep{Feng:centromere}
and replication timing \citep{koren:epigenetically, pohl:functional}. In
animal and plant genomes, centromeres are large heterochromatic zones, but
many yeast species have {\em point centromeres}, which are sequence
elements as small as 125~bp \citep{cottarel:125-base-pair}. The relative
simplicity of yeast centromeres has allowed their functional dissection, and
the abundance of sequenced yeast species has shed light on the evolution of
centromeric elements across hundreds of millions of years of evolution
\citep{Gordon:mechanisms}.

\enlargethispage{-65.1pt}

The Hemiascomycetes yeasts comprise a highly important taxon of model
organisms in genetics and genomics \citep{dujon:yeast,
hittinger:saccharomyces}, and some are crucial in biotechnology applications
such as recombinant protein expression \citep{boer:yeast}. Most yeast plasmid
expression systems are dependent on locating and identifying yeast centromeres
because they confer the property of stable segregation to episomal plasmids
\citep{murray:pedigree}. However, efforts to annotate yeast centromeres are
hindered by the extraordinary diversity among species \citep{Malik:major}.
Mapping centromeres in diverse species has been attempted, usually through
phylogenetic tools \citep{Gordon:mechanisms, souciet:comparative} or chromatin
immunoprecipitation \citep{Lefrancois:efficient}. However, both approaches
have drawbacks, the former due to the divergence of underlying functional
motifs and the latter due to non-specific signal. A method of mapping
centromeres that does not rely on evolutionary predictions or rare protein-DNA
interactions would therefore be useful for identifying centromeres in novel
species. These new centromere sequences could then be used, for example, to build new
plasmid-based strain engineering tools in species important for
research and biotechnology.

Chromosome conformation capture tools such as Hi-C and related protocols use
proximity ligation and massively parallel sequencing to probe the
three-dimensional architecture of chromosomes within the genome
\citep{lieberman-aiden:comprehensive, kalhor:genome, duan:three}. Hi-C and
related techniques create a {\em contact map}, consisting of a matrix of
genome-wide interaction counts between pairs of loci. Contact maps have
recently been shown to contain long-range contiguity information: Hi-C has
been used in the scaffolding of \textit{de novo} genome assemblies
\citep{burton:chromosome, kaplan:high-throughput}, molecular haplotyping
\citep{selvaraj:whole-genome}, and metagenomic deconvolution
\citep{burton:species-level, beitel:strain}. These methods have also paved the
way for a more systematic analysis of genome architecture, including
long-range gene regulation and chromatin architecture \citep{nora:spatial,
dixon:topological, mizuguchi:cohesin}. These advances raise the possibility
that contact maps might be used to determine the location of
subchromosomal genomic structures such as centromeres and nucleoli.

A recent study attempted to map centromere locations using Hi-C contact
probability maps \citep{marie-nelly:filling}. This approach exploits the
strong architectural features of yeast genomes to determine centromere
positions and rDNA clusters in \textit{Saccharomyces cerevisiae},
\textit{Naumovozyma castellii}, \textit{Nuraishia capsulata}, and
\textit{Debaryomyces hansenii}. 
In yeasts, centromeres are tethered by microtubules to the spindle
pole body, leading to centromere clustering \citep{mizuguchi:cohesin}.
Similar clustering is also present in other organisms, such as the
parasite \textit{Plasmodium falciparum} and the plant
\textit{Arabidopsis thaliana} \citep{ay:three-dimensional,
  feng:genome-wide}.
The clustering of elements creates a
distinct peak of interactions between chromosomes in the \textit{trans} Hi-C
matrix, and an X-shape in the \textit{cis}-elements of the inter-chromosomal
contact counts pearson correlation matrix.
\citet{marie-nelly:filling} exploit this
X-shape structure in \textit{trans} contact counts correlation
matrices to first detect a 40~kb window containing each centromere.
In a subsequent step,
they carve out 40~kb-by-40~kb windows of contact counts
for each pair of centromeres and refine
the prediction by fitting a Gaussian on the sum of \textit{trans} elements of
these windows, a procedure similar to those used for single molecule
localization or high resolution microscopy \citep{Ober:localization}. However,
this method has several limitations. First, the procedure relies on the
correct pre-localization of candidate centromeres. This step fails when other
sequences also colocalize (for instance, rDNA sequences). Second, the last
step of the procedure collapses the data of several \textit{trans} interaction
windows into a 1D profile and calls the different centromeres independently
from each other, thus potentially losing some valuable information.

Here we propose a novel method, Centurion, that jointly
calls all centromeres in a genome-wide Hi-C contact map. The key idea is that
a joint optimization can effectively exploit the clustering of centromeres in
3D. We first compare our method to the one described by
Marie-Nelly \textit{et al}.\ on four publicly
available high-resolution Hi-C
contact maps (\textit{S.\ cerevisiae} \citep{duan:genome-wide} and three stages
of \textit{P.\ falciparum} \citep{ay:three-dimensional}). This comparison
demonstrates that Centurion infers centromere positions more accurately than
the previously published method. We then apply our method to Hi-C data from 14
diverse yeast species \citep{burton:species-level}, yielding high-resolution
centromere location predictions for each chromosome in each species. For the
eight species that already have centromere annotations available, our
predictions match very closely with the existing calls. For species with
as-yet uncharacterized centromeres, our predictions will serve as the basis
for targeted experimental validation and could be used to create new plasmid
tools in these yeasts. Our results suggest that Centurion has great potential
to identify the centromere locations of many yeasts for which standard
techniques have failed to date. Furthermore, we demonstrate that Centurion
works well even with very limited sequencing depth Hi-C libraries generated
from pooled samples, making it a practical as well as powerful tool to use on
single microorganisms and metagenomic mixtures. Centurion is freely available
as open source software at
\href{http://cbio.ensmp.fr/centurion}{http://cbio.ensmp.fr/centurion}.

\section{Method}

\begin{figure*}
\includegraphics[width=\linewidth]{figure_1.pdf}
\caption{\textbf{Outline of Centurion's computational workflow}
\textbf{1.} Paired-end Hi-C reads are mapped and filtered to produce
genome-wide contact maps (see Methods). 
\textbf{2.} Contact maps are normalized to correct for technical and
experimental biases \citep{imakaev:iterative}.
\textbf{3.} Peaks in marginalized \textit{trans} contact counts are identified
as candidate centromere locations.
\textbf{4.} If necessary, a heuristic reduces the number of centromere
candidates that will be used to initialize the joint optimization. 
\textbf{5.} A joint optimization procedure finds the best set of centromere
coordinates, one per chromosome, minimizing the squared distance between the
2D Gaussian fits and the observed \textit{trans} contact counts. 
\textbf{6.} For organisms with known centromere locations, the accuracy of
predicted centromere locations is evaluated; otherwise, the method provides \textit{de
novo} centromere calls. 
}
\label{fig:methods}
\end{figure*}



\subsection{Single organism Hi-C data}
We use Hi-C data gathered in two previous studies: an asynchronous budding yeast
(\textit{S. cerevisiae}) sample \citep{duan:three} and three different stages of
the human malaria parasite \textit{P. falciparum} \citep{ay:three-dimensional}.
For the budding yeast Hi-C data we download and use the files HindIII + MspI
(intra and inter)
from \url{http://noble.gs.washington.edu/proj/yeast-architecture/sup.html}.
For the three stages of \textit{P. falciparum} we download and use the Hi-C raw contact counts
at 10~kb resolution from GEO archive (Accession codes: GSM1215592, GSM1215593, GSM1215594).

\subsection{Metagenomic Hi-C data}
For Hi-C data from metagenomic samples we use the two synthetic mixtures (M-Y, M-3D)
generated in \citep{burton:species-level}. We also perform additional
sequencing of the M-3D sample using two restriction enzymes that cut more frequently than
the 6-bp cutters HindIII and NcoI used in the original publication. We perform these
additional Hi-C experiments exactly as described in \citep{burton:species-level}
with the exception that we use Sau3AI (a 4-bp cutter that recognizes ``GATC'') and AflIII
(a 6-bp cutter that recognizes ``ACRYGT'') to fragment the DNA. We then combine the reads
from these two libraries (Sau3AI and AflIII) to produce
Hi-C contact maps.

We process the Hi-C libraries from these metagenomic samples in a
similar fashion to the 
Hi-C data from the above mentioned single organism samples, with the exception of two differences.
First, we map the reads to a meta-reference genome that concatenates the reference genomes
of all the organisms in the corresponding sample. This mapping strategy discards contacts
which cannot be uniquely assigned to a single organism, thereby reducing contamination
between contact maps. Second, because of the longer read lengths
for the metagenomic libraries compared to single organisms (80--101~bp versus 20--50~bp), we
post-process the non-mapped reads that contain a cleavage site for the restriction enzyme
used for the library generation, as previously described \citep{ay:identifying}. This post-processing
increases the number of informative contacts extracted from the metagenomic Hi-C libraries by
5-15\% depending on the read length and the cleavage site frequency. The resulting set of
informative contacts are processed further at appropriate resolution, as described below.

\subsection{Assembling the \textit{K. wickerhamii} genome}

Two input genome assemblies are used for creating the new \textit{K.
wickerhamii}
reference genome. The first is the publicly available \textit{K. wickerhamii} reference
genome originally sequenced by Baker \textit{et al}.~\cite{baker:extensive},
and the
second is the \textit{K. wickerhamii} associated cluster from
Burton \textit{et al}.~\cite{burton:species-level}. These assemblies are merged with CISA
\citep{lin:cisa} and then
merged using the mate-pair library from \citep{burton:species-level} using the
``scaffold'' command from IDBA \citep{peng:IDBA}. Hi-C reads are then aligned to
this assembly, and the seven scaffolds containing the 7 \textit{K. wickerhamii}
centromeres are identified. Lastly, this assembly is run through Lachesis
\citep{burton:chromosome}, with a restriction that the seven
centromere-containing scaffolds could not be merged.

\subsection{Data normalization}

Hi-C contact counts are subjected to many biases (GC-content, mappability,
etc) \citep{yaffe:probabilistic}. To correct for technical biases, we apply 
to the raw contact counts an iterative correction and eigenvector decomposition
(ICE) method proposed by Imakaev \textit{et al}.~\cite{imakaev:iterative},
based on the assumption
that all loci should interact equally. We then rescale the resulting matrix
such that the average normalized contact count is equal to the average raw
contact counts.

\subsection{Centromere calling}

We segment the full genome into $N$ windows of similar length ($N=611$ for
\textit{S. cerevisae} at 20~kb) and summarize the Hi-C data by the contact
count matrix $C\in\RR^{N \times N}$, where $C_{ij}$ is the normalized number
of physical interactions captured between loci in windows $i$ and $j$. 
For each window $i\in[1,N]$
we denote by $B(i) \in [1,L]$ the chromosome to which window $i$ belongs,
$L$ being the total number of chromosomes ($L=16$
for \textit{S. cerevisae}). We also denote by $x_i$ the genomic coordinate of
the center of the $i$-th window. Our objective is to infer the genomic
coordinates $p=(p_1, \ldots, p_L)$ of the centromeres of the $L$ chromosomes.
More precisely, centromeres usually consist of a sequence with a length ranging from several hundred
base pairs
for point centromeres to several thousand base pairs for regional centromeres.
In this work, we infer the mean position of these sequences.

Our main assumption is that, because centromeres colocalize in the nucleus, we
expect loci near centromeres in different chromosomes to be enriched in
Hi-C contacts.
To capture this enrichment, we model the contact counts between
windows $i$ and $j$ of different chromosomes $k$ and $l$ by a 2-D
Gaussian function centered on the corresponding centromeres $p_k$ and
$p_l$:
$$
a\exp \left(- \frac{(x_i - p_k)^2 + (x_j - p_l)^2)}{2\sigma^2}\right) + b\,,
$$
with parameters $a, b$ and $\sigma\geq 0$.
Then, denoting by $\mathcal{D}$ the set
of pairs of windows $(i,j)$ from different chromosomes with non-zero
counts, we jointly estimate the parameters $(a,b,\sigma)$ and the
positions of the $L$ centromeres by a least-squares fit of the Hi-C count data,
namely, by minimising in $a,b,\sigma\geq 0$ and $p=(p_1, \ldots, p_L)$ the
following objective function:
\begin{equation}\label{eq:opt}
\hspace*{-0.65cm}
\sum_{(i,j)\in\mathcal{D}} \left[ C_{ij} - a\exp \left(- \frac{(x_i - p_{B(i)})^2 + (x_j - p_{B(j)})^2)}{2\sigma^2}\right) - b \right]^2\,.
\end{equation}


Note that in this optimization, the position of each centromere is constrained
to be on its corresponding chromosome. Note also that for each non-zero entry
of the contact count matrix, we only fit the Gaussian
centered on the corresponding pair of loci. Thus, when the centromeres are
close to a chromosome boundary, we only fit a truncated
Gaussian.

\subsection{Initializing the optimization problem}

Because the optimization problem (\ref{eq:opt}) is non convex, the local minimum
found by the algorithm depends on the initialization of the parameters, in
particular of the centromeres' positions. We therefore need a heuristic
to initialize centromere positions. Because centromeres tend to interact
in \textit{trans} with other centromeres, a simple heuristic is to choose the position on each chromosome at the
center of the window with the largest total number of \textit{trans} contact
counts. However, we found that this heuristic was often not sufficient, because other loci besides
centromeres, such as
telomeres or rDNA clusters, can exhibit large numbers of \textit{trans} interactions. We therefore implemented another heuristic to
generate other good initializations and to explore more local minima. In
short, on each chromosome we detect a few local maxima (typically, two per
chromosome) of a smoothed \textit{trans} contact counts curve. We then initialize the
optimization by combining each choice of centromere location among the
candidates on each chromosome. If time constraints do not allow us to test all
such initializations (with 2 choices on 14 chromosomes, this corresponds to
$2^{14}=16384$ different initializations), then we can further reduce the
exploration of local minima by starting from the best candidate on each
chromosome (i.e., with the largest number of \textit{trans} contact counts),
optimizing the objective function from this initialization, and then moving to
other "nearby" local minima of the objective function by changing centromere
initialization to another candidate one centromere at a a time, until no
nearby local minimum is better than the one we have converged to.

A Python implementation of the proposed method is available at
\href{http://cbio.ensmp.fr/centurion}{http://cbio.ensmp.fr/centurion}.

\subsection{Measuring the performance}

To measure the performance of the centromere position prediction on datasets
for which we have the
ground truth, we compute the distance in base pairs between the
prediction $pred$ and the segment $(b, e)$ as follows:
$$
\max\left( (b-pred)_+ , (pred - e)_+\right)
$$
where $(u)_+$ is $u$ if $u\geq 0$ , $0$ otherwise.



\section{Results}


\subsection{Validating the method on \textit{S. cerevisiae} and \textit{P.
falciparum}}


\begin{figure*}
\includegraphics[width=\linewidth]{{figure_2}.pdf}
\caption{\textbf{Calling centromeres on \textit{P. falciparum} and \textit{S.
cerevisiae}}
\textbf{A}. Heatmap of the normalized \textit{trans} contact counts for
\textit{S. cerevisiae} Hi-C data at 40~kb overlaid with Centurion's
centromeres calls (black lines). The contact counts were smoothed with a
Gaussian filter ($\sigma =$ 40~kb) for visualization purposes. White lines
indicate chromosome boundaries.
\textbf{B.} Per chromosome errors of Centurion's centromere calls for
\textit{S. cerevisiae} using normalized (black) and raw (blue) Hi-C contact
maps at 40~kb resolution.
\textbf{C.} Heatmap of \textit{trans} contact counts for \textit{P. falciparum} trophozoite data at 40~kb
overlaid with Centurion's centromere calls (dashed black line) and ground
truth (red line) for chr 2, 3, 4 and 12.
\textbf{D.} Average errors of centromere calls for Centurion (black) and
\citet{marie-nelly:filling} method for \textit{S. cerevisiae} data from
\citet{duan:genome-wide} and the three stages of \textit{P. falciparum} when
both methods are initialized with the ground truth centromere coordinates.
}
\label{fig:high_resolution_results}
\end{figure*}



To evaluate the accuracy of our centromere prediction method, we first applied
it to two organisms with known centromere coordinates and available Hi-C data.
The first one is the widely studied budding yeast \textit{S. cerevisiae}. The
genome of \textit{S. cerevisiae} has 16 chromosomes and thus 16 centromeres,
all of which colocalize near the spindle pole body
\citep{jin:high-resolution}. All 32 telomeres of \textit{S. cerevisiae} tether
to the nuclear envelope. The repetitive ribosomal DNA of \textit{S. cerevisiae}
occurs on chromosome XII and is bundled into the nucleolus at the opposite
side of the nucleus from the spindle pole body \citep{venema:ribosome}. These
organizational principles constrain the chromosomes to fold into a distinct
configuration, known as the \textit{Rabl configuration}, which resembles a water lily shape
\citep{zimmer:principles}. The contacts between centromeres in \textit{S.
cerevisiae} chromosomes are known to result in a strong enrichment of
centromere-to-centromere Hi-C links \citep{duan:genome-wide}. We sought to
evaluate Centurion's ability to pinpoint the exact centromere locations
directly from a Hi-C contact map \citep{gotta:clustering}.

Using 40~kb-resolution Hi-C contact maps from Duan
\textit{et al}.~\cite{duan:genome-wide}
(Figure~\ref{fig:high_resolution_results}A and
\ref{fig:high_resolution_results}B), Centurion predicts centromere
coordinates with an average deviation of 11~kb from the known coordinates.
Notably, Centurion's Gaussian fitting procedure allows the centromere calls to
achieve finer resolution than is provided by the input contact maps. Using
20~kb resolution contact maps, the average deviation drops to 9~kb.
Furthermore, we observed that normalizing the contact maps
\citep{imakaev:iterative} yields substantially improved results, reducing the
average deviation to 2.5~kb for both the 20~kb and 40~kb resolution. We
investigated the differences in the prediction accuracy of our method among
the 16 different chromosomes. While our predictions were within 1~kb of the
known centromere coordinates for the chromosomes V, VI, IX, XIII and XV
(respectively, 59~bp, 235~bp, 111~bp, 289~bp and 163~bp away), they were more
than 5~kb away for chromosomes III, VII and XII (respectively, 5011~bp, 5327~bp
and 6457~bp away).
While the cause of this fluctuation of accuracy is not yet known,
chromosomes III and XII house the only major blocks of heterochromatin
in this genome other than telomeres
 (the silent mating loci and rDNA,
respectively), suggesting that linked heterochromatinized loci may interfere
with accurate centromere prediction.

We then applied our method to a second species, the malaria parasite
\textit{P. falciparum}, which is responsible for the most virulent
form of malaria \citep{who:malaria}. We recently used Hi-C to provide a global
picture of the genome architecture of \textit{P. falciparum} at three stages
(ring, schizont and trophozoite) throughout its erythrocytic life cycle in
human blood \citep{ay:three-dimensional}. Centromere coordinates for
\textit{P. falciparum} were only identified systematically relatively recently
\citep{hoeijmakers:plasmodium}. We applied Centurion to the contact maps of
each of these three stages at 10~kb, 20~kb and 40~kb resolutions
(Appendix Fig~\ref{suppfig:error_diff_res_pf}).
As with \textit{S. cerevisiae}, we observe some
variation in the accuracies of our predictions for each chromosome. However,
overall, the accuracy is very high. At 10~kb resolution, for example,
Centurion's centromere predictions fall within the known centromere location
for all 14 chromosomes during the schizont stage, 13 out of 14 for the ring stage and for
11 out of 14 chromosomes in the trophozoite stage. Overall, across the three
different stages Centurion correctly localizes 90\%, 64\%, and 45\% of
centromeres at 10~kb, 20~kb and 40~kb resolution, respectively. For the
incorrectly called centromeres, the average distance from Centurion's
prediction and the edge of the centromere is 495~bp, 1308~bp, and 2319~bp,
respectively.

We next sought to understand the sources of error in our predictions. Looking
closely at the contact counts matrices in the neighborhood of centromeres for
which the prediction is not accurate, we observed that loci in proximity to
centromeres seem to exhibit unusually sparse interaction counts. For example,
Figure~\ref{fig:high_resolution_results}C shows that in the trophozoite stage,
the centromere of chr~1 is close to a chromosome boundary and the chr~4
centromere is close to a locus with few interacting bins. The latter case
leads to bias from the normalization procedure because the few nonzero entries
in this sparse region are over-corrected. We also investigated whether the
accuracy of our prediction varies by life cycle stage and matrix resolution
(Appendix Fig~\ref{suppfig:raw_vs_normed_pf}). Many chromosomes are given
consistently poor centromere calls across all life cycle stages and at all
resolutions, corroborating the observations above that the predictions tend to
be influenced by biases intrinsic to the genome around those centromeres, such
as mappability or GC content.

We next compared the accuracy of our predictions to that of a previously
published method \citep{marie-nelly:filling}. Marie-Nelly
\textit{et al}.\
method often works well for identifying centromeres using Hi-C libraries with
very high sequencing depth; however, when Hi-C sequencing depth is limited or
when loci other than centromeres strongly cluster,
the first step of the procedure, called ``pre-localization,''
sometimes fails to identify the correct fixed size window in which the
centromeres reside.
We hypothesized that
the joint centromere calling by Centurion, which leverages data from all
chromosomes at once, might alleviate this instability. To test this
hypothesis, we applied the Marie-Nelly \textit{et al}.\
method to the same four
datasets (one \textit{S. cerevisiae} and three \textit{P. falciparum})
described above. As shown in Appendix 
Figure~\ref{suppfig:marie_nelly_vs_us_HR}, in each of these four
datasets Centurion identifies centromeres with better accuracy than the
Marie-Nelly \textit{et al}.\ method.
For instance, the colocalization of rDNA
clusters and virulence genes in \textit{P. falciparum} drastically changes the pattern
of the correlation matrix used by Marie-Nelly
\textit{et al}.\ to pre-localize
their centromere calls, thus confounding their prediction (Appendix
Fig.~\ref{suppfig:pearson_corr_pf}).

We also asked whether the improvement of Centurion over
Marie-Nelly \textit{et al}.\ method is due to the
initialization step, or
due to different objective functions used by each method. We initialized both
optimization problems with the ground truth and computed the resulting error.
Our results (Figure~\ref{fig:high_resolution_results}D) showed that
Centurion's error is still between 4- and 10- fold lower, thus demonstrating
the benefit of jointly calling centromeres.

\subsection{Resolution, sequencing depth and prediction accuracy}


\begin{figure*}
\includegraphics[width=\linewidth]{{figure_3}.pdf}
\caption{\textbf{Impact of Hi-C library sequencing depth on the stability of
the centromere calls}
Average variance of the results of Centurion on 500 generated datasets
obtained by downsampling the raw contact counts to the desired coverage. 
}
\label{fig:downsampled}
\end{figure*}



To assess the stability of our predictions, we simulated 500 bootstrapped data
sets of \textit{S. cerevisiae} and of each stage of \textit{P. falciparum}
with an expected total number of reads equal to the contact counts matrices.
These bootstrapped samples were obtained by drawing a contact count for each
pair of loci i and j from a Poisson distribution of intensity $c_{ij}$. We
then ran the optimization process on the bootstrapped data sets, starting with
initial values randomly placed within 40 kb of the centromere calls from our
optimization in Appendix Tables~\ref{supptable:sc_results},
\ref{supptable:rings_results}, \ref{supptable:trophs_results} and
\ref{supptable:schizonts_results}. Our results show that the optimization is
very stable (average variance of 25~bp for ring, 6~bp for schizont and 12~bp
for trophozoite), suggesting that the stochastic sampling of the sequencing
procedure does not significantly affect centromere predictions.

We then sought to investigate the extent to which the matrix resolution and
sequencing depth affect the accuracy of Centurion's predictions. As already
seen in Appendix Figures~\ref{suppfig:error_diff_res_sc} and
\ref{suppfig:error_diff_res_pf}, different species give different
results: for \textit{S. cerevisiae}, increasing the matrix resolution to 10~kb
results in lowered accuracy of centromere calls, while in \textit{P.
falciparum} the call quality improves slightly. We speculated that our ability
to call centromeres in a given species at a given resolution may depend on the
choice of restriction enzyme, the sequencing depth, and the resolution of the
contact map.

We next evaluated the effect of depth of sequence coverage on the quality of
our centromere predictions. We generated 500 low-coverage datasets by randomly
downsampling the raw contact counts. We then ran the optimization process on
these downsampled datasets, initializing with perturbed calls as before. We
observe that the low coverage centromere calls remain highly stable and
accurate. As illustrated in Figure~\ref{fig:downsampled}, results across all
data sets only begin to degrade when downsampling to less than 10\% of the
total number of reads, which corresponds to less than one count per bin on
average. Centurion is thus applicable to call centromeres at low cost or for
low-abundance species in metagenomic samples.

\subsection{Centromere calls on a metagenomic dataset}


\begin{figure*}
\includegraphics[width=\linewidth]{{figure_4}.pdf}
\caption{\textbf{Centromere calling on a metagenomic sample} \textbf{A.}
Heatmap of the \textit{trans} contact counts for \textit{K.\ wickerhamii}
overlaid with \textit{de novo}
centromere calls (black lines). The contact counts were smoothed with a
Gaussian filter ($\sigma = 40~kb$) for visualization purposes. White lines
indicate chromosome boundaries.
\textbf{B.} Box plots indicating the error (in kb) for each chromosome in
Centurion's centromere calls for eight yeasts with known centromere
coordinates from the combined metagenomic Hi-C samples M-3D and M-Y of
\citep{burton:species-level} on the 20~kb contact count matrices.}
\label{fig:metagenomic_results}
\end{figure*}


We next sought to call centromeres in several species simultaneously by
combining Centurion with metagenomic Hi-C libraries. We previously
\citep{burton:species-level} generated two Hi-C datasets from synthetic
mixtures: one containing 16 yeast strains (including four strains of
\textit{S. cerevisiae}), and one containing a mixture of 8 yeasts and 10
prokaryotic species. The two samples contain a total of 19 yeast species, some
of which are much better characterized than others: centromere positions are
already known for eight species (\textit{K. lactis}, \textit{L. kluyveri},
\textit{L. thermotolerans}, \textit{S. cerevisiae}, \textit{S. kudriavzevii},
\textit{S. mikatae}, \textit{S. pombe}, \textit{S. rouxii}) and partially for
one more (\textit{S. bayanus}) \citep{scannell:awesome, souciet:comparative,
mcdowall:pombase, dujon:genome}.

We aligned the reads from the metagenomic Hi-C datasets to these yeast
species' reference genomes (see Appendix~\ref{suppnotes:initialization}). The
quality of the individual species datasets differ greatly because the
organisms vary in abundance in the metagenomic samples, and because many
sequences are shared nearly identically between organisms, making the number
of uniquely mappable reads for each organism range between 109~k for one of
the \textit{S. cerevisiae} strains to 26~M for the bacteria \textit{V.
fischeri}. Consequently, the sparsity of the matrices is variable (Appendix
Tables~\ref{supptable:m2_qc} and \ref{supptable:my_qc}). Furthermore,
some contact counts matrices include at least
one interaction count for more than 99\% of all possible locus pairs, whereas
other matrices are below 5\%. Similarly, in the 40~kb matrices, the average
number of interchromosomal contact counts per bin varies from less than 0.004
to more than 200. In particular, the matrices for the four \textit{S.
cerevisiae} strains are very sparse: the reference genomes of the four strains
are very similar to one another; thus, we are not able to map reads uniquely.
We therefore discarded those strains from our analysis, as well as organisms
with incomplete reference genomes. We applied Centurion to the remaining 14
yeasts (\textit{E. gossypii}, \textit{K. lactis}, \textit{K. wickerhamii},
\textit{L. kluyveri}, \textit{L. waltii}, \textit{S. bayanus}, \textit{S.
kudriavzevii}, \textit{S. mikatae}, \textit{S. paradoxus}, \textit{S.
stipitis}, \textit{P. pastoris}, \textit{L. thermotolerans}, \textit{S.
pombe}, \textit{S. rouxii}) on both 20~kb and 40~kb contact maps.

Across these 14 species Centurion performs well, both on high-coverage
datasets (\textit{K. lactis}, \textit{L. kluyveri}, \textit{S. bayanus}) and
low-coverage datasets (\textit{S. mikatae}), at 20~kb and 40~kb, finding
centromeres at an average deviation from the ground truth of 10~kbp
(Figure~\ref{fig:metagenomic_results}B and Appendix
Figure~\ref{suppfig:metagenomic_sample_40}).
Given this success with yeasts with
known centromere positions, we next made \textit{de novo} centromere calls for the
other 6 yeast species present in the metagenomic samples. These regions,
visualized in Appendix Figures~\ref{suppfig:KL_calls}, \ref{suppfig:LK_calls},
\ref{suppfig:SB_calls}, \ref{suppfig:SM_calls}, \ref{suppfig:SK_calls},
\ref{suppfig:LT_calls}, \ref{suppfig:zP_calls}, \ref{suppfig:ZR_calls}, are strong candidates for
experimental validation by other approaches. One feature that is shared by
centromeres across all studied fungi is that they reside in regions of early
replication timing \citep{koren:epigenetically, pohl:functional}. Thus if our
centromere calls lie in regions of advanced replication timing in a species
for which replication timing has been profiled but centromeres have not
yet been identified, this data could be used to assess the validity of
our predictions. Accordingly, we
overlaid the positions of our centromere calls in \textit{P. pastoris}, where
replication has been recently profiled \citep{liachko:gc-rich}. In all four
chromosomes, \textit{P.\ pastoris} centromere predictions lay in regions of
early replication timing (Appendix Fig.~\ref{suppfig:ppas_timing}),
lending support to our predictions.

\subsection{The effect of the choice of restriction enzyme}

In addition to the resolution of our contact matrices, the underlying
resolution of the Hi-C data itself may limit the accuracy of our predictions.
Hi-C reads can only occur near the recognition site of the restriction enzyme
used in the Hi-C assay; indeed, the best resolution we can hope to achieve is
a matrix in which each corresponds to one restriction enzyme fragment. Some
restriction enzymes cut much more frequently than others. Thus, we speculated
that a Hi-C experiment using enzymes that cut more frequently might yield more
accurate results than an experiment using less frequently cutting enzymes.

To address this question, we compare the accuracy of centromere calling from
two Hi-C libraries created from a single metagenomic sample using different
combinations of restriction enzymes. The first library was created using the
two 6~bp-cutters, HindIII and NcoI. The second library uses Sau3AI, which has a
4~bp recognition site, and AflIII, which has a 6~bp recognition site with two
degenerate sites, making it effectively a 5~bp cutter. Digestion with
HindIII/NcoI yields a total of 8324 restriction fragments, whereas digestion
with Sau3AI/AflIII yields 42359 restriction fragments. We corrected for the
difference in Hi-C sequencing depth between Sau3AI/AflIII and the NcoI/HindIII
libraries by generating downsampled datasets with an equal number of reads
from each sequencing library. We then normalized the datasets and applied
Centurion. The sample includes three species for which we possess the ground
truth centromere locations, only one of which (\textit{L. thermotolerans})
had enough reads in both the NcolI/HindIII (~63000 reads) and the pooled
Sau3AI/AflIII (~55000 reads) datasets to correctly call the centromeres. The
error on the downsampled Sau3AI/AflIII datasets (8~kbp) was on average half as
large as the error on the the NoclI/HindIII datasets (16~kbp). Thus, we
conclude that using a restriction enzyme with more frequent cutting sites
enables more precise centromere calls at fine scales.

\section{Discussion}

While centromeres are a fundamental element in the biology of genomes, their
identification in diverse species has proven difficult due to sequence
divergence and limitations of available tools. In this work, we have
developed a novel method, Centurion, that uses centromere colocalization and
the pattern it creates in Hi-C contact maps to jointly call centromeres for
all chromosomes of an organism.
We first established the feasibility of this approach by demonstrating
that Centurion accurately calls regional centromeres on the parasite
\textit{P.\ falciparum} and the yeast \textit{S.\ pombe} as well as
point centromeres on several other yeasts with known centromere
coordinates.
For the species with high depth Hi-C sequencing, Centurion often identified
centromeres within 1~kb of the actual coordinates (41 times out of 58 for
three stages of \textit{P. falciparum} and \textit{S. cerevisiae} data). We
then used Centurion to infer centromeres of multiple yeast species (8 with
known, 6 with unknown centromere coordinates) from two metagenomic Hi-C
samples. Our results showed that Centurion still accurately identifies
centromere coordinates from samples with only limited sequencing depth. Thus,
Centurion can be used to accurately and efficiently identify centromere
locations in yeast species.

The task of centromere identification from Hi-C data has been attempted
recently by others \citep{marie-nelly:filling}. Centurion offers a few key
differences compared to the previous approach. The first difference is in the
pre-localization of candidate centromeres.
Marie-Nelly \textit{et al}.'s
method uses only the \textit{cis} Pearson correlation information independently per
chromosome to identify the initial candidates. However, the pattern created by
centromeres in the Pearson correlation matrix can be very similar to the patterns
generated by other genomic elements such as rDNA coding regions or by specific
gene clusters (e.g., virulence genes in \textit{P. falciparum}). 
Because
Marie-Nelly \textit{et al}.'s method restricts
the further search for the best
centromere coordinate to only the candidates from the pre-localization step,
an inaccurate candidate (e.g., an rDNA region instead of a centromere) will
prevent the method from finding the correct centromere location. Centurion, on
the other hand, utilizes \textit{trans} contact information jointly across all
chromosomes for its pre-localization step. Furthermore, Centurion allows
multiple candidates per chromosome during the second step of the optimization,
thereby leaving room for correcting potential errors in the pre-localization
step. The second difference between the two methods is in how they use the
submatrices that correspond to \textit{trans} contact maps flanking the pairs
of candidate centromeres from the pre-localization step. For an organism with
N chromosomes, Marie-Nelly \textit{et al}.'s
method carves out the N-1
\textit{trans} submatrices for each chromosome, sums these N-1 matrices and
then collapses the sum into a 1D vector of row/column sums. Then,
independently for each chromosome, the method fits a Gaussian to this 1D
vector, and the resulting peak corresponds to the predicted centromere
location. In this procedure, both the summation of N-1 matrices and the
collapsing of the resulting matrix into a 1D vector of sums result in loss of
important information embedded in 2D maps. Furthermore, performing the
Gaussian fit separately for each chromosome does not fully take into account
the joint co-localization of the other N-1 centromeres. To address these
issues, Centurion infers a 2D Gaussian fit that best explains the observed
\textit{trans} contact counts, jointly optimizing these 2D fits for all pairs
of centromeres. Both of these improvements in the pre-localization and the
optimization steps allow Centurion to perform better specifically for the
cases with limited sequencing depth.
Our approach could be improved in several respects. First, better
modeling of zero contact counts may improve inference for organisms
with many repeated sequences in the peri-centromeric regions, or data
sets with low sequencing depth. Second, one could model contact counts
as a Gaussian distribution centered on the pairs of centromere
locations. Maximising the log likelihood of such a model 
might yield improved performance.
Last, as described here, our method requires reference genomes for the
metagenomic samples. It would be possible to first build reference
genomes directly from the Hi-C data, using methods like Lachesis
\citep{burton:chromosome} or Graal \citep{marie-nelly:high-quality},
and then infer centromeres locations using the inferred
references. However, the inherent structure of Hi-C contact counts for
organisms with colocalizing centromeres will likely present a
challenge for these methods because pericentromeric sequences on
different chromosomes are likely to appear to be adjacent to one
another.

Finally, our new centromere predictions have practical applications.
Autonomously replicating plasmids and artificial chromosomes are useful tools
for research and strain engineering \citep{boer:yeast}. Identification of
centromeres in new species will facilitate building such constructs over an
expanded species range. \textit{P. pastoris}, for example, is a common
industrial chassis \citep{cregg:expression}, but existing plasmid tools in the
species have elevated loss rates \citep{liachko:autonomously} that could be
stabilized by addition of a centromere. Many of our centromere calls were
accurate to $<1$~kb, making experimental validation possible.

\section{Funding}

This work was supported by the European Research Council [SMAC-ERC-280032 to
J-P.V., N.V.]; the European Commission [HEALTH-F5-2012-305626 to J-P.V.,
N.V.]; the French National Research Agency [ ANR-11-BINF-0001 to J-P.V.,
N.V.]; the National Institute of Health/National Human Genome Research
Institute [HG006283 to J.S., T32HG000035 to J.N.B.];
National Institute of Health/National Institute of General Medical Sciences [P41 GM103533 to I.L.,
M.J.D., W.S.N.; R01AI106775 to F. A., W.S.N.]; National Science Foundation
[1243710 to I.L., M.J.D.]. M.J.D. is a Rita Allen Foundation Scholar and a Senior
Fellow in the Genetic Networks program at the Canadian Institute for Advanced
Research.

\section{Acknowledgements}

We thank Celia Payen for providing the yeast centromere annotations,
St\'{e}fan van der Walt for advice on peak detection algorithms, Fabrice
Varoquaux for help on understanding the specificity of \textit{A. thaliana}
genome
and Chlo\'{e}
Azencott for helpful comments on the manuscript.


