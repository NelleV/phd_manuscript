
% this file is called up by thesis.tex
% content in this file will be fed into the main document

%: ----------------------- introduction file header -----------------------
\chapter{Introduction and related work}

% the code below specifies where the figures are stored
\graphicspath{{1_introduction/figures/}}

% ----------------------------------------------------------------------
%: ----------------------- introduction content ----------------------- 
% ----------------------------------------------------------------------

%: ----------------------- HELP: latex document organisation
% the commands below help you to subdivide and organise your thesis
%    \chapter{}       = level 1, top level
%    \section{}       = level 2
%    \subsection{}    = level 3
%    \subsubsection{} = level 4
% note that everything after the percentage sign is hidden from output

%\figuremacro{largepotato}{A common glucose polymers}{The figure shows starch granules in potato cells, taken from \href{http://molecularexpressions.com/micro/gallery/burgersnfries/burgersnfries4.html}{Molecular Expressions}.}

%: ----------------------- HELP: adding figures with macros
% This template provides a very convenient way to add figures with minimal code.
% \figuremacro{1}{2}{3}{4} calls up a series of commands formating your image.
% 1 = name of the file without extension; PNG, JPEG is ok; GIF doesn't work
% 2 = title of the figure AND the name of the label for cross-linking
% 3 = caption text for the figure

%: ----------------------- HELP: www links
% You can also see above how, www links are placed
% \href{http://www.something.net}{link text}

%\figuremacroW{largepotato}{Title}{Caption}{0.8}
% variation of the above macro with a width setting
% \figuremacroW{1}{2}{3}{4}
% 1-3 as above
% 4 = size relative to text width which is 1; use this to reduce figures


%: ----------------------- HELP: tables
% Directly coding tables in latex is tiresome. See below.
% I would recommend using a converter macro that allows you to make the table in Excel and convert them into latex code which you can then paste into your doc.
% This is the link: http://www.softpedia.com/get/Office-tools/Other-Office-Tools/Excel2Latex.shtml
% It's a Excel template file containing a macro for the conversion.

%\begin{table}[htdp]
%\centering
%\begin{tabular}{ccc} % ccc means 3 columns, all centered; alternatives are l, r
%{\bf Gene} & {\bf GeneID} & {\bf Length} \\ 
% & denotes the end of a cell/column, \\ changes to next table row
%\hline % draws a line under the column headers
%human latexin & 1234 & 14.9 kbps \\
%mouse latexin & 2345 & 10.1 kbps \\
%rat latexin   & 3456 & 9.6 kbps \\
% Watch out. Every line must have 3 columns = 2x &. 
% Otherwise you will get an error.
%\end{tabular}
%\caption[title of table]{\textbf{title of table} - Overview of latexin genes.}
% You only need to write the title twice if you don't want it to appear in bold in the list of tables.
%\label{latexin_genes} % label for cross-links with \ref{latexin_genes}
%\end{table}
% ----------------------------------------------------------------------

Start from the beginning ...
