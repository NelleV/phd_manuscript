
% ----------------------------------------------------------------------
%                   LATEX TEMPLATE FOR PhD THESIS
% ----------------------------------------------------------------------

% based on Harish Bhanderi's PhD/MPhil template, then Uni Cambridge
% http://www-h.eng.cam.ac.uk/help/tpl/textprocessing/ThesisStyle/
% corrected and extended in 2007 by Jakob Suckale, then MPI-CBG PhD programme
% and made available through OpenWetWare.org - the free biology wiki


%: Style file for Latex
% Most style definitions are in the external file PhDthesisPSnPDF.
% In this template package, it can be found in ./Latex/Classes/
\documentclass[twoside,11pt]{Latex/Classes/PhDthesisPSnPDF}

%: Macro file for Latex
% Macros help you summarise frequently repeated Latex commands.
% Here, they are placed in an external file /Latex/Macros/MacroFile1.tex
% An macro that you may use frequently is the figuremacro (see introduction.tex)
\include{Latex/Macros/MacroFile1}

\newcommand{\todo}[1]{\textbf{[TODO: #1]}}
\newcommand{\note}[1]{\textbf{[NOTE: #1]}}
\newcommand{\OMIT}[1]{}
\newcommand{\Xb}{\textbf{X}}
\newcommand{\RR}{\mathbb{R}}
\newcommand{\Dcal}{\mathcal{D}}
\newcommand {\br}[1]{\left(#1\right)}

\usepackage{amsmath}
\usepackage{multirow}
\newtheorem{problem}{Problem}


%: ----------------------------------------------------------------------
%:                  TITLE PAGE: name, degree,..
% ----------------------------------------------------------------------
% below is to generate the title page with crest and author name

%if output to PDF then put the following in PDF header
\ifpdf  
    \pdfinfo { /Title  (nvaroquaux\_phd)
               /Creator (Nelle Varoquaux)
               /Producer (pdfTeX)
               /Author (nelle.varoquaux@ensmp.fr)
               /CreationDate (26:201505)  %format D:YYYYMMDDhhmmss
               %/ModDate (D:YYYYMMDDhhmm)
               %/Subject (xyz)
               /Keywords (3D structure, genome, Hi-C, inference) }
    \pdfcatalog { /PageMode (/UseOutlines)
                  /OpenAction (fitbh)  }
\fi

\title{Inferring the three dimensional structure of the genome}

% ----------------------------------------------------------------------
% The section below defines www links/email for author and institutions
% They will appear on the title page of the PDF and can be clicked
\ifpdf
  \author{\href{mailto:nelle.varoquaux@ensmp.fr}{Nelle Varoquaux}}
%  \cityofbirth{born in XYZ} % uncomment this if your university requires this
%  % If city of birth is required, also uncomment 2 sections in PhDthesisPSnPDF
%  % Just search for the "city" and you'll find them.
  \collegeordept{\href{http://u900.curie.fr}{Institut Curie, INSERM U900, Mines ParisTech}}
  \university{\href{http://ensmp.fr}{Ecole Doctorale "Complexité du vivant" - UPMC}}

  % The crest is a graphics file of the logo of your research institution.
  % Place it in ./0_frontmatter/figures and specify the width
  \crestuniv{\includegraphics[width=5cm]{logo_upmc.png}}
  \crest{\includegraphics[width=12cm]{logo_U900.png}}
  
% If you are not creating a PDF then use the following. The default is PDF.
\else
  \author{Nelle Varoquaux}
%  \cityofbirth{born in XYZ}
  \collegeordept{Institut Curie, INSERM U900, Mines ParisTech}
  \university{Mines ParisTech}
  \crestuniv{\includegraphics[width=5cm]{logo_upmc.png}}
  \crest{\includegraphics[width=12cm]{logo_U900.png}}
\fi

\renewcommand{\submittedtext}{PhD Report}
\degree{}
\degreedate{2015 March 26th}
\supervisor{Jean-Philippe Vert $^{1,2,3}$}
\affiliation{$^1$Institut Curie, Paris, France $^2$INSERM, U900, Paris, France $^3$Mines ParisTech, Fontainebleau, France $^4$CNRS UMR3215, Paris, France $^5$INSERM U934, Paris, France}
% ----------------------------------------------------------------------
       
% turn of those nasty overfull and underfull hboxes
\hbadness=10000
\hfuzz=50pt


%: --------------------------------------------------------------
%:                  FRONT MATTER: dedications, abstract,..
% --------------------------------------------------------------

\begin{document}

%\language{english}

% sets line spacing
\renewcommand\baselinestretch{1.2}
\baselineskip=18pt plus1pt


%: ----------------------- generate cover page ------------------------

\maketitle  % command to print the title page with above variables


%: ----------------------- cover page back side ------------------------
% Your research institution may require reviewer names, etc.
% This cover back side is required by Dresden Med Fac; uncomment if needed.
%
%\newpage
%\vspace{10mm}
%1. Reviewer: Name
%
%\vspace{10mm}
%2. Reviewer: 
%
%\vspace{20mm}
%Day of the defense:
%
%\vspace{20mm}
%\hspace{70mm}Signature from head of PhD committee:



%: ----------------------- abstract ------------------------

% Your institution may have specific regulations if you need an abstract and where it is to be placed in the document. The default here is just after title.


% Thesis Abstract -----------------------------------------------------


%\begin{abstractslong}    %uncommenting this line, gives a different abstract heading
\begin{abstracts}        %this creates the heading for the abstract page

Here is the story of my life ...

\end{abstracts}
%\end{abstractlongs}


% ---------------------------------------------------------------------- 


% The original template provides and abstractseparate environment, if your institution requires them to be separate. I think it's easier to print the abstract from the complete thesis by restricting printing to the relevant page.
% \begin{abstractseparate}
%   
% Thesis Abstract -----------------------------------------------------


%\begin{abstractslong}    %uncommenting this line, gives a different abstract heading
\begin{abstracts}        %this creates the heading for the abstract page

Here is the story of my life ...

\end{abstracts}
%\end{abstractlongs}


% ---------------------------------------------------------------------- 

% \end{abstractseparate}


%: ----------------------- tie in front matter ------------------------

%\frontmatter
\include{0_frontmatter/dedication}
% Thesis Acknowledgements ------------------------------------------------


%\begin{acknowledgementslong} %uncommenting this line, gives a different acknowledgements heading
\begin{acknowledgements}      %this creates the heading for the acknowlegments

First I would like to express my deep gratitude to Jean-Philippe Vert for
supervising my work and sharing his expertise during these three years, for
having welcomed me in his research team and giving me the opportunity to work
in two prestigious and stimulating institutes: Institut Curie and Mines
ParisTech. I would also like to thank William Noble for suggesting the subject
and the collaborations from which my PhD relied on, and mentoring me during
the past three years.

I would like to thank St\'ephane Robin and Marc Marti-Renom for accepting to
review my thesis, and sharing interesting comments and discussions with me. I
am also grateful to Emmanuel Barillot, William Noble and Julien Mozziconacci
for accepting to be part of the jury.

Many people have contributed, directly or indirectly to the work presented in
this thesis, and I would like to thank them here: Nicolas Servant, for whom I
am grateful for his availablity to answer my questions, for sharing his
expertise on Hi-C; Karine Le Roch and the amazing people from her team,
Sebastiaan Le Bol and Evelien Bunnik, for providing the collaboration, support
and ideas behind our project on \textit{P. falciparum}, once again William
Noble and members of his team, Ferhat Ay, Kate Cook and Wenxiu Hu, for their
expertise on the analysis of the 3D structure of the genome and our fruitful
collaborations; and all my other collaborators: Édith Heard, Éric Viara,
Chong-Jian Chen, Job Dekker, Bryan Lajoie, Jacques Prudhomme, Maitreya Dunham,
Ivan Liachko, Jay Shendure, Josh Burton. I'd like to thank all the members and
former members of the CBIO: Alice Schoenauer-Sebag for sharing her experience
(and frustration) on using the High Performance Ressources at our disposal,
but also her passion for the op\'era; Toby Hocking and Anne-Claire Haury,
without whom the CBIO was just not quite the same; Elsa Bernard, Erwan
Scornet, Véronique Stoven for all the stories shared around a coffee, Thomas
Walter, Yunlong Jiao, Chloé Azencott, Matahi Moarii, Pierre Chiche, Xiwei
Zhang, Victor Bellon, Judith Abécassis, Svetlana Gribkhova, Nino Shervashidze,
Andrea Cavagnino, Émile Richard, Édouard Pauwels, Kevin Vervier, Olivier
Collier, and Azadeh Khaleghi.

I'd like to thank my parents supporting with me during those three years, my
brother Ga\"el for convincing me to start my studies again to deepen my
knowledge of machine learning.

\end{acknowledgements}
%\end{acknowledgmentslong}

% ------------------------------------------------------------------------





%: ----------------------- contents ------------------------

\setcounter{secnumdepth}{3} % organisational level that receives a numbers
\setcounter{tocdepth}{3}    % print table of contents for level 3
\tableofcontents            % print the table of contents
% levels are: 0 - chapter, 1 - section, 2 - subsection, 3 - subsection


%: ----------------------- list of figures/tables ------------------------

%\listoffigures	% print list of figures


%: ----------------------- glossary ------------------------

% Tie in external source file for definitions: /0_frontmatter/glossary.tex
% Glossary entries can also be defined in the main text. See glossary.tex
%%%   Glossary is not supported by ShareLaTeX now, sorry!!!
\include{0_frontmatter/glossary} 
%
%\begin{multicols}{2} % \begin{multicols}{#columns}[header text][space]
%\begin{footnotesize} % scriptsize(7) < footnotesize(8) < small (9) < normal (10)
%
%\printnomenclature[1.5cm] % [] = distance between entry and description
%\label{nom} % target name for links to glossary
%
%\end{footnotesize}
%\end{multicols}



%: --------------------------------------------------------------
%:                  MAIN DOCUMENT SECTION
% --------------------------------------------------------------

% the main text starts here with the introduction, 1st chapter,...
\mainmatter

\renewcommand{\chaptername}{} % uncomment to print only "1" not "Chapter 1"


%: ----------------------- subdocuments ------------------------

% Parts of the thesis are included below. Rename the files as required.
% But take care that the paths match. You can also change the order of appearance by moving the include commands.


% this file is called up by thesis.tex
% content in this file will be fed into the main document

%: ----------------------- introduction file header -----------------------
\chapter{Introduction and related work}

% the code below specifies where the figures are stored
\graphicspath{{1_introduction/}}

\begin{abstract}{Résumé}

\end{abstract}

\begin{abstract}{Abstract}

We aim in this chapter at providing some background on biological and
mathematical concepts present in this thesis.

\end{abstract}


\section{Peeking under the hood of genome architecture}

Methods to investigate the 3D structure of the genome fall broadly into two
categories: bio imaging techniques and biochemical protocols. In the first
category, light microscopy allows single cell visualization of specific loci
and enables live cell imaging, sometimes at very high resoluton
\citep{cremer:chromosome-2010}. Yet, these techniques limit studies to a very
small number of loci. On the other hand, biochemical protocols, such as
chromosome conformation capture (3C) and its derivatives, enable to measure
physical interaction between DNA fragments \cite{dekker:capturing}, but
performing single cell experiments is troublesome, and tracking live cell
impossible. In this thesis, we are mostly interested in analysing 3C and such
datasets.

\subsection{3C, 4C, 5C and Hi-C data}

 In recent years, the technique of chromosome conformation capture (3C)
\citep{dekker:capturing}, which identifies physical contacts between different
genomic loci and yields information about their relative spatial distance in
the nucleus, has paved the way for the systematic analysis of the 3D structure
of DNA. 3C techniques and its derivatives are based on 5 experimental steps
\citep{lieberman-aiden:comprehensive, kalhor:genome}.

\begin{itemize}
\item \textbf{Cross-linking} : results in the cross-linking of DNA segments to
proteins and to cross-linking of proteins with each other.
\item \textbf{Restriction digest} A restriction enzyme is added in excess to
the cross-linked DNA. The restriction enzyme will cut the DNA at specific
nucleotide sequences, separating the non-cross-linked DNA from the
cross-linked chromatin. Recognition sequences in DNA differ from each
restriction enzyme, producing different lengths and sequences of strands.
The selection of the restriction enzyme depends on the type of studies
targeted in the experiment.
\item \textbf{Intramolecular Ligation} The third step is an intramolecular
ligation step. DNA fragments are binded together. There are two major types
of ligation junctions: the first is the ligation of two neighboring DNA
fragments, and the second is the junction that is formed when ligating one end
of the fragment to the other end of the same fragment. The latter represents
around 30\% of the junctions formed.
\item \textbf{Reverse Cross-links} The fourth step consists of reversing the
first step: the reversal of cross-links.
\item \textbf{Quantitation} Polymerase chain reaction (PCR) is used to amplify
the DNA copies and to assess the frequencies of the fragments of interest,
which are then sequenced.
\end{itemize}

\begin{figure}
\begin{center}

\includegraphics[width=0.8\linewidth]{figures/hic_protocol.png}
\end{center}
\caption{\textbf{Hi-C Protocol.} The procedure relies on cross linking,
restriction enzymes digestions, intra molecular ligation, deproteinization and
deep sequencing. Reads are then aligned to the reference genome, and binned at
$10kb$, $40bk$ or $100kb$ depending on coverage.}
\end{figure}


After paired-end sequencing, each pair of reads can be associated to one
\citep{lieberman-aiden:comprehensive} or several \citep{ay:identifying} DNA
interactions. We can then create a symmetric hollow matrix of integers, for
which entries correspond to the number of reads that fall into a bin. We
denote by $C$ the interaction frequency matrix, and $c_{ij}$ the interaction
frequency between locus $i$ and locus $j$.

These protocols are complex, and yield highly biased interaction frequencies
\citep{imakaev:iterative, cournac:normalization, yaffe:probabilistic}.
\citet{imakaev:iterative} proposes a simple iterative method, called ICE, to
normalize the data. In short, the authors assume that the bias of each entry
$c_{ij}$ of the matrix can be written as the product of two biases $\beta_i$
and $\beta_j$ corresponding to biases induced by loci. Hence, we can write
$c_{ij} = \beta_i \beta_j p_{ij}$, where $p_{ij}$ is the probability of locus
$i$ interacting with locus $j$. Thus, $\sum_i p_{ij} = 1$. This is a non convex
optimization problem that can be solved exactly by an iterative process. To
avoid degeneracies, we filter out the top 2\% sparse loci from our entry
matrix before applying ICE. To give an intuition, this method projects each
vector of interactions onto the $\ell_1$ unit ball. In practice, it yields an
expected interaction frequency count: $k p_{ij}$, where $k$ is the mean
interaction frequency.

Thought still quite recent, chromosome conformation capture and its genome
wide derivatives are now widely used to discover how DNA folds in a bunch of
different organisms \citep{duan:three, sexton:three-dimensional,
tanizawa:mapping, ay:three-dimensional}. The challenge is now to increase the
Hi-C resolution, using very large data sets with deeper sequencing
\citep{rao:3d, jin:high-resolution}. As any genome-wide sequencing data, Hi-C
usually requires several millions or billions of paired-end sequencing reads,
depending on genome size and on the desired resolution. Managing these data
thus requires optimized bioinformatics workflows able to extract the contact
frequencies in reasonable computational time and with reasonable storage
requirements. The overall strategy to analyze Hi-C data is converging among
recent studies and summarized in \cite{lajoie:hitchhiker}. Our collaborators
and we have built HiC-Pro, an an easy-to-use and complete pipeline to process
Hi-C data from raw sequencing reads to the normalized contact maps.

\section{The study of chromosome organization}

The study of chromosome organization broadly falls into two categories:
model-based studies and data-driven studies. The former methods consider the
polymer nature of DNA to leverage the theoretical and computational work done
statistical physics of polymers, to built, with as few assumptions as
possible many chromosome conformations. Those chromosome conformations
are then used to compare against experimental data, such as Hi-C contact
counts matrices, in order to iteratively improve the models. These models
offer mechanistical insights into the folding of DNA. The latter approaches
use the experimental data to infer 3D models, by typically minizing a cost
function ensuring the models are as consistent as possible with the data.

\subsection{DNA as a polymer}

Polymer physics divide homopolymers, polymers with identical polymer into
three main type, which are then extended to build more complex models (1) the
\textit{random coil}, (2) the \textit{swollen coil}, (3) the
\textit{equilibrium polymer}. These polymers are characterized by
relationships such as the one between the size of a polymer subchain $L(s)$ as
a function of its lengths $s$, between the size of the polymer $L(N)$ and the
total length of this polymer $N$, or between the contact probability between
monomers $P(s)$ and the linear distance between monomers $s$. DNA is thus a
polymer, each pair of nucleic acid forming a monomer, and the distance $s$ the
genomic distance between two loci.

The \textit{random coil} corresponds to an unconstrained polymer, best
described by a random walk. A \textit{random coil} of length $N$ has an
expected size of $N^{1/2}$, and so has any of its subchain: $L(s) \sim
s^{1/2}$. The contact probability between two monomers is $P(s) \sim
s^{-3/2}$. These relationships lead to a low density polymer, where contact
between monomers is sparse.

\fixme{Add the swollen coil, ie, the self avoiding random walk}

If the polymer is constrained in a small volume, the polymer folds into an
\textit{equilibrium globule} state. This polymer behaves as a random walk,
until it bounces of the boundary of the constrained space, and starts another
random walk inside the confined volume. The expected size of this polymer is
$N^{1/3}$ (\fixme{WHY !!! Why not the volume defined by the constrained space
???}) The size of a subchain of a polymer follows the relationship: $L(s) =
s^{1/2}$ for $s < N^{2/3}$ and constant elsewise: it is the same as a random
coil until it plateaus. The probability of contact between two monomers is
$P(s) = s^{-3/2}$ for $s < N^{2/3}$ and constant elsewise: once again, it is
the same relationship as the random coil, until it becomes constant.
Interestingly, this polymer is uniformely distributed in the constrained
space, and the density of the polymer is independent of the total length $N$
and the volume $V$.

Another interesting polymer behaviour is the \textit{fractal globule}: when
the chain is sufficiently long and the constrained volume sufficiently small,
the polymer forms knotted crumples of increasing size. The polymer is then
constrained by the available volume, and other parts of the polymer, which
creates topological constraints forcing the polymer to collapse into crumples.
First proposed by \citet{grosberg:role}, and further analysed by
\citet{mirny:fractal}, the polymer presents interesting properties: the size
of any subchain follows the same law: $R(s) \sim s^{1/3}$, and the probability
of contact between two monomers is inversely propuortional to the linear
distance that separates them: $P(s) \sim s^{-1}$. In particular, the
probability of contact doesn't exhibit the same plateau as the equilibrium
polymer.

\subsection{3D models of DNA folding}

We will review some techniques developped to study the folding of DNA, mostly
data-driven models. \citet{rosa:computational} provide a more more complete
overview of computational models of genome architectures.

Several techniques have been developed to infer three-dimensional models of
the genome from interaction counts data. They fall into three categories: the
first finds an average structure by optimizing an objective function as
\citep{tanizawa:mapping, duan:three, ben-elazar:spatial}. The
second samples local minima from a optimization problem leading to the study
of the population of local minima \citep{bau:three-dimensional}. The last
samples the posterior distribution \citep{rousseau:three}.

\citet{tanizawa:mapping} model the 3D genome of the fission yeast (3
chromosomes) by a string of $622$ beads, each bead $x_i$ being the center of a
$20$kb section. The first step was to infer physical distances $\delta_{ij}$
from frequency interactions. They studied eighteen pairs of genes using FISH
measurements, and fitted the Hi-C data on the distances with a non linear
regression curve. The second step was to compute the coordinates of the beads,
such that the distances between the beads matches the inferred physical
distances to the best, with additional biological motivated constraints.

\citet{duan:three} converts the interaction frequencies into distances by
examining the relationship between interaction frequencies and genomic
distances. Then, a multidimensional scaling (MDS) is used to place each bead
so that the wish distances are respected as well as possible.

\citet{tanizawa:mapping} and \citet{duan:three} optimizes a problem of the
form:
\begin{equation*}
\renewcommand{\arraystretch}{2}
\begin{array}{ccll}
\underset{x_1,\ldots, x_n}{\text{minimize}} & & 
\underset{i<j\leq n}{\sum} \big(\|x_i - x_j\|_2 - \delta_{ij}\big)^2 &\\
\text{subject to}
& & \text{biological motivated non convex constraints.}
\end{array}
\end{equation*}

\citet{tanizawa:mapping} published one solution, but did not mention the non
convexity of the problem. Hence, we assume they seeked the best local minimum
\citet{duan:three} ran the optimization process 30 times, and, observing the
obtained solutions, found that they did not differ much. No formal study was
done to compare the solutions.

\citet{ben-elazar:spatial} formulated a non metric multidimensional scaling
optimization problem. They first filtered the interaction count matrix so that
remained only the most significant interactions. They then interpolated the
missing values to obtain a smooth, symmetric, positive definite matrix.

\citet{bau:three-dimensional} used IMP (Integrative Modeling Platform), also
used in nuclear magnetic resonance (NMR) microscopy to construct a 3D model of
the $\alpha$-globin module. Chromosomes are represented by beads, each beads
linked by restraining oscillators. IMP seeks a solution at the equilibrium of
those beads. Three types of restraints are used: the first  corresponds to
harmonic oscillators, with strengths inversely proportional to the 5C
score, computed from the interaction frequencies. The second ensures that two
beads cannot be too close to each other. The third ensure that two consecutive
beads cannot be separated too much. The last two springs have strength only
when the constraints are not fulfilled. The optimization of this problem
yields different configuration with similar IMP scores. A population of 50000
structures was computed. The 10000 structures with the smaller objective
function were then chosen as the population of local minima to be studied.

\citet{rousseau:three} describes a formal probabilistic model of interaction
frequencies and their relationship with physical distances by hypothesizing
that interaction frequencies are inversely proportional to distances. They
then use a Markov Chain Monte Carlo sampling procedure on an optimization
problem that uses the same objective function as \citet{tanizawa:mapping} and
\citet{duan:three} to produce an ensemble of 3D structures.

Finally, \citet{tjong:physical} constructs a very simple model by
modeling chromosomes as a flexible fiber, and using additional biologically
motivated constraints, such as the positioning of centromeres and telomeres,
they formulate an optimization problem. Generating $200000$ feasible
structures, they show that Hi-C data can be fully explained by this very
simple model.

\begin{table}[ht!]
\caption{\bf A comparison of 3D inference methods}
\begin{center}
\begin{tabular}{lrcrr}
\hline
\emph{Publication} & \emph{Name} & \emph{Cons or Pop} & \emph{MDS-based} & Available \\
\hline
\citet{duan:three} & - & Cons & & Y \\
\citet{tanizawa:mapping} & & Cons & & N \\
\citet{ay:three-dimensional} & & Cons & & N\\
Nature methods & & Cons & & N\\
\citet{ben-elazar:spatial} & & Y & & N \\
\citet{varoquaux:statistical} & Y & N & & Y\\
\citet{bau:three-dimensional} & & & &\\
\citet{umbarger:three-dimensional} & & & &\\
\citet{zhang:inference} & chromSDE & & &\\
\citet{rousseau:three} & & & &\\
\citet{hu:bayesian} & & & &\\
\citet{kalhor:genome} & & & &\\
\citet{tokuda:dynamical} & & & &\\
\citet{wong:predictive} & & & &\\
\citet{gehlen:chromosome} & & & &\\
Alber & & & &\\
\end{tabular}
\end{center}
\end{table}


\section{Long range interactions}


Contact counts maps have recently been re-purposed for diverse applications,
far outside the study of the conformation of chromosomes: \textit{de novo}
genome assembly \citep{burton:chromosome, kaplan:high-throughput},
deconvolution of metagenomic samples \citep{burton:species-level,
beitel:strain}, and genome annotation \citep{marie-nelly:filling,
varoquaux:accurate}. Indeed, these maps contain long range contiguity
information that can be used in a wide range of applications.
In this section, we review the applications and methods used.


\section{Contributions}



% this file is called up by thesis.tex
% content in this file will be fed into the main document



\chapter{A statistical approach for inferring the 3D structure of the genome} % top level followed by section, subsection
\graphicspath{{2_chapter/figures/}}


\section{Motivation:} Recent technological advances allow the
measurement, in a single Hi-C experiment, of the frequencies of
physical contacts among pairs of genomic loci at a genome-wide
scale. The next challenge is to infer, from the resulting DNA-DNA
contact maps, accurate three dimensional models of how chromosomes
fold and fit into the nucleus. Many existing
inference methods rely upon {\em multidimensional scaling} (MDS), in
which the pairwise distances of the inferred model are optimized to
resemble pairwise distances derived directly from the contact
counts. These approaches, however, often optimize a heuristic objective
function and require strong assumptions about the biophysics of
DNA to transform interaction frequencies to spatial distance, and thereby
may lead to incorrect structure reconstruction.

\section{Methods:}
We propose a novel approach to infer a consensus three-dimensional
structure of a genome from Hi-C data. The method incorporates a
statistical model of the contact counts, assuming that the counts
between two loci follow a Poisson distribution whose intensity decreases
with the physical distances between the loci. The method can automatically
adjust the transfer function relating the spatial distance to the Poisson
intensity and infer a genome structure that best explains the observed data.

\section{Results:}
We compare two variants of our Poisson method, with or without
optimization of the transfer function, to four different MDS-based
algorithms---two metric MDS methods using different stress functions,
a nonmetric version of MDS, and ChromSDE, a recently described, advanced
MDS method---on a wide range of simulated datasets. We demonstrate that the
Poisson models reconstruct better structures than all MDS-based methods,
particularly at low coverage and high resolution, and we highlight the
importance of optimizing the transfer function. On publicly available Hi-C data
from mouse embryonic stem cells, we show that the Poisson methods lead to more
reproducible structures than MDS-based methods when we use data generated using
different restriction enzymes, and when we reconstruct structures at different
resolutions.
\section{Availability:} A Python implementation of the proposed method
is available at \href{http://cbio.ensmp.fr/pastis}{http://cbio.ensmp.fr/pastis}.


\section{Introduction}

Spatial and temporal three-dimensional (3D) genome architecture is
thought to play an important role in many genomic functions, but is
still poorly understood \citep{vansteensel:genomics}. In recent years, the
technique of chromosome conformation capture (3C)
\citep{dekker:capturing}, which identifies physical contacts between
different genomic loci and yields information about their relative
spatial distance in the nucleus, has paved the way for the systematic
analysis of the 3D structure of DNA. Coupled with high-throughput
sequencing, genome-wide conformation capture assays, broadly referred
to as {\em Hi-C} \citep{lieberman-aiden:comprehensive}, have emerged
as promising techniques to investigate the global structure of DNA at
various resolutions. Hi-C has opened new avenues to understanding many
biological processes including gene regulation, DNA replication,
somatic copy number alterations and epigenetic changes
\citep{shen:map,ryba:evolutionarily, de:DNA, dixon:topological}.

A typical Hi-C experiment yields a DNA {\em contact map}, that is, a
matrix indicating the frequency of interactions between all pairs of
loci at a given resolution.  A fundamental question is then to
reconstruct the 3D structure of the genome from this contact map. Two
general approaches have been proposed for that purpose: (i)
\emph{consensus methods} that aim at inferring a unique mean structure
representative of the data and (ii) \emph{ensemble methods} that
yield a population of structures.

Consensus approaches \citep{duan:three,
  tanizawa:mapping, bau:three-dimensional} model each
chromosome by a chain of beads, convert the contact map frequencies
into pairwise distances (which we refer as \emph{wish
  distances}) using various biophysical models of DNA, and infer a 3D
conformation that best matches the pairwise distances by solving a
multidimensional scaling (MDS) problem
\citep{kruskal:multidimensional2}. Converting interaction counts to
physical wish distances requires, however, strong assumptions which
are not always met in practice. For example, this mapping may change
from one organism to another \citep{fudenberg:higher-order}, from one
resolution to another \citep{zhang:inference}, from one genomic distance 
range to another \citep{ay:statistical}, or from one time point to 
another during the cell cycle \citep{le:high-resolution, ay:three-dimensional}.

To alleviate this problem, \citet{zhang:inference} proposed ChromSDE,
a method that jointly optimizes the 3D structure and a parameter of
the function that maps contact frequencies to spatial distances, in
addition to modifying the objective function of MDS. \citet{ben-elazar:spatial}
proposed an approach akin to \emph{nonmetric MDS}
\citep{kruskal:multidimensional}, where the 3D structure and the wish
distances are alternatingly optimized in an attempt to preserve coherence
between the ranking of pairwise distances and the ranking of pairwise
contact frequencies.

As for the ensemble methods, \citet{rousseau:three} and
\citet{hu:bayesian} describe two formal probabilistic models of contact
frequencies and their relationship with physical distances. They then
use a Markov chain Monte Carlo (MCMC) sampling procedure to produce an
ensemble of 3D structures consistent with the observed contact counts.
\citet{kalhor:genome} propose an optimization framework that generates
a population of structures by enforcing each contact to define an active
constraint in only a fraction of the inferred structures, thereby mimicking the
heterogeneity of contacts coming from each cell in the Hi-C sample.
Applying a similar method to budding yeast, \citet{tjong:physical}
demonstrate that a large population of structures inferred using known
physical constraints of yeast genome architecture can recapitulate, to a
large extent, the consensus contact map observed from Hi-C experiments.

Both consensus and ensemble models have benefits and
limitations. Ensemble approaches are biologically more accurate,
because Hi-C data is derived from a population of cells, each with
potentially a unique 3D architecture. An inferred population of 3D
structures may therefore better reflect the diversity of structures
than a single consensus structure.  In concordance with such ensemble
methods, a recent development in Hi-C
technology, assaying chromatin conformation at a single cell level,
demonstrates that chromatin structure varies highly from cell to cell
by modeling the single-copy X chromosomes of a male mouse cell line
\citep{nagano:single-cell}.

However, an ensemble approach
raises the question of interpretability: one often has to fall back to
interpreting a mean signal from the population structure
\citep{kalhor:genome} or to selecting a few structures, representative
in some way of the diversity of the population
\citep{rousseau:three}.  Consensus
methods, in contrast, provide a single structure more amenable to
visual inspection and analysis.  This structure can be seen as a useful
\emph{model} to recapitulate the rich information captured in Hi-C data
and to allow easy integration with other sources of data, such as
RNA-seq, which are usually also population based. In addition, despite
the stochasticity of cell-to-cell variations, certain hallmarks
of genome organization observed by consensus methods, such as chromosome
territories or topological domain organization, are conserved across
different cells \citep{nagano:single-cell,hu:bayesian}.
Computationally, ensemble methods are more demanding than consensus methods
since they need to sample from a very large dimensional space of possible
structures with complicated likelihood landscapes. Optimization-based
consensus methods are usually faster to converge to a local optimum,
but may miss the global optimum corresponding to the best structure when
the objective function is non-convex.


In this work, we focus on the consensus approach, and we propose a new
method to infer a 3D structure from Hi-C data. We propose to
replace the arbitrary loss function minimized by existing MDS-based
approaches by a better-motivated likelihood function derived from a
statistical model, similar to the one use by a previous ensemble method \citep{hu:bayesian}.
Specifically, our proposed method models the interaction frequency between two
loci by a Poisson model (PM), the intensity of which decreases with the
increasing spatial distance between the pair of loci.
Similar to the problem of inferring the wish distances
from interaction frequencies faced by MDS-based approaches, our model
faces the difficulty of transforming spatial distances into
intensities of the Poisson distribution. To solve this problem,
we propose two variant methods. The first method (PM1) uses a default
transfer function motivated by a biophysical model, whereas the second
method (PM2) uses a parametric family of transfer functions, the parameters
of which are automatically optimized together with the 3D structure to best
explain the observed data.

We compare both PM variants to four MDS-based methods, including metric
MDS with two stress functions, nonmetric MDS and ChromSDE. We demonstrate
on simulated data that the new models reconstruct more accurate 3D structures
than all MDS-based methods, especially at low coverage and high resolution.
We also assess the negative effect of using an incorrect transfer function,
and we show that PM2 is able to overcome this difficulty. On real data, we show
that, compared to MDS-based methods, PM1 and PM2 generate more similar models
when applied to replicate experiments performed with different restriction enzymes
or when applied to the same data at varying resolutions. The results suggest that
the Poisson model methods we describe here provide promising alternatives to
current methods for consensus DNA structure inference.

\section{Approach}

We model chromosomes as series of beads in 3D, each bead representing
a genomic window of a given length, and we denote by $\Xb =
(x_1,\ldots,x_n) \in \RR^{3 \times n}$ the coordinate matrix of the
structure, where $n$ denotes the total number of beads in the genome
(for example, $n=1216$ at 10kb resolution for the yeast genome) and
$x_i\in\RR^3$ represents the 3D coordinate of the $i$-th bead. The Hi-C
data can be summarized as an $n$-by-$n$ matrix $\mathbf{c}$ in which
each row and column corresponds to a genomic locus, and each matrix
entry $c_{ij}$ is a number, called the {\em contact frequency} or {\em
  contact count}, indicating the number of times locus $i$ and $j$
were observed to contact one another. The matrix is by construction
square and symmetric.

\subsection{Data normalization}
The raw contact count matrix suffers from many biases, some technical (from
the sequencing and mapping) and others biological (inherent to the physical
properties of chromatin) \citep{yaffe:probabilistic, imakaev:iterative}.
Therefore, before inferring the 3D structure of the genome, we normalize each
raw contact matrix using iterative correction and eigenvalue decomposition
(ICE) \citep{imakaev:iterative}, a method based on the assumption that all
loci should interact equally. Due to mappability issues, some beads have zero
contact counts. We remove these beads from the optimization and only try to
infer the positions of beads with nonzero contact counts.


\subsection{MDS-based methods}

\subsubsection{Metric MDS}

Metric MDS is a classical method to infer coordinates of points given their
approximate pairwise Euclidean distances~\citep{kruskal:multidimensional2}. To
use MDS in the context of DNA structure inference from Hi-C data, we need to
assign each pair of beads ($i$, $j$) a physical wish distance
$\delta_{ij}$---i.e., the distance that we aim to capture with our 3D
model---derived from the bead pair's contact count $c_{ij}$. Performing this
assignment requires us to decide how contact counts are transformed into physical
distances. In Section~\ref{sec:evaluating_parameters} we discuss a commonly used
transformation of the form $\delta_{ij} = \gamma
c_{ij}^{-3}$ if $c_{ij}>0$ motivated by polymer physics.
Metric MDS then places all the beads in 3D space such that
the Euclidean distance $d_{ij}(\Xb) = \|x_i - x_j\|$ between the beads $i$ and
$j$ is as close as possible to the wish distance $\delta_{ij}$. Denoting by
$\Dcal$ the subset of indices whose distances we wish to constrain (typically,
the set of pairs $(i,j)$ with non-zero contact counts $c_{ij}>0$), metric MDS
attempts to minimize the following objective function, usually called the \emph{raw
stress}:
\begin{equation}\label{eq:mds1}
\renewcommand{\arraystretch}{2}
\begin{array}{ccll}
\underset{\mathbf{X}}{\text{minimize}} & &
\underset{(i,j) \in \mathcal{D}}{\sum} \big(d_{ij}(\Xb) - \delta_{ij}\big)^2 \,.&\\
\end{array}
\end{equation}

In two previous studies that use metric MDS,
\citet{duan:three} and \citet{tanizawa:mapping} infer the 3D structure of DNA
from Hi-C data by solving Equation \ref{eq:mds1}, limiting $\Dcal$ to pairs of
indices with statistically significant contact counts (FDR 0.01\%). Both methods use
additional constraints such as
minimum and maximum distances between adjacent beads, minimum pairwise
distances between arbitrary beads to avoid clashes, and organism-specific constraints that concern the
positioning of centromeres, telomeres and ribosomal RNA coding regions. In
the experiments we present here, we simply solve Equation~\ref{eq:mds1} without
any constraints but including all pairs of beads with positive counts in $\Dcal$, and we call the
resulting method MDS1. In general, we have observed that
adding constraints related to minimal and maximal distances between beads
is unnecessary, because the structures found by MDS1 typically fulfill all of these constraints (data not shown).

A drawback of the raw stress of Equation~\ref{eq:mds1} in our context is that the
quadratic form is dominated by large values, corresponding to pairs of loci
with large wish distances (i.e., small contact counts). Because these counts are
less reliable than large contact counts, we propose a variant of MDS1, which
we call MDS2, where we weight the contribution of a pair $(i,j)$ in the stress
by a factor inversely proportional to the square wish distance between the corresponding
beads:
\begin{equation}\label{eq:mds2}
\renewcommand{\arraystretch}{2}
\begin{array}{ccll}
\underset{\mathbf{X}}{\text{minimize}} & &
\underset{(i,j) \in \mathcal{D}}{\sum} \delta_{ij}^{-2} \big(d_{ij}(\Xb) - \delta_{ij}\big)^2 \,.&\\
\end{array}
\end{equation}
While other weighting schemes could be proposed to decrease the influence of
pairs with large wish distances, we found this formulation to be quite robust
in practice. Notice that MDS2 can be thought of as a quadratic approximation of
the raw stress (minimized by MDS1) applied to log-transformed distances, because
in the setting $d_{ij}(\Xb) \approx \delta_{ij}$ it holds that:
\begin{equation*}
\begin{split}
\sum_{(i,j) \in \mathcal{D}} \br{\log d_{ij}(\Xb) - \log \delta_{ij}}^2
& = \sum_{(i,j) \in \mathcal{D}} \log\br{\frac{d_{ij}(\Xb)}{\delta_{ij}}}^2 \\
& \approx \sum_{(i,j) \in \mathcal{D}} \br{ \frac{d_{ij}(\Xb)}{\delta_{ij}}-1}^2 \,.
\end{split}
\end{equation*}
Both MDS1 and MDS2 implicitly
ignore non-interacting pairs of beads (i.e., pairs with zero contact
counts).

In addition to MDS1 and MDS2, we include in our benchmark ChromSDE
\citep{zhang:inference}, a recently proposed method which also attempts to
minimize a weighted stress function penalized by an additional term to push
non-interacting pairs far from each other. In addition, ChromSDE optimizes the
exponent of the transfer function that maps from contact counts to
wish distances. However, it
does not infer the relative positions of chromosomes. Accordingly, we compare
only the reconstruction of each individual chromosome produced by each method. Note
that, because intra-chromosomal counts are more reliable than inter-chromosomal
counts, ChromSDE should not be penalized compared to the other methods by
only considering intra-chromosomal counts.

\subsubsection{Nonmetric MDS (NMDS)}
\label{sec:nmds}

The derivation of the transfer function from contact counts to 3D wish distances, 
needed by metric MDS-based methods, relies on strong
assumptions about the physics of DNA
(Section~\ref{sec:evaluating_parameters}). NMDS \citep{shepard:analysis,kruskal:multidimensional} offers an alternative 
way to proceed, which was proposed in the context of DNA structure inference 
from Hi-C data by \citet{ben-elazar:spatial}. Instead of inferring physical
distances from the contact matrices, NMDS relies on the sole
hypothesis that if two loci $i$ and $j$ are observed to be in contact more
often than loci $k$ and $\ell$, then $i$ and $j$ should be closer in 3D space
than $k$ and $\ell$.  Using this hypothesis, NMDS attempts to solve the
following problem:
\begin{problem}
Given a set of similarities $c_{ij}$ (e.g., the contact frequency between $i$ and $j$), find 
$\mathbf{X} \in R^{3 \times n}$ such that:
\begin{equation}
\label{eq:ordinal_constraint}
c_{ij} \geq c_{k\ell} \Leftrightarrow \|x_i - x_j\|_2 \leq \|x_k -
x_\ell\|_2 \,.
\end{equation}
\end{problem}
Equation~\ref{eq:ordinal_constraint} is known as the nonmetric
constraint, or the ordinal constraint. This problem was first introduced
by \citet{shepard:analysis} and formalized as an optimization problem
by \citet{kruskal:multidimensional}. It can be solved by minimizing the
cost function:
\begin{equation}
\underset{\mathbf{X,\Theta}}{\text{minimize}}
\  \sum_{i, j} \frac{\br{\|x_i - x_j\|_2 -
\Theta(c_{ij})}^2}{\Theta(c_{ij})^2},
\end{equation}
with respect to the embedding $\mathbf{X}$ and the function $\Theta$,
where $\Theta$ is a decreasing function. Algorithms to solve this
optimization problem involve iterating over two steps: (1) fixing
$\Theta$ and minimizing the objective function with respect to $\mathbf{X}$ (hence falling
back to solve MDS2), and (2) fitting $\Theta$ to the
new configuration $\mathbf{X}$ subject to the ordinal
constraints. This second step of the algorithm can be performed using an
isotonic regression method, such as the pool adjacent violator
algorithm \citep{best:minimizing}.

A trivial solution of this problem is to set $\Theta$ equal to
$0$. In this case all points will collapse on the origin. To avoid this collapse,
we add additional constraints on $\mathbf{X}$ or on
$\Theta$, such as $\sum_{i, j} \|x_i - x_j\|_2 = K$ for some constant
value of $K$.

\subsection{Poisson model}
\label{sec:pm}

Instead of metric or non metric MDS-based methods, which attempt to minimize a stress function that measures a discrepancy between the wish distances and the 3D distances of the structure, we propose to cast the problem of structure inference as a maximum likelihood problem. For that purpose, we need to define a probabilistic model of contact counts parametrized by the 3D structure that we want to infer from contact count observations.

For that purpose, we take a model similar to the one used in the BACH algorithm~\citep{hu:bayesian} and model the contact frequencies $(c_{ij})_{(i,j)\in\Dcal}$ as independent Poisson random variables, where the Poisson parameter of $c_{ij}$ is a decreasing function of $d_{ij}(\Xb)$ of the form $\beta d_{ij}(\Xb)^\alpha$, for some parameters $\beta>0$ and $\alpha<0$.
We can then express
the likelihood as
\[
\ell(\mathbf{X}, \alpha, \beta) =
  \prod_{i, j} \frac{(\beta d_{ij}^{\alpha})^{c_{ij}}}{c_{ij}!}
	\exp (- \beta d_{ij}^{\alpha})\,.
\]
By maximizing the log likelihood, a new optimization problem naturally
emerges from this formulation:
\begin{equation}
\renewcommand{\arraystretch}{2}
\begin{array}{cll}
\underset{\alpha, \beta, \textbf{X}}{\text{max}} &
\mathcal{L}(\mathbf{X}, \alpha, \beta) = \underset{i<j\leq n}{\sum}  c_{ij}
\alpha \log d_{ij} + c_{ij} \log \beta - \beta d_{ij}^\alpha &\\
\end{array}
\end{equation}
With this new formulation, we can either provide the parameter $\alpha$, using
prior knowledge, and only optimize the structure and $\beta$ (which depends on
the dataset), or we can use a nonmetric approach, by inferring $\alpha$.
We refer to the former as PM1 and to the latter as PM2.

PM2 is solved using a coordinate-descent algorithm:
first choose randomly an $\mathbf{X}$ configuration, then iterate
between maximizing $\mathcal{L}$ with respect to $\alpha$ and $\beta$
 and, fixing $\alpha$ and $\beta$ and maximizing $\mathcal{L}$
with respect to $\mathbf{X}$.  In this work, we try to
initialize $\textbf{X}$ with a good approximation of the solution by
first evaluating the parameters $\alpha$ and $\beta$ using some prior
knowledge and initialize $\textbf{X}$ with the inferred structure from
the MDS.

%\begin{equation}
%\frac{\partial \mathcal{L}}{\partial{\alpha}}(\alpha, \beta, \textbf{X}) =
%  \sum_{i, j \in \mathcal{D}} c_{ij} \log d_{ij} - \beta  d_{ij}^{\alpha} \log d_{ij}
%\end{equation}

%\todo{If space allows, we could write some more equations, like the minimum in $\beta$ or the gradients.}
%\todo{We should change the name EM (expectation-minimization) because what we do is quite different from standard EM. In standard EM, one would model observed data (count matrix C) and hidden variable (X) through a parametric model $P_a(C,X)$. To optimize the parameter a, one would first perform a E step where we would compute $E_a(X|C)$ , and then adjust the parameters. What we do is not computing the expectation of the structure, but instead the best structure (that maximizes $P_a(X|C)$, assuming a uniform prior on X) before reoptimizing the weights. To make a comparison with a more classical model, it is as if for hidden markov models we would run a Viterbi to find the most likely hidden variables, then optimize the parameters to maximize the likelihood of the hidden variables, as opposed to the EM approach (Baum-Welsh) where instead of Viterbi we need to estimate the expectation of the hidden variables. Mathematically, the EM procedure ensures that the likelihood of the observations observe, while what we do does not. By the way, a real EM may be relevant, in fact it would probably involve sampling a family of structures (as in the population methods), then do the average distances over the populations and optimize the parameters on that (details to be checked...).}

All optimization problems (MDS1, MDS2, NMDS, PM1 and PM2) were solved using IPOPT, an interior point filter
algorithm \citep{wachter:on} and the isotonic regression implementation from
the Python toolbox Scikit-Learn for NMDS \citep{pedegrosa:scikit}.


\subsection{Default contact-to-distance transfer function}
\label{sec:evaluating_parameters}

A prerequisite for both the MDS and the PM1 model (and for good initialization
of the NMDS and PM2 methods) is a function that converts from contact counts
to wish distances.  Extensive previous studies of the behaviour of polymers in
general and DNA in particular have yielded proposed relationships between, on
the one hand, the genomic distance $s$ and contact counts $c$ and, on the
other hand, genomic distance $s$ and physical distances $d$ for several
classes of polymers \citep{grosberg:role, lieberman-aiden:comprehensive,
fudenberg:higher-order}. For a fractal globule polymer, representative of
mammalian DNA, the contact count is inversely proportional to the genomic
distance ($c \sim s ^{-1}$), whereas the volume scales linearly with the
subchain length ($d^3 \sim s$), from which we deduce a relationship between
$d$ and $c$ of the form $d \sim c ^{-1/3}$. For an equilibrium globule,
representative of a smaller genome such as {\em S.\ cerevisae}, the
relationships differ: $c \sim s ^{-3/2}$ and $d \sim s ^ {1 / 2}$ up to a
maximum distance, corresponding to the size of the nucleus in which the DNA is
confined. Conveniently, coupling those two relationships for either type of
polymer yields the same mapping between contact counts and physical distances:
\begin{equation}
d \sim c ^{-1/3}.
\end{equation}
Thus, by default we convert contact counts $c_{ij}$ into 3D wish distances
$\delta_{ij}$ using the following relationship:
\begin{equation}
\delta_{ij} = \gamma c_{ij} ^{-1/3},
\end{equation}
where $\gamma$ defines the scale of the structure. It is important to note that 
this relationship holds true for only a subset of the full genomic distance range and
that this range varies for different genomes. In practice, we will not
infer $\gamma$ for the MDS and NMDS problem: the structures can easily be
rescaled after convergence to match biological knowledge of the organism
studied.

\subsection{Data}
\label{sec:data}

In order to test various 3D architecture inference methods, we
conducted experiments on both simulated datasets and publicly
available genome-wide Hi-C datasets.

For the simulation, we generated 170 data sets using the yeast genome
architecture proposed by \citet{duan:three}. Because the repetitive
rDNA on yeast chromosome XII cannot be observed in practice, we
discard all contacts involving these loci, and we do not infer the
position of the corresponding rDNA. We generate these 170 datasets
using the following model:
\begin{equation}\label{eq:simu}
c_{ij} = P(\beta d_{ij}^\alpha),
\end{equation}
where $\alpha = -3$ (corresponding to the theoretical exponent discussed in Section~\ref{sec:evaluating_parameters})
and $\beta$ varies between $0.01$ and $0.7$ ($0.01$, $0.01$, $0.02$, $0.03$,
$0.04$, $0.05$,
$0.06$, $0.07$, $0.08$, $0.09$, $0.1$, $0.15$, $0.2$, $0.3$, $0.4$, $0.5$, $0.6$,
$0.7$) with 10
different random generator seeds, thus obtaining 10 different datasets per
parameter.
The $\beta$ parameter controls the number of contact counts in the
datasets. A low $\beta$ will yield a dataset with few counts; hence, the
corresponding wish distance matrix will be less likely to be close to the true
distance matrix. To estimate how noisy the generated data is, we compute the
following measure of signal-to-noise ratio (SNR):
\begin{equation}
SNR = \frac{\sum{c_{ij}}}{\sqrt{\sum (\beta d_{ij} ^{\alpha} - c_{ij})^2}}\,.
\end{equation}
The numerator (the signal) corresponds to the number of counts, and
the denominator (the noise) corresponds to the sum of deviation
between each count and its expected value.  We use this first ensemble
of simulated datasets to assess the robustness to noise of the
different methods. Note that in actual data, the SNR gets smaller when we sequence fewer reads or when we infer a structure at a higher resolution.

We simulated another ensemble of datasets to compare nonmetric and metric
methods when the parameters provided to the different algorithms are not
the correct ones. We generate 20 datasets according to
Equation~\ref{eq:simu}, with $\alpha$ between $-4$ and $-2$ ($-4$, $-3.5$,
$-3$, $-2.5$, $-2$) and $\beta$ between
$0.4$ and $0.7$ ($0.4$, $0.5$, $0.6$, $0.7$).

We also applied our methods to publicly available Hi-C data from mouse
embryonic stem cells (mESC) \citep{dixon:topological}.  We started with the
data at 20~kb resolution and considered only chromosomes 1 to 19, with both
available restriction enzymes (HindIII and NcoI).  We then subsampled the data
at resolutions of 100~kb, 200~kb, 500~kb and 1~Mb.
Note that the methods studied here infer a single copy per
chromosomes, thus yielding a consensus model for both homologous chromosomes.


\subsection{Structure similarity measures}
In order to assess the ability of a method to reconstruct a known
structure from simulated data, or the stability of the reconstructed structure with respect to change in resolution or library preparation, we need quantitative measures of similarity between 3D structure. We use two such measures: the root mean square deviation (RMSD) and the distance error, which we now explain.

The RMSD is a
standard way to compare two sets of structures described by their
coordinates $\mathbf{X}, \mathbf{X}' \in R^{3 \times n}$, widely used for example to compare protein 3D structures. It is given by:
\begin{equation*}
RMSD = \min_{\mathbf{X}^*} \sqrt{\sum_{i = 1}^n (\mathbf{X}_{i} -
  \mathbf{X}_{i}^*)^2} \,,
\end{equation*}
where the structure $\mathbf{X}^*$ is obtained by translating,
rotating and rescaling $\mathbf{X}'$ ($\mathbf{X}^* = s \mathbf{R X'} -
\mathbf{t}$, where $\mathbf{R} \in R^{3 \times 3}$ is a rotation
matrix, $\mathbf{t} \in R^{3}$ is a translation vector, and $s$ is a
scaling factor). Because ChromSDE
does not infer the relative position of chromosomes, the RMSD values we report below
are sums of RMSDs computed independently on each chromosome.

We also directly compare the 3D distance matrices corresponding to the two structures with the distance error:
\begin{equation*}
\text{distanceError} = \sqrt{\sum_{i, j = 0}^n (d_{ij}(\Xb) - d_{i, j}(\Xb'))^2} \,.
\end{equation*}


The main difference between the optimization formulated by ChromSDE
and those of the other methods is the penalty assigned to
non-interacting beads.  Due to this penalty, ChromSDE should recover
better long distances than other MDS-based methods.  This property is
not well captured by the RMSD measure, therefore, we also compute how
well the distance matrix is recovered with the distance error, which assigns most of the weight to long distances. We
expect that methods based on MDS, which optimize an objective function
based on the distance matrix, should perform better on this measure
than others.

\section{Results}

To assess the relative strength of our new Poisson model-based methods, PM1 and PM2, we compare them to a panel of four MDS-based methods: MDS1, MDS2, NMDS and ChromSDE on simulated and real data.

\subsection{Simulated Hi-C data}

We first tested the six methods on data simulated as explained in Section~\ref{sec:data}.

\begin{figure}
\includegraphics[width=\linewidth]{{varoquaux.78.fig1}.pdf}
\caption{\textbf{Performance evaluation on simulated data, varying the parameter $\beta$.}
\textbf{A} RMSD of each experiment for varying values of the parameter $\beta$. ChromSDE
failed to yield consistent results for 14 experiments (It reported the wrong number of beads
in the results file.), and the PM2 algorithm
failed to converge at the desired precision for one experiment (It exceeded the
maximum number of iterations.).
\textbf{B} Distance error of each experiment for varying values of $\beta$.
\textbf{C} Average SNR for each $\beta$. Higher SNR corresponds to better quality data.
}
\label{fig:generated_data}
\end{figure}

\subsubsection{Performance as a function of SNR}

We ran all six methods---MDS1, MDS2, NMDS, PM1, PM2 and ChromSDE---on the 170
simulated datasets with varying SNR levels. Our goal here is to assess how
well the different methods manage to reconstruct a known 3D structure from
simulated data at different SNR levels. Remember that SNR estimates how far
the empirical counts differ from their expectations; in real Hi-C data, SNR
typically decreases when we have fewer reads in total, or when we want to
increase the resolution of the structure. In this first series of experiments,
we provide the correct count-to-distance or distance-to-count transfer
functions to the methods that need them (MDS1, MDS2, PM1). In this setting,
for infinite SNR, all methods should consistently estimate the correct
structure.

Figure~\ref{fig:generated_data} shows the performance of the different methods
in terms of RMSD (top) and distance error (middle) as a function of the
$\beta$ parameter, which controls the SNR (bottom). As expected, all methods
perform well when the SNR is high, but exhibit marked differences in
performance for finite SNR. In the low SNR setting (SNR $< 2$), both PM1 and
PM2 significantly outperform all MDS-based methods, in both RMSD and distance
error. Interestingly, we observe no significant difference between PM1 and
PM2, which shows that there is no price to pay in terms of inferred structure
if we don't specify the exponent of the distance-to-count transfer function.
In this setting, PM2 is able to estimate the structure accurately enough to
produce a structure of the same quality as PM1. Among MDS-based methods, we
see that NMDS generally outperforms MDS2, which itself outperforms MDS1. This
observation highlights that in the non-asymptotic, low SNR setting, the choice
of stress function influences the performance of MDS. ChromSDE performs better
than other MDS-based methods on datasets with a low SNR, corresponding to
datasets with low coverage and, consequently, many non-interacting pairs of
beads. This may be due to the way ChromSDE explicitly handles such pairs. On
the other hand, in a more favorable setting (SNR $>2$), ChromSDE does not
perform as well as other MDS-based method; we hypothesize that when the
coverage is high enough, taking into account non-interacting pairs of beads
does not add any additional information. Since ChromSDE is not better than
other MDS-based methods, and requires much longer to run, we do not report its
performance on the next experiments and instead focus on the differences
between the other MDS-based methods and the PM methods.

\subsubsection{Metric versus nonmetric methods: robustness to incorrect parameter estimation}

\begin{figure}
\includegraphics[width=\linewidth]{{varoquaux.78.fig2}.pdf}
\caption{\textbf{Performance evaluation for simulated data, varying
    the parameter $\alpha$.} The figure plots the average RMSD of the
  inferred structures for a range of $\alpha$ values.  As $\alpha$
  increases, the SNR of the dataset also increases.}
\label{fig:rmsd_alpha}
\end{figure}

Three of the methods tested, which we collectively refer to as \emph{metric} methods, require as input a count-to-distance or distance-to-count transfer function: MDS1, MDS2 and PM1. In reality, however, the DNA may not follow the ideal physical laws underlying the default transfer function discussed in Section~\ref{sec:evaluating_parameters}, and the structures inferred from these methods may diverge from the correct one because of miss-specification of the transfer function.

To assess this phenomenon, and evaluate the robustness of the different methods (including NMDS and PM2, which automatically infer a transfer function), we now study
the performance of the methods on datasets generated with varying
$\alpha$ parameters. We therefore run the MDS1, MDS2, NMDS, PM1 and PM2 methods on the second
ensemble of simulated datasets. We provide the default transfer function to
all metric methods, thus inducing a miss-specification for all simulated
datasets with $\alpha\neq -3$.

Figure~\ref{fig:rmsd_alpha} shows the RMSD of each method, averaged
over the datasets with different $\beta$, as a function of $\alpha$.
The performance curve of PM1, which is the best method when the data
are simulated with the correct parameter $\alpha=-3$, exhibits a
characteristic U-shape centered around $\alpha = -3$. This curve
confirms that PM1 performs better when given the true parameter and
performs worse as $\alpha$ moves away from $-3$. On the other hand,
the performance curves of the two other metric methods, MDS1 and MDS2,
do not exactly follow this trend: MDS1 and NMDS perform increasingly
better when $\alpha$ decreases, and MDS2 achieves the best performance
when $\alpha = -3.5$. This phenomenon occurs because in our
simulation, when $\alpha$ decreases, the SNR for a given $\beta$
increases, counterbalancing the negative effect of the transfer
function miss-specification.  Thus, for MDS-based methods, it is
apparently more important to have more data than to have a correct
$\alpha$ parameter. Finally, we see that, as expected, the non-metric
approaches, NMDS and PM2, are more robust to transfer function
misspecification than the metric approaches, because they
automatically estimate it. When the parameter is wrong, PM2
outperforms the other methods for low SNR, whereas for high SNR, NMDS
performs better.

\subsection{Real Hi-C data}

We now test the different methods on real Hi-C data. Since in this case the
true consensus structure is unknown, we investigate the behaviors of the
different methods in terms of their ability to infer consistent structures
from different datasets and across resolutions.

\begin{table*}
\begin{center}
\begin{tabular}{ccccccccccccc}
\hline
Resolution &  Corr &
\multicolumn{2}{c}{MDS1} &
\multicolumn{2}{c}{MDS2} &
\multicolumn{2}{c}{NMDS} &
\multicolumn{2}{c}{PM1} &
\multicolumn{2}{c}{PM2} \\
&&& RMSD & Corr & RMSD & Corr & RMSD &Corr & RMSD & Corr & RMSD & Corr \\
\hline
1~Mb  & 0.981 & 13.13 & 0.945 & 5.54 & 0.964 &  5.80 & 0.965 & 7.28  & 0.931 & \textbf{4.92} & \textbf{0.976} \\
500~kb & 0.959 & 10.00 & 0.942 & 5.68 & 0.959 & 5.67 & 0.959 & 7.14 & 0.913 & \textbf{4.66} & \textbf{0.968} \\
200~kb & 0.845 & 5.64 & 0.940 & 3.74 & 0.945 & 3.73 & 0.946 & 4.01 & 0.891 & \textbf{3.42} & \textbf{0.958} \\
100~kb &  0.605 & 5.07 & 0.736 & 2.53  & 0.676 & 2.52 & 0.666 & \textbf{2.51} & 0.664 & 2.76 & \textbf{0.771}\\
\hline
\end{tabular}
\end{center}

\caption{\textbf{Stability across enzyme replicates.} For each resolution, the
table lists the Spearman
correlation the two enzyme replicate datasets, and, for each inference
method, the average RMSD and Spearman correlation between pairs of structures
inferred from the two datasets.  Boldface values correspond to the best RMSD
or correlation values among all five methods.  In general, higher resolution
leads to a lower correlation between pairs of inferred structures.}

\label{table:results_real_data}
\end{table*}


\begin{table}
\begin{center}
\begin{tabular}{clllll}
\hline
   & MDS1 & MDS2 & NMDS & PM1 & PM2 \\
\hline
{\em RMSD }        & 14.86 & 12.92 & 12.98 & 13.03 & \textbf{11.48} \\
{\em Correlation } &  0.781 & 0.754 & 0.738 & 0.737 & \textbf{0.807} \\
\hline
\end{tabular}
\end{center}
\caption{\textbf{Stability across resolution.} The table lists the
  average RMSD and Spearman correlation between pairs of structures of
  different resolutions. In bold are the lowest average RMSD and
  highest average Spearman correlation. These values were computed on
  mouse ESC HindIII libraries \cite{dixon:topological})}
\label{table:real_hic_stability}
\end{table}

\begin{figure*}
\begin{center}
\includegraphics[width=0.8\linewidth]{{varoquaux.78.fig3}.png}
\end{center}
\caption{\textbf{Predicted structures for chromosome 1 at different resolution}
Contact counts matrices and predicted
structures for the MDS2, NMDS, PM1 and PM2 methods at 1~Mb (\textbf{A}),
 500~kb (\textbf{B}), 200~kb (\textbf{C}),
100~kb (\textbf{D})}
\label{fig:real_data}
\end{figure*}

\subsubsection{Stability to enzyme replicates}

The Hi-C assay depends upon a restriction enzyme to cleave the DNA after
cross-linking, and the same sequence library can be analyzed multiple times
using different enzymes.  Although the resulting restriction fragments will
differ, we expect {\em a priori} that the overall genome architecture should
be the same from such replicate experiments.  We therefore evaluate each
genome architecture inference method with respect to the similarity of the
structures inferred from two replicate Hi-C experiments that differ only in
the choice of restriction enzyme.  Specifically, we apply each method to two
enzyme replicates, HindIII and NcoI, carried out in mouse ES cells
\citep{dixon:topological} for chromosomes 1--19.

To measure the stability of the methods, we compute (1) the Spearman
correlation between the two pairwise Euclidean distance matrices of
the pairs of predicted structures and (2) the RMSD between the
rescaled predicted structures.  Note that, before computing our two
error measures, we filter out from the pair of structures any beads
for which the inference hasn't been done on either dataset, i.e.,
beads that have zero contact counts in either data set.

To give a sense of how similar the two replicate datasets are, we also
compute the Spearman correlation directly on the data, rather than on
the inferred structures.  As expected
(Table~\ref{table:results_real_data}), the higher the resolution is,
the lower the correlation between the pairs of datasets is and the
more different the inferred structures are.
Across different enzyme replicates, the PM2 method yielded significantly
higher correlation than all of the other methods ($p < 0.05$,
signed-rank test
adjusted for multiple tests with a Bonferroni correction).



\subsubsection{Stability to resolution}

\citet{zhang:spatial} show that the mapping from contact counts to
physical distance differs from one resolution to another, underscoring
the importance of good parameter estimation. To study the stability of
the structure inference methods to changes in resolution, we compute
the RMSD between pairs of structures inferred at different
resolutions.  Let $(\textbf{X}, \textbf{Y}) \in (R^{3 \times n}, R^{3
  \times m})$ be a pair of predicted structures such that $n < m$
(i.e., $\textbf{X}$ is a structure at a lower resolution than
$\textbf{Y}$). We compute a downsampled structure $\textbf{Y}^* \in
X^{3 \times n}$ at the same resolution as $\textbf{X}$ by averaging
the coordinates of beads.  We then compute the RMSD between this new
structure $\textbf{Y}^*$ and $\textbf{X}$, as well as a corresponding
Spearman correlation of the distance matrices.

Results are shown in Figure~\ref{fig:real_data} and
Table~\ref{table:real_hic_stability}. PM2 is significantly ($p < 0.05$)
more stable to resolution changes, both in terms of RMSD and correlation of
distances.

\section{Discussion and conclusion}

In this work, we present a novel method for inferring a consensus genomic 3D
structure from Hi-C data. The method maximizes a likelihood derived from a
statistical model of the relationship between the contact counts and physical
distances, and includes an automatic tuning of the parameters defining the
link between a 3D distance and the Poisson parameter of the corresponding
contact count. We showed in simulations that the new method outperforms a
panel of MDS-based approaches, including ChromSDE, which optimize an often
ad-hoc stress function. The improvement is particularly important at low SNR,
corresponding to more difficult problems where we want to increase the
resolution of the model with a fixed total number of reads; this is typically
the situation where one expects a correct maximum likelihood estimator to
outperform more {\em ad hoc} estimators. We also showed that misspecification
in the count-to-distance transfer function can harm the performance of metric
methods, while our model can adapt to unknown distributions within a
parametric family. Finally, we also demonstrated, on real Hi-C data, the
robustness of our methods to resolution change and enzyme duplicated datasets.

Our probabilistic model of reads is similar to the model proposed
by~\citet{hu:bayesian}; however, instead of generating a family of structures
by MCMC we use the model for direct maximum likelihood estimation of a
consensus structure. Although the consensus structure might not be a
definitive structure {\em in vivo}, it provides us with a rich model for
further analysis, conserving hallmarks of genome organization such as the
water lily form of the budding yeast \citep{duan:three} or topological domains
\citep{kalhor:genome}.

The Poisson model underlying our approach remains very basic and could be
subject to many improvements.  For example, physical constraints, such as the
size of the nucleus, could be incorporated into the model. Better models for
zero entries may be possible, because those can either come either from
non-interacting loci or from measurement errors due to, e.g., mappability
problems. Overall, expressing the structure inference problem as a maximum
likelihood problem offers a principled way to improve the method by improving
the probabilistic model of measured dat.
               

% this file is called up by thesis.tex
% content in this file will be fed into the main document

\chapter[Genome architecture of the \emph{P. falciparum}
genome]{Three-dimensional modeling of the {\em P. falciparum} genome during
the erythrocytic cycle reveals a strong connection between genome
architecture and gene expression} % top level followed by section, subsection
\label{chap:plasmodium}

\graphicspath{{3_plasmodium/}}

\begin{work}

This chapter has been published in a slightly modified form in
\citep{ay:three-dimensional}, as joint work with Evelien Bunnik, Ferhat Ay,
Sebastiaan Bol, Jacques Prudhomme, Jean-Philippe Vert, Bill Noble and Karine
Le Roch.

\end{work}

\begin{abstract}{Résumé}

Dans ce projet, nous nous intéressons à l'architecture du parasite {\em
Plasmodium falciparum}, responsable de la forme la plus virulente et mortelle
du paludisme chez l'homme. Le développement de ce parasite est contrôlé par
des changements précis et coordonnés dans l'expression de ses gènes au cours
son cycle cellulaire. Les mécanismes régulant ces changements sont à l'heure
actuelle peu connus. Nous nous intéressons dans ce travail au lien entre
l'architecture spatiale du génome et la régulation des gènes.
Nous étudions la conformation du {\em P. falciparum} à trois moments de
son cycle cellulaire asexué érythrocyte (cycle cellulaire du parasite lors de
sa présence dans les cellules sanguines humaines). Grâce au protocole de
capture de la conformation des chromosomes associé au séquençage haut débit,
nous obtenons des cartes haute résolution des fréquences de contact entre
paires
de loci, à partir desquelles nous construisons des structures consensus
tridimensionnelles pour chaque étape du développement cellulaire du parasite.
Nous observons dans ces modèles une forte colocalisation des centromères,
des télomères, de l'ADN ribosomal, ainsi que les gènes "virulence". Ces
contraintes conduisent à une architecture complexe du génome, qui ne peut être
simplement expliquée par un modèle de volume d'exclusion comme celui de la
levure. Par ailleurs, les cartes de contacts exhibent des domaines
particuliers à la position des clusters internes de gènes "virulence",
suggérant l'importance du rôle de ces gènes dans l'architecture du génome.
Lors de l'état trophozoite, à mi chemin dans le cycle erythrocytique alors que
le génome est très fortement transcrit, celui-ci adopte une conformation plus
ouverte, et les chromosomes interagissent plus entre eux. Nous observons de
plus que les gènes à proximité des centres répressifs
sous-télomériques sont sous-exprimés. Par ailleurs, la colocalisation de
groupes de gènes spécifiques au parasite, tels que des gènes impliqués dans
l'invasion des cellules sanguines humaines, ont des profils d'expression
proches. Toutes ces observations suggèrent une très forte association entre
l'organisation spatiale du génome de {\em P. falciparum} et l'expression de
ses gènes. Une meilleure compréhension des processus biologiques impliqués
dans la dynamique de la conformation du génome pourrait contribuer à la
découverte de nouvelles stratégies pour combattre le paludisme.
\end{abstract}


\begin{abstract}{Abstract}
The development of the human malaria parasite {\em Plasmodium falciparum} is
controlled by coordinated changes in gene expression throughout its complex
life cycle, but the corresponding regulatory mechanisms are incompletely
understood. To study the relationship between genome architecture and gene
regulation in {\em Plasmodium}, we assayed the genome architecture of {\em P.
falciparum} at three time points during its erythrocytic (asexual) cycle.
Using chromosome conformation capture coupled with next-generation sequencing
technology (Hi-C), we obtained high-resolution chromosomal contact maps, which
we then used to construct a consensus three-dimensional genome structure for
each time point. We observed strong clustering of centromeres, telomeres,
ribosomal DNA and virulence genes, resulting in a complex architecture that
cannot be explained by a simple volume exclusion model. Internal virulence
gene clusters exhibit domain-like structures in contact maps, suggesting that
they play an important role in the genome architecture. Midway during the
erythrocytic cycle, at the highly transcriptionally active trophozoite stage,
the genome adopts a more open chromatin structure with increased chromosomal
intermingling. In addition, we observed reduced expression of genes located in
spatial proximity to the repressive subtelomeric center, and colocalization of
distinct groups of parasite-specific genes with coordinated expression
profiles. Overall, our results are indicative of a strong association between
the {\em P. falciparum} spatial genome organization and gene expression.
Understanding the molecular processes involved in genome conformation dynamics
could contribute to the discovery of novel antimalarial strategies.
\end{abstract}

\section{Introduction}

Malaria remains a major contributor to the global burden of disease, with an
estimated 219 million infected individuals and 660,000 deaths annually
\citep{who:malaria}. One of the main limiting factors for the development of
novel therapies is our poor understanding of mechanisms regulating the
parasite's complex life cycle, which involves several distinct parasitic
stages in the human and mosquito hosts. Regulation of these developmental
stages is thought to be controlled by coordinated changes in gene expression.
In addition, virulence associated with the human malaria parasite, {\em
Plasmodium falciparum}, is known to be directly linked to the parasite's
ability to tightly control the expression of genes involved in antigenic
variations on the surface of infected red blood cells. Some progress has been
made in elucidating mechanisms controlling the expression of these virulence
genes \citep{duraisingh:heterochromatin, freitas-junior:telomeric}.
Furthermore, a limited number of putative sequence-specific transcription
factors has been identified in the parasite genome \citep{balaji:discovery,
coulson:comparative}, including 27 ApiAP2 plant-like TFs, and drastic changes
in chromatin structure related to transcriptional activity have been observed
throughout the parasite erythrocytic cycle \citep{ponts:nucleosome}. However,
general and specific mechanisms controlling the expression of the ~6,372
parasite genes remain poorly understood.

In higher eukaryotes, several analyses have emphasized the role of genome
architecture in regulating transcription. Compartmentalization of the nucleus,
chromatin loops and long-range interactions contribute to a complex regulatory
network \citep{homouz:3d, kalhor:genome,  lieberman-aiden:comprehensive,
dixon:topological}. In {\em P. falciparum}, little is known about the effect
of genome organization on gene expression. Recent data indicate that genes
involved in control of parasite virulence ({\em var} genes) are associated
with repressive centers at the nuclear periphery
\citep{duraisingh:heterochromatin, dzikowski:mechanisms,
lopez-rubio:genome-wide} and that ribosomal DNA gene clusters are also
colocalized \citep{mancio-silva:clustering, lemieux:genome-wide}. However, a
global picture of the nuclear architecture throughout the parasite
erythrocytic cycle progression and its role in transcriptional regulation is
not yet available.

Chromosome conformation capture coupled with next generation sequencing (Hi-C)
measures the population average frequency of contacts between pairs of DNA
fragments in 3D space and can be used to model the spatial architecture of the
genome \citep{lieberman-aiden:comprehensive, duan:three-dimensional, kalhor:genome}. Here,
we performed a variant of the Hi-C protocol, tethered conformation capture
\citep{kalhor:genome}, to model at 10 kb resolution the spatial organization
of the {\em P. falciparum} genome throughout its erythrocytic cycle. Our
results indicate that the {\em P. falciparum} genome is highly structured,
with strong colocalization of centromeres, telomeres, active rDNA genes and
virulence gene clusters. These virulence genes exhibit distinctive contact
patterns and may therefore contribute to establishing the three-dimensional
structure of the {\em P. falciparum} genome. We identified discrete
chromosomal territories during the early and late stages of the parasite
erythrocytic cycle, which are partially lost in the highly transcriptionally
active trophozoite stage. Global chromosome movements during the erythrocytic
cycle are coherent with levels of transcriptional activity during the
different stages, and the three-dimensional genome architecture shows strong
correlation with gene expression levels. Collectively, our results suggest
that the {\em P. falciparum} genome organization and gene expression are
strongly interconnected.

\section{Results}

\subsection{Assaying genome architecture of {\em P. falciparum} at three stages using Hi-C}


\begin{figure}[h]
\centering
\includegraphics[width=\linewidth]{figures/fig1.pdf}
\caption{{\bf Tethered conformation capture of the {\em Plasmodium falciparum}
genome.}
\textbf{a,} Experimental protocol. \textbf{b,} Contact probability as a
function of genomic distance, with log-linear fits for the three erythrocytic
stages, as well as an experimental control. \textbf{c,} Normalized contact
count matrices at 10 kb resolution for chromosome 2 and chromosome 7 in the
schizont stage. \textbf{d,} Contact p-values (negative log10 scale) for
chromosome 2 and chromosome 7 in the schizont stage. In \textbf{(c)} and
\textbf{(d)}, yellow boxes denote clusters of VRSM genes, and blue dashed
lines indicate the centromere location.}
\label{fig:fig1}
\end{figure}



To study the genome architecture of {\em P. falciparum}, we harvested
parasites at three stages of the infected red blood cell cycle: after invasion
of red blood cells at the ring stage (0h), during high transcriptional
activity at the trophozoite stage (18h) and near the end of the cycle at the
schizont stage (36h), just before the newly formed parasites are released into
the bloodstream. Next, we applied the Hi-C protocol \citep{kalhor:genome} with
modifications to accommodate the extremely AT-rich genome of the malaria
parasite (Fig.~\ref{fig:fig1}a, Appendix Note 1, Appendix File 1).
As a control, we prepared a sample for which chromatin contacts were not
preserved by crosslinking of DNA and proteins.

We evaluated the quality of the resulting data for each sample. First, we
confirmed that the contact probability between two intrachromosomal loci
exhibits a log-linear decay with increasing genomic distance
(Fig.~\ref{fig:fig1}b, Appendix Fig.~\ref{suppfig:power-law}). Second,
we obtained lower numbers of interchromosomal contacts from crosslinked
samples relative to both random expectation and our control sample
(Appendix Table~\ref{table:ICP}). Third, we observed that the percentage
of long-range contacts (either interchromosomal or intrachromosomal $>$20 kb)
was significantly higher than control and comparable to the numbers observed
in yeast \citep{duan:three-dimensional} (Appendix Table~\ref{table:ICP}). Together,
these results indicated that we successfully assayed the {\em P. falciparum}
genome architecture with a high signal-to-noise ratio. We then coalesced the
mapped read pairs into a raw contact count matrix at 10 kb resolution, and we
corrected for potential technical and experimental biases
\citep{imakaev:iterative} (Fig.~\ref{fig:fig1}c, Appendix
Fig.~\ref{suppfig:ICE}) . The resulting normalized contact maps were used to
identify a subset of high-confidence contacts for each stage (Methods,
Appendix Note 2, Appendix File 2)~\citep{ay:statistical}. We
identified pairs of genes that show evidence of stage-specific contacts
(Methods) and then applied gene set enrichment analysis to the set of genes
that participate in such contacts.  This analysis identified significant
enrichment of VRSM genes for the ring and trophozoite stages (Appendix
Table~\ref{table:GSEAcompareStages}). This observation suggests that the
proximity between some VRSM clusters changes from the ring to trophozoite
stages, even though both stages show overall colocalization of VRSM clusters.
A similar enrichment analysis conducted using contacts that are specific to
two out of three stages resulted in no significant enrichment due to the small
number of genes involved in such contacts.

Normalized contact count and confidence score matrices exhibit a canonical
``X'' shape, indicative of a folded chromosome architecture anchored at the
centromere, as previously observed in yeast \citep{duan:three-dimensional,
tanizawa:mapping} and the bacterium {\em C. crescentus}
\citep{umbarger:three-dimensional} (Fig.~\ref{fig:fig1}c-d, Appendix
Fig.~\ref{suppfig:perChrFigs}). However, chromosomes that harbor
non-subtelomeric clusters of genes involved in antigenic variation and immune
evasion (Appendix File 3; VRSM genes: {\em var}, {\em rifin}, {\em
stevor} and {\em Pfmc-2tm})---chromosomes 4, 6, 7, 8 and 12---exhibit
additional folding structure (Fig.~\ref{fig:fig1}c-d, Appendix
Fig.~\ref{suppfig:perChrFigs}).

\subsection{Three-dimensional modeling recapitulates known organizational principles of {\em Plasmodium} genome}

\begin{figure}[h]
\centering
\includegraphics[width=\linewidth]{figures/fig2.pdf}
\caption{{\bf 3D modeling and validation with DNA FISH.}
 \textbf{a,} 3D structures of all three stages. The nuclear radii used to
 model ring, trophozoite and schizont stages were 350, 850, and 425 nm,
 respectively. Centromeres and telomeres are indicated with light blue and
 white spheres, respectively. Midpoints of VRSM gene clusters are shown with
 green spheres. \textbf{b,} Validation of colocalization between a pair of
 interchromosomal loci with VRSM genes (chr7: 550,000 - 560,000 that harbors
 internal VRSM genes and chr8: 40,000 - 50,000 that harbors subtelomeric VRSM
 genes) by DNA FISH (left) and by the three-dimensional model for the
 corresponding stage (right). The location of the loci in the 3D model is
 indicated with light blue spheres and pointed by black arrows. \textbf{c,}
 Validation same as in \textbf{(b)} for a pair of interchromosomal loci that
 harbor no VRSM genes (chr7: 810,000 - 820,000 and chr11: 820,000 - 830,000).
 }
 \label{fig:fig2}
 \end{figure}




To better characterize the genome architecture, we generated for each stage
100 consensus 3D structures, each of which summarizes the population average
(Fig.~\ref{fig:fig2}a, Methods), using multidimensional scaling (MDS) with two
primary constraints \citep{duan:three-dimensional}: (i) the DNA must lie within a sphere
with a specified diameter \citep{bannister:making, weiner:3d} and (ii)
adjacent 10kb loci must not be separated by more than 91 nm
\citep{bystricky:long-range}. \emph{P. falciparum} undergoes an atypical form
of cell division, resulting in schizont stage parasites with multiple
independent nuclei, each containing 1n chromosomes. Note that our model
assumes that a single copy of each chromosome is present in each structure,
thus averaging the signal from these multiple nuclei per cell.


We performed a series of experiments to assess the robustness of our 3D
inference procedure.  Our results showed only slight changes in the inferred
3D models when we varied the parameter used in conversion of contact counts to
expected distances (Appendix Table~\ref{table:stabilityToBeta}). This
was also true when we removed from the inference the two types of spatial
constraints related to nuclear volume and to distances between adjacent beads
(Appendix Table~\ref{table:stabilityToConstraints}). Finally, our
experiments on the impact of the initialization step (Methods) showed that
structures inferred from different initial configurations are highly similar
(Appendix Fig.~\ref{suppfig:compareStructurePairs}), do not fall into
discrete clusters (Appendix Fig.~\ref{suppfig:CHindices}) and all such
structures exhibit common organizational hallmarks (Appendix
Fig.~\ref{suppfig:clusteringIn100Structures}). Because of the stability of
our inference procedure, hereafter we generally present and discuss the
results for only one representative structure per stage.

Although the modeling procedure contains no explicit constraints on telomere
or centromere locations, we observe strong colocalization of both sets of loci
across all three stages (Fig.~\ref{fig:fig2}a, Appendix
Fig.~\ref{suppfig:3Dcentromeres}, Appendix Table~\ref{table:witten}),
with centromeres and telomeres localizing in distal regions of the nucleus. To
understand further the colocalization patterns of centromeres and telomeres in
each stage, we divided each chromosome into three compartments
(left--mid--right or telomeric--centromeric--telomeric) using eigenvalue
decomposition (Methods) and then performed hierarchical clustering on the
matrix of pairwise distances between compartments (Appendix
Fig.~\ref{suppfig:compDists}). At each stage, we observed clusters that are
comprised primarily of either centromeric or telomeric compartments. In
particular, during the trophozoite stage, all the centromeric compartments
fall into two main clusters suggesting strong colocalization of all
centromeres for this stage (Appendix Fig.~\ref{suppfig:compDists}d).
Such strong colocalization has previously been observed by immunofluoresence
microscopy at the trophozoite and schizont stages but not at the early ring
stage \citep{hoeijmakers:plasmodium}. However, when the size of the nucleus is
used as a marker of the parasite asexual cycle stage \citep{bannister:making,
weiner:3d}, the cells that are presented as trophozoites in this previous
study \citep{hoeijmakers:plasmodium} are more similar to our ring stage
parasites, indicating that centromere clustering also occurs early in the
erythrocytic cycle. Furthermore, if the centromeres are stochastically
distributed between a small number of foci within a population, then an assay
that measures average signal, such as Hi-C, will indeed demonstrate an
aggregate clustering for the centromeres and not complete dispersion as
suggested by a recent study \citep{lemieux:genome-wide}. These results suggest
that {\em P. falciparum} nuclei are highly structured around centromeres and
telomeres, consistent with known organizational principles gathered through
multiple independent microscopy experiments \citep{duraisingh:heterochromatin,
dzikowski:mechanisms, lopez-rubio:genome-wide, hoeijmakers:plasmodium}.


\subsection{Virulence gene clusters on different chromosomes colocalize in 3D}

In addition to centromeres and telomeres, we observed for all VRSM gene
clusters, both internal and subtelomeric, a significant colocalization with
one another (Fig.~\ref{fig:fig2}a, Appendix Table~\ref{table:witten}).
The significant colocalization for VRSM clusters as well as for centromeres
and telomeres were all reproducible when we used contact counts instead of 3D
distances to perform colocalization tests similar to Appendix
Table~\ref{table:witten} (data not shown). Given colocalization of the
telomeres, colocalization of subtelomeric clusters is not surprising. However,
the proximity of internal VRSM clusters with one another and with subtelomeric
clusters is unexpected under the random polymer looping model and, to the best
of our knowledge, observed experimentally for the first time. To further
validate these results inferred from our 3D models, we performed DNA
fluorescence {\em in situ} hybridization (FISH) (Methods, Appendix Note
3) on an interchromosomal pair of strongly interacting (at 10 kb resolution)
VRSM clusters: the internal cluster from chromosome 7 and a subtelomeric
cluster from chromosome 8. We observed strong colocalization by FISH ($>$90\%
of cells, Fig.~\ref{fig:fig2}b, Appendix Fig.~\ref{suppfig:fish}a,
Appendix Table~\ref{table:FISHprimers}), providing independent support
for the clustering of VRSM genes. Although previous FISH results indicated
that {\em var} genes form 2 to 5 clusters in 3D per cell
\citep{freitas-junior:frequent, lopez-rubio:genome-wide}, others recently
showed single foci for the VRSM gene-associated repressive histone mark
H3K9me3 and heterochromatin protein 1 (PfHP1) \citep{dahan:pfsec13}, as well
as for H3K36me3 that marks both active and silenced {\em var} genes
\citep{ukaegbu:recruitment}. Because our experimental strategy (Hi-C) captures
a population average, we are unable to distinguish between multiple VRSM gene
clusters in 3D if the genes are randomly distributed among clusters from cell
to cell. Using FISH experiments, we also observed strong colocalization
($>$90\% of cells, Fig.~\ref{fig:fig2}c, Appendix
Fig.~\ref{suppfig:fish}b) for a pair of interchromosomal loci located outside
VRSM clusters with consistent strong interactions at all three stages, while
colocalization was not observed for a pair of non-interacting interchromosomal
loci ($<$10\% of cells, Appendix Fig.~\ref{suppfig:fish}c). These
results demonstrate that our population average Hi-C data agrees with a
majority of single cell FISH images.

\subsection{Highly transcribed rDNA units colocalize in 3D during the ring stage}

 
 \begin{figure}[h]
 \centering
\includegraphics[width=\linewidth]{figures/fig3.pdf}

 \caption{{\bf Colocalization of highly transcribed rDNA units.} 
  Virtual 4C plots generated at 25 kb resolution using as a bait the A-type
  rDNA unit on chromosome 7 from crosslinked Hi-C libraries of \textbf{(a)}
  ring, \textbf{(b)} trophozoite, \textbf{(c)} schizont stages and
  \textbf{(d)} from the trophozoite control library. Vertical red line
  indicates the midpoint of the A-type rDNA unit on chromosome 5. Normalized
  contact counts from 50 kb up- and downstream of the 25 kb bin containing the
  rDNA unit are used, omitting the rDNA-containing window itself to exclude
  repetitive DNA. For each window $w$ on chromosome 5, the contact enrichment
  is calculated by dividing the contact count between the bait and $w$ to the
  average interchromosomal contact count for the bait locus.
  }
  \label{fig:fig3}
  \end{figure}



Similar to VRSM genes, the rDNA genes are strictly regulated during the
parasite life cycle. In {\em P. falciparum}, these genes are dispersed on
different chromosomes in five rDNA units containing the 18S, 5.8S and 28S
genes and one repeat unit consisting of three copies of the 5S gene. A
previous FISH study suggested that all rDNA units localize at a single
nucleolus but also claimed that the two units on chromosomes 5 and 7 that are
actively transcribed during the ring stage (A-type units) are dispersed in the
ring stage \citep{mancio-silva:clustering}. However, a more recent Hi-C study
of ring stage parasites demonstrated strong clustering of these two A-type
units in multiple strains \citep{lemieux:genome-wide}. Analysis of our Hi-C
data confirmed overall enrichment of contacts between chromosomes 5 and 7 in
all three stages and showed a particular peak of enrichment centered at the
rDNA unit on chromosome 5 among all interchromosomal contact partners of the
rDNA unit on chromosome 7 in the ring stage (3.32x, Fig.~\ref{fig:fig3}a). We
observed less striking enrichment of contacts that are not specific to or
centered on the rDNA units for the other two stages (trophozoites (1.99x),
schizonts (1.23x), Fig.~\ref{fig:fig3}b-c) during which the two rDNA units are
not transcribed \citep{mancio-silva:clustering}. Reanalysis of the Lemieux
{\em et al.} data using our processing pipeline also showed this enrichment
consistently in three different NF54-derived strains in the ring stage (6.06x,
4.47x and 4.61x, respectively, Appendix
Fig.~\ref{suppfig:rDNAunitsOn3D}a-c). Control libraries from both studies do
not exhibit this enrichment (Fig.~\ref{fig:fig3}d, Appendix
Fig.~\ref{suppfig:rDNAunitsOn3D}d). Our 3D models for the ring stage place
these two A-type rDNA units near the nuclear periphery. Together with the
strong colocalization between A-type rDNA, these results suggest the existence
of perinuclear transcriptionally active compartments. Such compartments may
play a role in separating out the single active var gene per cell from compact
chromatin around (sub)telomeric regions marked by the repressive H3K9me3
modification \citep{lopez-rubio:genome-wide}. We did not observe an overall
colocalization between all rDNA units in the ring stage, including the three
18S, 5.8S, 28S units and one 5S unit that are not expressed during asexual
erythrocytic cycle (Appendix Table~\ref{table:witten}).  This
observation suggests that genomic location may influence rDNA expression by
the preferential colocalization of the expressed rDNA units, away from the
non-expressed units.

\subsection{Transcriptionally active trophozoite stage exhibits an open chromatin structure}

Assaying three different time points, we observed significant changes in
chromatin structure throughout the erythrocytic cycle. To visualize high-level
changes, we generated animations showing the movement of chromosomes as the
parasite progresses through its cell cycle (Appendix Files 4-18). We then
characterized global chromatin changes by analyzing the relationship between
contact frequency and genomic distance (Fig.~\ref{fig:fig1}b, Appendix
Fig.~\ref{suppfig:power-law}). The gradient of the log-linear fit is very
close to -1 in both the ring and schizont stages (-0.98 and -0.96,
respectively) indicative of a fractal globule genome architecture that is
usually found in higher eukaryotes \citep{lieberman-aiden:comprehensive}.
Intriguingly, the intermediate and most active transcriptional stage yields a
log-linear fit value with gradient -1.14, a value between the fractal (-1) and
the equilibrium globule (-1.5) model suggested in yeast
\citep{fudenberg:higher-order} and indicative of more chromosomal
intermingling. Indeed, a value of -1.17 has been demonstrated to correspond to
a state of ``unentangled rings'' similar to the fractal globule state, in
which the rings may correspond to long chromosomal regions looped on or
anchored to a nuclear scaffold \citep{vettorel:statistics}. It is important to
note that the value of the gradient is determined solely by Hi-C contact
counts and,  therefore, the above mentioned difference is independent of our
3D modeling and the change in the nuclear radius from one stage to another.
Furthermore, the difference in the gradient value for trophozoites compared to
the two other stages is consistent for each chromosome, suggesting that all
chromosomes change their folding behavior during the trophozoite stage
(Appendix Table~\ref{table:scalingFactors}).

In order to further investigate whether trophozoites show a more open
chromatin structure than the two other stages, we systematically compared our
data across all three stages. First, we computed and compared intra and
interchromosomal contact probabilities for each stage (Appendix
Fig.~\ref{suppfig:intraVSinter}). We observed that intrachromosomal contacts,
even at very large distances, are more prevalent than interchromosomal
contacts for all three stages, suggesting the existence and preservation of
chromosome territories throughout the erythrocytic cycle. However, the
enrichment in intrachromosomal contacts was the lowest for trophozoite stage
for distances above 300~kb, suggesting a relative loss of territories in this
stage compared to the other two. Second, we quantified how preserved the
chromosomal territories are at each stage by estimating the degree of
chromosome intermingling in our 3D models. We randomly sampled small spheres
in the nucleus and asked, for each chromosome {\em i}, what percentage of the
spheres that contain any locus from chromosome {\em i} also contain a locus
from another chromosome {\em j}. Our results using different sphere sizes, and
controlling for the varying nuclear diameter, consistently exhibited the
highest amount of intermingling for the trophozoite stage and the highest
territory preservation for the schizont stage (Appendix
Fig.~\ref{suppfig:territory}).

To understand the architectural dynamics responsible for the systematic
changes in chromatin compaction, we computed the relative movements among
chromosome compartments during the erythrocytic cycle. Despite the increase in
nuclear volume, many interchromosomal compartment pairs came closer together
in the transition from the ring to trophozoite stage (Appendix
Fig.~\ref{suppfig:compMovement}a, red color). Subsequently, most
interchromosomal compartments moved away from each other in the transition to
the schizont stage (Appendix Fig.~\ref{suppfig:compMovement}b, blue
color), resulting in more compact chromatin that favors formation of
chromosome territories. These results are consistent with a previously
proposed model, in which the {\em P. falciparum} nucleus exhibits a more open
chromatin configuration at the trophozoite stage, enabling interchromosomal
contacts and high levels of transcriptional activity \citep{ponts:nucleosome}.

\subsection{{\em Plasmodium} genome architecture cannot be explained by volume exclusion}


  \begin{figure}[h]
  \centering
\includegraphics[width=\linewidth]{figures/fig4.pdf}
  \caption{{\bf Volume exclusion modeling. }
  Observed/expected contact frequency matrices illustrate, for each locus,
  either the depletion (blue) or enrichment (red) of interaction frequencies
  compared to what would be expected given their genomic distances.
  \textbf{a,} Observed/expected contact frequency matrices derived from {\em
  S. cerevisiae} chr 7 from volume exclusion modeling (left) and Hi-C data
  (right). \textbf{b,} Observed/expected matrices from volume exclusion
  modeling (left) and Hi-C data (right) for {\em P. falciparum} chr 7 during
  the trophozoite stage.}
  \label{fig:fig4}
  \end{figure}



We next assessed whether the primary architectural features in {\em P.
falciparum} arise from a population of constrained but otherwise random
configurations of chromatin following a simple volume exclusion (VE) model, as
recently shown for {\em Saccharomyces cerevisiae} \citep{tjong:physical}. We
therefore repeated the Tjong {\em et al.} simulations using the same set of
constraints and successfully recovered the strong correlation between the
simulated map and the experimentally observed yeast contact map (raw
correlation of 0.91; normalized correlation of 0.57; Fig.~\ref{fig:fig4}a,
Methods, Appendix Note 4, Appendix
Fig.~\ref{suppfig:VEconvergence}). In contrast, our simulations for the ring,
trophozoite and schizont stages of {\em P. falciparum} yielded markedly lower
correlations (normalized correlation of 0.34, 0.39 and 0.49, respectively) and
strikingly different contact maps compared to the experimentally observed maps
(Fig.~\ref{fig:fig4}b). One significant reason for the observed discrepancy
between yeast and {\em P. falciparum} is the lack of structure around clusters
of VSRM genes in the simulated data (Fig.~\ref{fig:fig4}b). Accordingly, we
conclude that the simple volume exclusion model, which so convincingly
explains the yeast genome architecture, is insufficient to explain the
observed architecture of {\em P. falciparum} genome, highlighting the need for
a genome-wide assay such as Hi-C to obtain accurate structural models.

\subsection{VRSM gene clusters form domain-like structures}

  \begin{figure}[h]
\includegraphics[width=\linewidth]{figures/fig5.pdf}
  \centering
  \caption{{\bf Role of internal VRSM gene clusters in shaping genome
  architecture.}
   \textbf{a-d,} Heatmaps of scaled pairwise Euclidean distances derived from
   the 3D model at 10 kb resolution for \textbf{(a, b)} two chromosomes that
   harbor internal VRSM gene clusters and \textbf{(c, d)} two chromosomes that
   do not. Yellow boxes indicate locations of VRSM clusters. }
   \label{fig:fig5}
   \end{figure}



Our results from the volume exclusion modeling and from visual inspection of
the contact maps suggest that the internal VRSM gene clusters are associated
with distinctive structural features. All eight of the internal VRSM clusters
induce a striking cross-like shape, both in the contact count and 3D distance
matrices (Fig.~\ref{fig:fig5}a-b, Appendix
Fig.~\ref{suppfig:perChrFigs}). Quantification of this phenomenon revealed a
consistent contact pattern across all eight internal VRSM clusters
(Appendix Fig.~\ref{suppfig:TADs}), suggesting that VRSM gene clusters
adopt a compact, domain-like structure. Although these domain-like structures
resemble topologically associated domains (TADs) described in mammals
\citep{dixon:topological, nora:spatial}, the VSRM domains are much smaller
(10--50 kb) compared to TADs (0.1--1 Mb). Furthermore, because VRSM genes have
no orthologs in human and mouse, mechanisms regulating these domain-like
structures likely differ from the one in mammalian genomes. Further
understanding of how these VRSM domains are formed in  \emph{Plasmodium} would
shed light on genome architecture associated regulation of VRSM gene
expression.

Another interesting pattern involving internal VRSM clusters emerged from
further inspection of chromosome compartments.  Five of the eight internal
VRSM clusters (two on chromosome 4, one on chromosome 7 and both clusters on
chromosome 12) occur at compartment boundaries (third and fourth rows of
Appendix Fig.~\ref{suppfig:perChrFigs}). This striking overlap  suggests
that VRSM genes may contribute to or rely upon the boundaries of chromosomal
compartments. Taken together with the domain-like structures around these VRSM
clusters, these results confirm that genome architecture is likely to be
involved in the strict regulation of virulence genes during the erythrocytic
cycle.

\subsection{Expression is highly concordant with 3D localization for {\em Plasmodium} genes}

\begin{figure}[h]
\includegraphics[width=\linewidth]{figures/fig6.pdf}
\centering
\caption{{\bf Relationship between 3D architecture and gene expression.}
\textbf{a,}Correlation between expression profiles of pairs of
interchromosomal genes as a function of number of contacts linking the two
genes. To generate this plot all interchromosomal gene pairs are first sorted
in increasing order of their expression correlation and then binned into 20
equal width quantiles ($5$th, $10$th, ..., $100$th). For each bin, the average
expression correlation between gene pairs (x-axis) and the average normalized
contact count linking the genes in each pair together with its standard error
(y-axis) are computed and plotted. Interchromosomal gene pairs that have
contact counts within the top 20\% for each stage have more highly correlated
expression profiles than the remaining gene pairs [Wilcoxon rank-sum test,
p-values 2.48e-206 (ring), 0 (trophozoite), and 0 (schizont)].
\textbf{b,} Correlation between expression profiles of pairs of
interchromosomal genes as a function of 3D distance between the genes. This
plot is generated similar to \textbf{a} but with using 3D distances instead of
contact counts (y-axis). In order to summarize results from multiple 3D
structures per each stage, we plot the median value among 100 structures with
a red line and shaded the region corresponding to the interval between 5th and
95th percentile with gray. Interchromosomal gene pairs closer than 20\% of the
nuclear diameter have more highly correlated expression profiles than genes
that are far apart [Wilcoxon rank-sum test, p-values 7.17e-221 (ring), 0
(trophozoite), and 1.57e-88 (schizont)].
\textbf{c,} Gene expression as a function of distance to telomeres. To
generate this plot all genes are first sorted by increasing distance to the
centroid of telomeres (x-axis) and then binned similar to \textbf{a} into 20
equal width quantiles. The average log expression
value~\citep{bunnik:polysome}
together with its standard error (y-axis) is plotted for genes in each bin. In
order to summarize results from multiple 3D structures per each stage, we plot
the median value among 100 structures with a red line and shaded the region
corresponding to the interval between 5th and 95th percentile with gray. Genes
that lie within 20\% of the nuclear diameter to the centroid of the telomeres
showed significantly lower expression levels [Wilcoxon rank-sum test, p-values
1.54e-12 (ring), 1.69e-32 (trophozoite), 3.37e-20 (schizont)].
\textbf{d,} First kCCA expression profile component score, corresponding to
the projection of the gene expression profile onto the extracted kCCA profile
for the trophozoite stage.}
\label{fig:fig6}
\end{figure}




Next, we investigated the relationship between the three-dimensional genome
structure and gene expression using four published expression data sets
\citep{leroch:discovery, lopez-barragan:directional, otto:new,
bunnik:polysome}. First, we observed that, for each of the three stages,
interchromosomal pairs of genes that strongly interact (contact counts within
the top 20\%) as well as gene pairs that are in close proximity ($<$20\% of
the nuclear diameter) showed more correlated expression profiles than genes
that are far apart (Fig.~\ref{fig:fig6}a,b), as previously observed in yeast
\citep{homouz:3d}. To assess whether these observed trends are confounded by
similarly expressed VRSM genes that strongly interact with each other and are
placed together near telomeres by our 3D model, we repeated the above analyses
by excluding all VRSM genes (Appendix
Fig.~\ref{suppfig:expVSdistWithoutVRSM}). Even though the observed trends are
weakened by exclusion of VRSM genes, the decrease in 3D distance and increase
in contact count with increasing expression correlation remained significant
(Appendix Fig.~\ref{suppfig:expVSdistWithoutVRSM}). It is also important
to note that, for these analyses, we excluded intrachromosomal gene pairs to
only focus on the relationship between 3D proximity and gene expression by
eliminating the confounding effect caused by genes that lie nearby on a
chromosome and show similar expression profiles. Second, we analyzed gene
expression in relation to the repressive subtelomeric clusters
\citep{duraisingh:heterochromatin, dzikowski:mechanisms,
lopez-rubio:genome-wide} and other nuclear landmarks. The subset of genes that
lie within 20\% of the nuclear diameter to the centroid of the telomeres
showed significantly lower expression levels than more distal genes
(Fig.~\ref{fig:fig6}c). The repressive effect of the subtelomeric clusters is
apparent in all three stages and is strongest at the trophozoite stage, in
which subtelomeric VRSM clusters are known to be tightly repressed
\citep{chen:developmental}. If we remove the VRSM genes from the analysis, the
repressive effect is still significant at the trophozoite stage, which is
known to be the most active transcriptional stage of the erythrocytic cycle
(Appendix Fig.~\ref{suppfig:distToCenter}a,b). Similar analysis showed
higher expression levels for genes located near the nuclear center, as well as
for genes close to the centroid of the centromeres (Appendix
Fig.~\ref{suppfig:distToCenter}c,d). Furthermore, we observed significant and
consistent colocalization across all three stages for 11 of the 15 expression
clusters identified in \cite{leroch:discovery} (Appendix
Table~\ref{table:witten}). Strikingly, the trophozoite stage showed
significant colocalization for clusters associated with genes that are
repressed during this stage (clusters 1, 3, 4, and 13-15) as well as genes
that exhibit high levels of expression (clusters 6, 9, 10, and 12), confirming
the strong relationship between 3D location and gene expression.

To further explore the relationship between gene expression and 3D structure,
we employed an unsupervised learning method known as {\em kernel canonical
correlation analysis} (kCCA) \citep{bach:kernel}. This methodology identifies
a set of orthogonal gene expression profiles that exhibit coherence with
respect to the 3D structure (Methods). For all stages, the projection of gene
expression patterns onto the first extracted profile exhibits a striking
transcriptional gradient across the 3D structure, from the telomere cluster to
the opposite side of the nucleus  (Fig.~\ref{fig:fig6}c, Appendix
Fig.~\ref{suppfig:kCCAsecond}a,c,e). The coherence with 3D structure drops
significantly in the second component of the kCCA (Appendix
Fig.~\ref{suppfig:kCCAsecond}b,d,f), suggesting that gene expression is
strongly influenced by distance to the subtelomeric repressive center. To
further interpret the kCCA results we employed gene set enrichment analysis
\citep{subramanian:gene} on the ranked lists of projections onto the first
kCCA component. The results showed, for all three stages, significant
enrichment (q-value $<$ 0.01) of gene sets related to antigenic variation and
translation (i.e. ribosome proteins) on the telomeric and non-telomeric side,
respectively, of the extracted kCCA expression profile (Appendix
Tables~\ref{table:RingsFirstPro},~\ref{table:TrophsFirstPro},~\ref{table:SchizontsFirstPro}).
Similar to the colocalization test results for expression clusters of
\cite{leroch:discovery}, clusters of genes that are repressed (clusters 4, 13,
and 14) and expressed (clusters 6 and 9-12) in the trophozoite stage showed
consistent enrichment in the strongest kCCA profile (Appendix
Table~\ref{table:kCCAforClusters}). In addition, genes exclusively expressed
in sporozoites (cluster 1) and gametocytes (clusters 3) were also strongly
enriched, indicating that the repression of these genes during the asexual
erythrocytic cell cycle may be related to their localization within the
nucleus. Finally, for GO terms related to parasite invasion (rhoptry, myosin
complex, motor activity; q-value $<$ 0.1) and for the cluster of invasion
genes (cluster 15), we observed an enrichment relative to the second kCCA
component, suggesting that expression of invasion genes may also be regulated
by the 3D genome structure (Appendix
Tables~\ref{table:kCCAforClusters},~\ref{table:secondPro}).


\section{Discussion}

This study presents the first analysis of genome architecture during the cell
cycle of a eukaryotic pathogen. Overall, our data demonstrate that the genome
of {\em P. falciparum} exhibits a higher degree of organization than the
similarly sized budding yeast genome. Although localization of chromosomes
within the {\em P. falciparum} nucleus is partially dictated by size
constraints, the simple volume exclusion model observed in yeast is
insufficient to explain the 3D architecture of the {\em P. falciparum} genome.
In particular, a striking spatial complexity is added by clusters of virulence
genes, which function as critical structural elements that shape the genome
architecture. Furthermore, our model correlates well with expression levels of
parasite-specific gene sets and shows strong clustering of repressed genes and
highly transcribed rDNA units, indicative of a non-random genomic organization
that contributes to gene regulation during the asexual erythrocytic cycle.
Considering the strong association between nuclear architecture and gene
expression as well as the observed domain-like structures, {\em Plasmodium}
species may be excellent model organisms to study the impact of genome
structure on gene regulation. The lower complexity of genome organization in
organisms with similarly sized genomes, such as yeast, may indeed be less
informative for such investigations.

Assaying multiple time points during the parasite's erythrocytic cycle
revealed intriguing changes in genome structure between the different
developmental stages. Our results show that the genome adopts a more open
conformation during the trophozoite stage consistent with high transcriptional
activity in this stage of the erythrocytic cycle, followed by compaction of
chromosomes into discrete chromosome territories before re-invasion of a new
host cell. A similar pattern was observed previously for nucleosome occupancy,
with strong histone depletion at the trophozoite stage and nucleosome
replacement at the schizont stage \citep{ponts:nucleosome}. Based on these
observations, we hypothesize that the spatial genome organization of {\em P.
falciparum}, coupled with its dynamic chromatin structure, acts as an
important alternative mechanism of transcriptional regulation, possibly
compensating for the lack of a diverse collection of specific transcription
factors \citep{balaji:discovery, coulson:comparative} and the low capacity of
the parasite to regulate gene expression in response to metabolic stress
\citep{ganesan:genetically, leroch:systematic}. These changes in genome
architecture could mainly be indicative of differences between the various
developmental stages of the parasite, but could also be related to cell cycle
progression itself. Given the importance of nuclear architecture for
regulation of gene expression, disruption of its genome organization is likely
to interfere with parasite development through the erythrocytic cycle and
could therefore be lethal to the parasite. Compounds targeting proteins
involved in establishing and maintaining the three-dimensional genome
structure in {\em P. falciparum} may thus have potent antimalarial activity.

A recently published Hi-C study suggested that chromosomal territories are
absent in the ring stage parasites, especially for larger chromosomes
\citep{lemieux:genome-wide}. In contrast, our data provides multiple lines of
evidence for the existence of chromosome territories throughout the
erythrocytic cell cycle. In particular, we observed that intrachromosomal
contacts, even at very large distances, are more prevalent than
interchromosomal contacts. This observation is supported by our own Hi-C data
in three stages as well as by our reanalysis of the Lemieux {\em et al.} data
(Appendix Fig.~\ref{suppfig:intraVSinter}b-e). The difference between
the two analyses can be traced to our improved method for discretizing the
genomic distance axis, which avoids bins with few observations and, hence,
high variance (Appendix Fig.~\ref{suppfig:intraVSinter}a versus b). Even
though further experiments may be necessary to reconcile these differences,
our results strongly suggest that {\em P. falciparum} chromosomes occupy
distinct territories, similar to other eukaryotic genomes.

Clustering of virulence gene families into a distinct nuclear compartment is
likely to play an important role in the formation of repressive
heterochromatin that controls the silencing of these genes. Heterochromatin
around virulence genes is characterized by histone modifications H3K36me3
\citep{jiang:pfsetvs} and H3K9me3 \citep{duraisingh:heterochromatin,
lopez-rubio:genome-wide}, both of which were shown to be essential for
maintaining {\em var} gene repression. The formation of heterochromatin is
directed by the interaction of PfSIP2 with specific DNA motifs in promoters of
virulence genes and in subtelomeric domains \citep{flueck:major}, but
additional factors are likely to contribute to this process. The question
remains, however, how the formation of this repressive center is regulated and
whether the colocalization of virulence gene clusters is a cause or a
consequence of their transcriptional silencing. One experiment that would shed
light on this issue would be to relocate a {\em var} gene to a different
location in the genome and to monitor how the introduction of this novel {\em
var} gene locus influences genome structure, although technical challenges
that come with manipulation of the {\em P. falciparum} genome may prevent such
procedures. Virulence genes are expressed on the surface of red blood cells
and are therefore important antigens for the humoral immune system. A better
understanding of virulence gene silencing will provide us with more
opportunities to interfere with this process, which would ultimately benefit
vaccine development.

In this study, we modeled the {\em P. falciparum} genome architecture based on
the average signal from a population of parasites. However, it can be expected
that considerable variability in genome conformation exists from cell to cell,
as recently demonstrated in mouse \citep{nagano:single-cell}. While
challenging, it would be interesting to perform Hi-C analysis on individual
parasites to reveal the extent of inter-cellular variation in {\em P.
falciparum} genome architecture. This experiment would also allow a more
detailed analysis of the clustering of {\em var} genes in one or multiple
repressive centers, as well as the differential localization of the single
active {\em var} gene.

In conclusion, this study demonstrates the unique role of genome organization
in transcriptional regulation in the human malaria parasite. In other
eukaryotes such as human and mouse, genome organization has been shown to
participate in gene regulation through formation of specific chromatin loops
that bring enhancers and enhancer-like elements in proximity to their target
promoters. However, a global reorganization of the entire genome correlated
with changes in transcriptional capacity, as described here for {\em P.
falciparum}, has not been observed for any of the genomes studied so far.
Therefore, our data proposes a novel mechanism of gene regulation for {\em P.
falciparum} that can operate without relying on specific transcription factors
or enhancer elements. Similar to other eukaryotes, gene expression in {\em P.
falciparum} is likely to be regulated by multiple layers of control at both
transcriptional and translational levels. However, the necessity to
transcriptionally repress distinct groups of parasite-specific genes may have
driven {\em P. falciparum} to adopt this exceptional genome organization.


\section{Methods}
\subsection{Experimental protocols}

\subsubsection{{\em P.\ falciparum} strain and culture conditions}
{\em P.\ falciparum} strain 3D7 was maintained in human O+ erythrocytes in
5\% haematocrit according to a previously described protocol \citep{trager:human}.
Cultures were synchronized twice at ring stage with 5\% D-sorbitol treatments
performed eight hours apart \citep{lambros:synchronization}. Parasites were
harvested 48 hours after the first sorbitol treatment (0h; ring stage), and
then 18 hours (early trophozoite stage) and 36 hours (late schizont stage)
thereafter. The developmental stage of the parasites was verified by microscopy
using Giemsa-stained blood smears prior to harvesting.

\subsubsection{Cross-linking}
Aspirated {\em P.\ falciparum} cultures were pooled into 50 ml centrifuge
tubes and filled up to 35 ml with phosphate buffered saline (PBS) warmed
to 37$^\circ$C. Cultures were treated with 3 ml 16\% formaldehyde
(1.25\% final concentration) and incubated for 25 min at 37$^\circ$C while rocking.
Formaldehyde was quenched with 5.2 ml 1.25 M glycine (final concentration 150 mM)
for 15 min at 37$^\circ$C while rocking, followed by 15 min at 4$^\circ$C while
rocking. PBS was used instead of formaldehyde and glycine for the not cross-linked
control. Cultures were spun at 660 $\times$ g for 20 min at 4$^\circ$C. Not
cross-linked control parasites were treated with 5 volumes 0.15\% saponin in water
and incubated 10 min at 4$^\circ$C while rocking. PBS was used instead of saponin
for the cross-linked parasites. Parasites were spun at 660 $\times$ g for 15 min
at 4$^\circ$C. Pellets were washed multiple times until clean and stored at -80$^\circ$C.

\subsubsection{Tethered conformation capture procedure}
We applied an adapted Hi-C method referred to as tethered conformation capture
(TCC) \citep{kalhor:genome} to map the intra and interchromosomal contacts in
{\em Plasmodium falciparum}. For a detailed description of the overall protocol
see Appendix Note 1.

\subsubsection{DNA-FISH}
For each 10 kb locus of interest, we determined the location for which on average
the highest number of contact counts were observed and designed DNA probes targeting
the 2 kb region surrounding this location. Probes were prepared using Fluorescein-High
Prime and Biotin-High Prime kits (Roche) according to manufacturer's instructions.
Template DNA was prepared by PCR (5 min at 95$^\circ$C, 35 cycles of 30 sec at
98$^\circ$C followed by 150 sec at 62$^\circ$C, and 5 min at 62$^\circ$C) using
the KAPA HiFi DNA Polymerase HotStart ReadyMix. Sequences of primers used for probe
generation are shown in Appendix Table~\ref{table:FISHprimers}. For a detailed
description of the DNA-FISH protocol see Appendix Note 3. The percentage of
colocalization was determined by visual inspection of $>$100 cells per condition.

\subsection{Computational methods}

\subsubsection{Mapping and filtering of sequence data}
\label{met:mapping}
We first trimmed each end of the paired-end reads from all samples to 40~bp.
%as our read lengths varied between 47 to 101~bp.
We used FastQC~\citep{andrews:fastqc}
% (\url{http://www.bioinformatics.babraham.ac.uk/projects/fastqc})
reports of aggregate read qualities for each sample to determine the amount
of trimming required from each end of the read to keep the highest quality
$40$-bp region.
%For short reads we only trimmed the reads from the $5'$ end, whereas for long reads we trimmed from both the $5'$ and the $3'$ ends.
% read lengths = 51 51 51 51 51 51 47 51 47 51 47 51 97 101 47 51 97 101
% trim 5' = 8 8 8 8 8 8 7 8 7 8 7 8 27 27 7 8 27 27
% trim 3' = 3 3 3 3 3 3 0 3 0 3 0 3 30 34 0 3 30 34

To filter out reads from human DNA, we mapped the trimmed paired-end reads to
the human genome (UCSC hg19) using the short read alignment mode of BWA (v0.5.9)~\citep{li:fast} 
with default parameter settings. Each end of the paired reads was mapped
individually. We post-processed the alignment results to extract reads that
mapped with an edit distance of at most 3. We then eliminated all pairs for
which at least one of the ends mapped to the human genome without any filtering
on the mapping quality or uniqueness. This loose mapping criteria is used to
assure that any read pair that is likely to come from human blood contamination
in the parasite samples is filtered out from our further analysis of
{\em Plasmodium} genome architecture.

We mapped the remaining paired-end reads to the \emph{Plasmodium falciparum 3D7}
reference genome (PlasmoDB v9.0). We post-processed the alignment results further
to extract the reads that mapped (i) uniquely to one location in the reference
genome, (ii) with an alignment quality score of at least 30 (which corresponds to
a 1 in 1000 chance that the mapping is incorrect), and (iii) with an edit distance
of at most 2. We extracted the paired-end reads with both ends mapping to the
{\em Plasmodium} genome. We then identified potential PCR duplicates, i.e., pairs
of read-pairs with identical genomic coordinates, and retained only one copy of
each. We also filtered out reads that map to intrachromosomal loci that are
$\le$1~kb apart. We refer to the remaining reads as \emph{informative reads}. We
computed chromosomal contact maps using only these informative reads. Appendix
File 1 summarizes the results of applying this pipeline to our sequencing libraries.


\subsubsection{Calculating noise level and percentage of long range contacts}
\label{met:ICP}
We calculated two measures that provide estimates of the noise level and efficiency
of the assay. The first is the interchromosomal contact probability (ICP)
index \citep{kalhor:genome}:
\[
ICP=\frac{\sum{\text{interchr contact counts}}} {\sum{\text{intrachr contact counts ($>$1 kb)}}}
\]
In the denominator, the intrachromosomal contact counts exclude contacts between
pairs of loci $\le$1~kb apart. Smaller ICP values indicate a better signal-to-noise
ratio, assuming that the real data (signal) will be enriched for intrachromosomal
contacts, whereas noise will be dominated by interchromosomal contacts. The second
number is the percent of long-range contacts (PLRC) extracted from the initial set
of paired-end reads that remain after filtering the reads that mapped to human genome:
\[
PLRC=\frac{\sum{\text{interchr contact counts}} + \sum{\text{intrachr contact counts ($>$20 kb)}}} {\text{Number of raw reads after human DNA filtering}}
\]
The bigger this percentage is, the more information the dataset provides about
non-adjacent chromatin contacts for the amount of sequencing in hand.


\subsubsection{Aggregating data relative to 10~kb windows}
\label{met:10kb}
Digesting the DNA with a frequently cutting restriction enzyme yields a very large
number of possible pairs of restriction fragments (i.e., locus pairs). In our case,
digesting the {\em Plasmodium} genome with MboI, which cuts at the 4~bp recognition
site ``GATC'', yielded 28,784 fragments (mean length 810~bp) corresponding to
33,114,193 intrachromosomal and 336,629,028 interchromosomal locus pairs. For 3D
modeling, we partitioned the {\em Plasmodium} genome into a collection of
non-overlapping 10~kb windows, and we assigned each restriction fragment to the
10~kb window that covers the majority of the bases in the fragment. This operation
reduced the number of possible fragments from 28,784 to 2,337 and the number of
possible locus pairs from $3.7 \times 10^8$ to 2,715,615 (228,539 intrachromosomal
and 2,487,076 interchromosomal).

\subsubsection{Normalizing raw contact maps}
\label{met:normalization}
For each possible pair of 10~kb loci, we refer to the total number of informative
read pairs that link the two loci as the {\em contact count}, and we refer to the
two-dimensional matrix containing these contact counts as the {\em raw contact map}.
We normalized the raw contact maps in two steps.  First, we ranked loci by their
percentage of intrachromosomal contacts with zero counts, and we filtered out the
top 2\% of this list. This removes all loci for which the signal to noise ratio
is too low (typically, regions of low mappability). Second, we applied an iterative
correction and eigenvector decomposition (ICE) method~\citep{imakaev:iterative}
that attempts to eliminate systematic biases in Hi-C data. The method estimates a
bias vector with one entry per locus. The tensor product of the bias vector with
itself generates a bias matrix $B$ that can be used to convert the raw contact map
into a normalized contact map.

\subsubsection{Estimating power-law fits to intrachromosomal contact probabilities}
\label{met:power-law}
It has been observed in the literature that for a pair of intrachromosomal loci,
the relationship between genomic distance and the expected contact count can be
estimated by a log-linear model \citep{lieberman-aiden:comprehensive, fudenberg:higher-order}.
This log-linear model is captured by a power-law fit of the form $P(s) \sim s^\alpha$
where $s$ denotes the genomic distance, $P(s)$ denotes the expected contact
probability at distance $s$ and $\alpha$ is the gradient of the log-linear fit.
For each stage, we first calculated $P(s)$ by segregating all intrachromosomal locus
pairs into $b=50$ equal-occupancy bins. This procedure involves enumerating all
possible intrachromosomal locus pairs (including pairs that have a contact count of zero),
sorting the pairs in increasing order according to their genomic distances, and then
segregating the resulting list into $b$ quantiles. For each bin $i$, we computed the
average number of contact counts per locus pair $\hat{c}_i$, and the average contact
distance $\hat{s}_i$ over all locus pairs in the bin. Then, for each bin $i$,
$P(\hat{s}_i)= \frac{\hat{c}_i}{N}$ where $N$ is the sum of all observed intrachromosomal
contact counts. We then found the best linear fit to $\log P(s)$ versus $\log s$ in
a given genomic distance range. Note that the control library  ``TROPH.-cont.''
was not subjected to normalization.


\subsubsection{Assigning statistical significance to normalized contact maps}
\label{met:fithic}
To obtain a set of high confidence contacts for each stage, we subjected the contact
maps at 10 kb resolution to a statistical confidence estimation procedure
\citep{ay:statistical}. We first accounted for the effect of genomic distance on the
intrachromosomal contact probability by fitting a smoothing spline to capture this
effect. We then accounted for biases using the normalization procedure described
above. Finally, we calculated p-values for intra and interchromosomal contacts and
corrected them jointly for multiple hypothesis testing to compute q-values, which
are used to filter contacts at a desired false discovery rate. For a detailed
description of the statistical significance estimation procedure see Appendix Note 2.

\subsubsection{Identifying stage-specific contacts}
\label{met:spageSpecific}
We determined the contacts that are specific to only one stage or to two out of three stages
as follows. First, we sorted the lists of contacts at 10~kb resolution according to increasing
p-values computed as described above for each stage. Then, we extracted contacts that are
ranked in top 1,000 in each stage and checked to see whether they appear among top 10,000
contacts for the other two stages. We labeled these contacts as stage-specific because they
are among the strongest contacts for one stage but not among moderately-strong contacts for
the other two stages. Similarly, we labeled contacts that are in top 1,000 in two out of
three stages but not in top 10,000 for the third stage. To perform gene set enrichment
analysis (GSEA), we extracted the lists of genes that are involved in stage-specific contacts
(only ring, only trophozoite or only schizont) as well as contacts common to two stages
(common to ring and trophozoite, common to ring and schizont or common to trophozoite and schizont).


\subsubsection{Inferring the 3D structures}
\label{met:inferring}
Our method for inferring the 3D structures is based on the method of \cite{duan:three-dimensional}.
Each chromosome is modeled as a series of beads on a string, spaced approximately
10~kb apart. We associated with each pair of beads $x_i$ and $x_j$ a physical
{\em wish distance} $\delta_{ij}$---i.e., the distance that we aim to capture with
our 3D model---derived from the bead pair's contact count $c_{ij}$. We then
placed all the beads in 3D space such that the distance $d_{ij}$ between the beads
$i$ and $j$ is as close as possible to the wish distance $\delta_{ij}$.

\paragraph{Wish distances: }

To obtain the wish distances, we note that two proximal intrachromosomal loci
are likely to come into contact due to random looping of the DNA, and that
this ``polymer packing'' contact likelihood can be expressed as a function of
the genomic distance $s$ between the loci. We then assumed that two loci with
observed contact count $c_{ij}$ will have the same physical distance
$\delta^{ij}$ as two intrachromosomal loci with expected contact count
$c_{ij}$ by polymer packing. The relationship between the expected contact
frequencies and the genomic distances $s$ suggests that \textit{P.\
falciparum}'s DNA behaves like a fractal globule polymer
\citep{lieberman-aiden:comprehensive} (Appendix Fig.
\ref{suppfig:power-law}). Any crumpled polymer exhibits a well-defined
relationship between its genomic length $s$ and the physical distance $d$
\citep{grosberg:role}:
\begin{equation}
d \sim s^{1 / 3}
\label{eq:fractal_globule}
\end{equation}
Therefore, using the relationship between genomic distances $s$ and
contact frequencies $c$, obtained by the fitting of the linear
model, and the relationship between physical distances $d$ and genomic
distances $s$ (Equation~\ref{eq:fractal_globule}), we inferred a
mapping between contact frequencies $c$ and physical distances $d$
up to a factor. We arbitrarily set the distance of the two beads with
the smallest non-zero contact count $c_\text{min}$ to be at a
certain percentage $\beta$ of the nucleus diameter. Note that
$c_\text{min}$ is not necessarily equal 1 since the contact counts are
normalized. The $\beta$ parameter hence sets the scaling of the physical
distances. We then obtain:
\begin{equation}
\delta_{ij} = \frac{\beta 2 r}{c_\text{min}^{\alpha / 3}} c_{ij}^{\alpha / 3}
\end{equation}
where $r$ is the nucleus radius, and $\alpha$ the coefficient obtained
in the linear model fitting (range: 30--500~kb, $\alpha=-0.963$ for rings,
$\alpha = -1.124$ for trophozoites, $\alpha = -1.013$ for schizonts).
We set all distances larger than the nucleus diameter to this value.

\paragraph{Optimization: } Given the resulting physical wish distances, we
defined the following optimization problem to find a structure $\mathbf{X}
\in R^{3 \times n}$, where $n$ is the number of beads:
\begin{equation*}
\begin{array}{ccll}
\underset{\mathbf{X}}{\text{minimize}} & &
\underset{\delta_{ij} \in \mathcal{D}}{\sum} \frac{1}{\delta_{ij}^2}\big(d_{ij} - \delta_{ij}\big)^2 &\\
\text{subject to}
& & x_i^Tx_i \leq r_{\rm max}^2, \quad
& i = 1:n\\
& & d_{i, i+1} \leq b^{\rm max} , \quad
& i = \{1:n \;|\; {\rm chr}_i = {\rm chr}_{i+1}\}\\
\end{array}
\end{equation*}
where $d_{ij}$ is the Euclidean distance between beads $x_i$ and
$x_j$, $\mathcal{D} = \{ \delta_{ij} | \delta_{ij} \neq 0\}$ is
the set of non-zero wish distances, and $b^{\rm max}$ is defined below.

The constraints are as follows:
\begin{enumerate}
\item \emph{All loci must lie within a spherical nucleus centered on the origin. }
  Electron microscopy experiments show that the nucleus roughly resembles a
  sphere, with the radius depending on the stage of the organism.  In
  this work, we use a nuclear radius of $r = 350$~nm for the ring
  stage, $r=850$~nm for the trophozoite stage and $r=425$~nm for the
  schizont stage \citep{bannister:making, weiner:3d}.
\item \emph{Two adjacent loci must not to be too far apart.}
  1000~bp of chromatin occupies a distance between 6.6--9.1~nm
  \citep{berger:high}. Because we use 10~kb resolution, we set $b^\text{max} = 91$~nm.
\end{enumerate}

\paragraph{Initialization:}
We create a population of 100 independently optimized structures by initializing
$\textbf{X}$ randomly from a standard normal distribution.

\paragraph{Measuring similarities between structures:}

To compare pairs of structures ($X$, $Y$) we used the standard RMSD measure:

\begin{equation}
\text{RMSD} = \text{min}_{X^*} \sqrt{\frac{1}{n} \sum_{i = 0}^n (x^*_i - y_i) ^ 2}
\end{equation}
where $X^*$ is obtained by translating and rotating $X$. To compare structures
of different scale (e.g., different $\beta$ values), we seek, in addition of the
translation and rotation factor, the scaling factor that minimizes the RMSD
between structures.

Another similarity measure we use to compare two structures is the average difference
of their pairwise distance matrices (at 10 kb resolution), which we denote by
\textit{distance difference}:
\begin{equation}
\text{distance difference} = \frac{1}{n(n - 1)/2} \sum_{i > j} |d^X_{ij} - d^Y_{ij}|
\end{equation}
where $d^X$ and $d^Y$ are the Euclidean distance matrices of the structures
$X$ and $Y$.

\paragraph{Clustering the population of structures:}

In order to see whether the structures fall into discrete groups, we computed
the RMSD between pairs of structures and performed hierarchical clustering on 
the resulting $100\times100$ distance matrix for each stage 
(Appendix Fig.~\ref{suppfig:CHindices}).

\paragraph{Choosing the parameter $\beta$:}
As noted above, the parameter $\beta$ controls the scaling of the inferred 3D
structure. A small value of $\beta$ will yield a structure with a very dense
center, and a large value of $\beta$ will push all beads against the nuclear
envelope. The literature suggests that chromatin should abut the nuclear
envelope \citep{weiner:3d}. Assuming the chromatin should also occupy the
center of the nucleus, we ran the entire optimization multiple times, and we
selected a value of $\beta$ that yields a chromatin density as close as
possible to a uniform distribution.

This procedure required that we estimate the density of chromatin at a
distance $\ell$ from the center of the nucleus.  To do so, we first
created an intermediate function
\[
f(\ell) = \sum_{i = 1}^N g\left(\ell - \sqrt{x_i ^2 + y_i^2 + z_i^2}\right),
\]
where $g(\cdot)$ is a Gaussian ($\mu = 0$, $\sigma = 10$~nm).  The
standard deviation $\sigma$ of the Gaussian corresponds to the
uncertainty of the position of each bead. The estimated density $D(\ell)$
was then computed as a generalized histogram, using discretized distance bins
$\ell_i$. To ensure that the volume was constant for each bin, the bin spacings
were defined as $\ell_i = i^{1 / 3} \ell_1$, where we chose $\ell_1 = \frac{r}{3}$.
We then normalized the histogram to sum to one.

Let $D_i$ be the density of bin $i$ and let $n_{\text{bins}}$ be the
number of bins. To select $\beta$, we defined the scoring function

\begin{equation}
\text{score} = \sqrt{\sum_{i = 1}^{n_\text{bins}} \left(D_i -
\frac{1}{n_\text{bins}}\right)^2},
\end{equation}
which corresponds to the mean squared error between the estimated density and
the expected density. The resulting density scores are shown in Appendix
Table~\ref{table:density}, with the minimal value for each stage in boldface.

\subsubsection{Eigenvalue decomposition and chromatin compartments}
\label{met:compartments}
To identify chromatin compartments, for each stage, we
carried out eigenvalue decomposition on the matrix of Euclidean
distances between locus pairs. For each chromosome we used the
intrachromosomal 3D distance matrix at a resolution of 10~kb, where
each 10~kb locus is represented by the 3D coordinate of its
midpoint. We then calculated the Spearman correlation between each
pair of rows of the 3D distance matrix and applied eigenvalue
decomposition (using the \emph{eig} function in MATLAB) to this
correlation matrix. The sign of the first eigenvector defined a
compartment assignment for each 10~kb locus at each stage. We also
aggregated all three stages and calculated a set of aggregate compartments
(Appendix Fig.~\ref{suppfig:perChrFigs}, fourth row of figures on each page)
which divided each chromosome into three main compartments (i.e.,
telomeric-centromeric-telomeric or left(L)-mid(M)-right(R)).

\subsubsection{Kernel canonical correlation analysis}
\label{met:kCCA}

We used an approach based on kernel canonical correlation analysis (kCCA)
\citep{bach:kernel,vert:graph,vert:extracting} to extract gene expression profiles that
simultaneously capture the variance of the gene expression data and exhibit
coherence with respect to the 3D structure.

Let $\mathcal{G}$ be the set of $n$ genes. Each gene $g\in\mathcal{G}$ is
characterized by its log expression profile $e(g) = \left( e_1(g), \ldots,
e_p(g)\right) \in \mathbb{R}^p$ at $p$ timepoints
and by its position $x(g) \in \mathbb{R}^3$
in 3D
space. We assume that the set of gene expression profiles is mean centered and
unit variance scaled, i.e., $\sum_{g\in\mathcal{G}}e_i(g) = 0$ and
$\frac{1}{|\mathcal{G}|}\sum_{g\in\mathcal{G}} e_i(g)^2 = 1$ for
$i=1,\ldots,p$.

Let $v\in\mathbb{R}^p$ be a direction in the expression profile space. To
assess whether $v$ is representative of the observed expression profiles, we computed
the percentage of variance explained among the gene expression profiles once
they are projected onto $v$, defined by
\begin{equation}\label{eq:variance}
V(v) = \frac{\sum_{g \in \mathcal{G}} \left(v^T e(g)\right)^2}{\|v\|^2}\,.
\end{equation}
The larger $V(v)$ is, the more $v$ explains the differences between gene
expression profiles, and the more likely $v$ is to correspond to some
biological event which influences the expression of many genes. $V(v)$ is, for
example, maximized by principal component analysis.

Instead of just asking the profile $v$ to capture variance among gene expression, we
simultaneously asked it to exhibit coherence with respect to the 3D structure.
For that purpose, we defined for every $f\in\mathbb{R}^n$ a function $S(f)$
that quantifies how smoothly $f$ varies in 3D. $f$ can be thought of as a
vector of scores, one score being assigned to each gene.  Because we know the
3D coordinates of each gene we can imagine $f$ as a set of scores in 3D.
Following a standard approach in kernel methods \citep{scholkopf:learning}, we
quantified the smoothness of $f$ with the function
\begin{equation}\label{eq:smoothness}
S(f) = \frac{f^\top K^{-1}_{3D} f}{||f||^2}\,,
\end{equation}
where $K_{3D}$ is the $n\times n$ matrix whose $(i,j)$ entry is the Gaussian
kernel between genes $i$ and $j$, namely, $\exp\left(-||x(i) - x(j)||^2 /
2\sigma^2\right)$. The smaller $S(f)$ is, the more smoothly $f$ is distributed
in 3D.

We then combined the ideas of capturing variance (Equation~\ref{eq:variance}) and being
smooth in 3D (Equation~\ref{eq:smoothness}) by designing a joint objective function
over $v$ and $f$ to ensure that (i) $v$ captures a lot of variance, (ii) $f$
is smooth in 3D, and (iii) $f$ is maximally correlated with the vector
$\left(v^\top e(g)\right)_{g\in\mathcal{G}}$. In words, we aimed to ensure that
genes which are positively correlated with $v$ (and those which are negatively
correlated) tend to be co-localized in 3D. We designed the function by following
the approach of \cite{bach:kernel}, who show that $v$ and $f$
can be found by solving a kCCA problem equivalent to the following generalized
eigenvalue problem:
\[
\left(\begin{array}{cc}0 & K_E K_{3D} \\ K_{3D} K_E & 0\end{array}\right)
\left(\begin{array}{c}\alpha \\ \beta\end{array}\right)
= \rho
\left(\begin{array}{cc}(K_E+\delta I)^2 & 0 \\ 0 & (K_{3D}+\delta I)^2\end{array}\right)
\left(\begin{array}{c}\alpha \\ \beta\end{array}\right)\,,
\]
where $K_{3D}$ is the $n \times n$ matrix whose $(i,j)$-th entry is $e(i)^\top
e(j)$, and $\delta$ is a small regularization parameter. Once we found the
generalized eigenvectors $(\alpha,\beta)^\top$, ranked by decreasing
eigenvalue $\rho$, we recovered a pair $(v,f)$ by $v = \sum_{g\in\mathcal{G}}
\alpha_g e(g)$ and $f=K_{3D} \beta$.

We computed the profiles for several values of $\sigma$ (0.01, 0.02, 0.05,
0.1) and $\delta$ (0.01, 0.02, 0.04, 0.06) and obtained highly correlated results
(correlation $>0.99$ for all pairs of profiles). Therefore, we chose
$\sigma = 0.01$ and $\delta = 0.02$ for the rest of the analysis.

\subsubsection{Gene set enrichment analysis}
\label{met:GSEA}
To detect set of genes highly or poorly correlated kCCA profiles, we apply gene
set enrichment analysis (GSEA) \citep{subramanian:gene}. Unlike a traditional
GO term enrichment analysis, this method takes as input a ranked list of genes
rather than a set of genes; hence, GSEA takes full advantage of the results of
the kCCA.  The procedure detects sets of genes enriched at the top or at the
bottom of the ranked list of genes. We applied GSEA to the ranked list of projections
of expression profiles on the first and second extracted profile. Corresponding
p-values were computed using $4,000$ permutations. We also used GSEA in our
comparison gene sets that are involved in contacts that are specific to either
one stage or two out of three stages.


\subsubsection{Volume exclusion model}
\label{met:volume-exclusion}

Following the methodology of \cite{tjong:physical}, we
constructed a population of three-dimensional structures by modeling
chromosomes as random configurations subject to the following
constraints:
\begin{enumerate}
\item Each chromosome is modeled as a series of $N$ beads spaced 3.2~kb apart,
  with consecutive beads restrained to be 30~nm apart.
\item Overlaps between beads are prevented by imposing a volume
  exclusion constraint for all pairs of beads.
\item All chromosomes lie within a spherical nucleus of a specified
  radius.
\item All centromeres are colocalized in a small sphere of radius
  50~nm abutting the nuclear envelope.
\item All telomeres are located within 50~nm of the nuclear envelope.
\end{enumerate}
We formulated an optimization problem that includes, in addition to the
constraints, a penalty term that accounts for chromatin stiffness by
placing an angular restraint between three consecutive beads:
\begin{equation}
\frac{1}{2} k_{\text{angle}} \sum^{N - 2}_{i = 1} \left( 1 - \frac{x_{i + 1} -
x_i}{\|x_{i + 1} - x_i\|} \cdot \frac{x_{i + 2} - x_{i + 1}}{\|x_{i + 2} -
  x_{i + 1}\|} \right)^2\,,
\end{equation}
where $x_i \in \mathbb{R}^3$ is the coordinate vector of bead $i$.
We used the Integrated Modeling Platform (IMP) \citep{bau:three-dimensional}
to generate 5,000 budding yeast structures with a nuclear radius of
1000~nm, and 5,000 {\em Plasmodium} structures for each of the three
stages with nuclear radii of 350~nm, 850~nm and 425~nm, respectively.

Following \cite{tjong:physical}, we used the population of structures to
generate a volume exclusion (VE) contact frequency matrix $C$,
considering that two beads are in contact when they are $\leq$45~nm apart.
The contact frequency matrix was then aggregated to a resolution of
32~kb and normalized following the ICE procedure as described above,
resulting in a contact frequency matrix
$c_{ij}^{VE}$ for $i,j=1,\ldots,N$ according to the VE model.

In order to compare the VE contact matrix to experimental Hi-C data,
we similarly computed the Hi-C contact count matrix at a resolution of
3.2~kb, aggregated it at 32~kb, and normalized the same way as the VE
contact frequency matrix to get a Hi-C contact matrix
$c_{ij}^{HIC}$ for $i,j=1,\ldots,N$.

We then compared both matrices by computing the row-based Pearson correlation
\citep{tjong:physical} defined as the average Pearson correlation between their rows.
\begin{equation}\label{eq:pearson}
\frac{1}{N} \sum_{i=1}^{N} \frac{
  N \sum_{j \neq i}^N c_{ij}^{\text{HIC}}{c}_{ij}^{\text{VE}} -
  \sum_{j \neq i}^N {c}_{ij}^{\text{HIC}} \sum_{j \neq i}^N {c}_{ij}^{\text{VE}}
}{
\sqrt{N \sum_{j \neq i }^N({c}_{ij}^{\text{HIC}})^2 - (\sum_{j \neq i}^N
{c}_{ij}^{\text{HIC}})^2}
\sqrt{N \sum_{j \neq i }^N({c}_{ij}^{\text{VE}})^2 - (\sum_{j \neq i}^N
{c}_{ij}^{\text{VE}})^2}}\,.
\end{equation}
Furthermore, we also computed a \emph{normalized} row-based Pearson correlation
between the matrices by replacing the counts $c^{VE}_{ij}$ and $c^{HIC}_{ij}$
in~(Equation~\ref{eq:pearson}) by their ratio to an expected count $c^E_{ij}$ that we
would expect if there was no structural information in the matrix, besides
the obvious decrease of contacts between loci at increasing genomic
distance. To estimate the
expected frequencies $c^E_{ij}$ used to define the ratios, we fit an
isotonic regression to the mapping between genomic distance and the average
contact frequency at this genomic distance. The isotonic regression
allows us
to fit a non-increasing mapping between genomic distance and contact
frequency, thus correcting the effect of enrichment of contact frequencies
at chromosome ends. This mapping allowed us to define $e_{ij}^E$ as the expected
count corresponding to the genomic distance between loci $i$ and $j$ in the
case of intrachromosomal contacts, and to the genome-wide average of
inter chromosomal counts in case of interchromosomal contacts.




% ----------------------- contents from here ------------------------

%figure exemple
%\figuremacroW{hicpro_fragments}{Classes of 3C products \label{ligationtypes}}{Following the Hi-C protocol, digested fragments are ligated together to generate 3C products. A valid 3C product is expected to involve two different restriction fragments. Once aligned on the genome, all strand combinations are expected according to the ligation orientation, and are expected to be observed in the same proportion. Reads aligned on the same restriction fragment are uninformative and can be classified as dangling end or self-circle products. Singleton, or pairs aligned on the same fragment and the same strand are rejected.}{0.8}

%table example
%\begin{table}[H]
%\centering
%\begin{tabular}{M{4cm}|M{2cm}M{2cm}M{3cm}}
% & {\bf IMR90 GSE35156} & {\bf IMR90 GSE35156} & {\bf IMR90\_CCL186 GSE63525} \\
%\hline
%\#Reads & 397 200 000 & 397 200 000 & 1 535 222 082 \\
%\#Input Files & 10 & 84 & 160 \\
%\#Jobs in parallel & 1 & 42 & 80 \\
%\#CPU per Job & 8 & 4 & 4 \\
%Max Memory (RAM) per Job & 7 Gb & 7 Gb & 7 Gb \\
%Wall Time & 27:18:58 & 01:30:00 & 07:52:10 \\
%-- Mapping & 11:53:52 & 00:26:00 & 05:50:05 \\
%-- Filtering & 14:41:46 & 00:14:00 & 00:05:00 \\
%-- Merge multiple Inputs and remove duplicates & 00:11:32 & 00:22:10 & 00:35:25 \\
%-- Contact maps builder & 00:05:51 & 00:04:23 & 00:25:09 \\
%-- ICE normalization & 00:26:25 & 00:23:03 & 02:01:26 \\
%\end{tabular}
%\caption[CPU]{\textbf{Details of CPU time} -  HiC-Pro was run on both IMR90 dataset in order to generate contact maps at resolution 20kb, 40kb, 150kb, 500kb and 1Mb. Contact maps at 5kb were also generated for the IMR90\_CCL186 dataset. CPU time for each step of the pipeline is reported.}
%\label{perf} 
%\end{table}
% ---------------------------------------------------------------------------
% ----------------------- end of thesis sub-document ------------------------
% ---------------------------------------------------------------------------
               

% this file is called up by thesis.tex
% content in this file will be fed into the main document

\chapter[Identification of centromere locations using Hi-C]{
Accurate identification of centromere locations in yeast genomes using Hi-C}
\label{chap:centurion}
\graphicspath{{4_centurion/figures/}}

\begin{work}

This chapter has been published in a slightly modified form in
\citep{varoquaux:accurate}, as joint work with Ivan Liachko, Josh Burton,
Ferhat Ay, Jay Shendure, Maitreya Dunham, Jean-Philippe Vert and Bill Noble.

\end{work}

% Résumé en francais

\begin{abstract}{Résumé}

Les centromères sont des éléments génomiques permettant la ségrégation
correcte des chromosomes lors de la division cellulaire. Malgré leur
importance dans le développement de la cellule et l'important effort de
recherche qui leur est dédié, la position des centromères chez la levure est
souvent, même de nos jours, difficile à inférer et est inconnue chez la
plupart des espèces. Récemment, le protocole de capture de conformation des
chromosome Hi-C, initialement développé pour étudier la structure 3D du
génome, a été reciblé pour diverses applications: séquençage
\textit{de novo} de génome, déconvolution d'échantillon métagénomique, et
inférence de la position des centromères chez la levure. Nous décrivons ici
une méthode, nommée Centurion, qui permet l'inférence conjointe de la position de
tous les centromères à la fois d'un organisme à partir de données Hi-C en exploitant la
propriété qu'ont les centromères de certains organismes à colocaliser dans le noyau.
Nous démontrons dans un premier temps la précision de notre
algorithme, en identifiant les centromères dans des données Hi-C à haute
couverture chez la levure de boulanger \textit{S. cerevisiae} et le parasite
responsable de la malaria \textit{P. falciparum}. Nous utilisons ensuite
Centurion pour prédire la position des centromères dans 14 autres espèces de
levure d'un échantillon métagénomique. Parmi tous les organismes que nous
étudions, Centurion prédit 89\% de centromères à moins de 5~kb de leur
position. Nous démontrons par ailleurs la robustesse de notre approche sur des
jeux de données à faible couverture. Finalement, nous inférons la position
des centromeres dans 6 espèces qui n'ont pour l'instant aucune annotation. Ces
résultats montrent que Centurion peut être utilisé pour l'identification de
centromères pour différentes espèces de levures, ainsi que pour d'autres
organismes.

\end{abstract}
% Abstract

\begin{abstract}{Abstract}
Centromeres are essential for proper chromosome segregation. Despite extensive
research, centromere locations in yeast genomes remain difficult to infer, and
in most species they are still unknown. Recently, the chromatin conformation
capture assay, Hi-C, has been re-purposed for diverse applications, including
\textit{de novo} genome assembly, deconvolution of metagenomic samples, and
inference of centromere locations. We describe a method, Centurion, that
jointly infers the locations of all centromeres in a single genome from Hi-C
data by exploiting the centromeres' tendency to cluster in 3D space. We first
demonstrate the accuracy of Centurion in identifying known centromere
locations from high coverage Hi-C data of budding yeast and a human malaria
parasite. We then use Centurion to infer centromere locations in 14 yeast
species. Across all microbes that we consider, Centurion predicts 89\% of
centromeres within 5~kb of their known locations. We also demonstrate the
robustness of the approach in datasets with low sequencing depth. Finally, we
predict centromere coordinates for six yeast species that currently lack
centromere annotations. These results show that Centurion can be used for
centromere identification for diverse species of yeast and possibly other
microorganisms.
\end{abstract}

\section{Introduction}

Centromeres are chromosomal regions whose function enables faithful chromosome
segregation via formation of the kinetochore \citep{Bloom:centromeric}. These
elements are also key regulators of genome stability \citep{Feng:centromere}
and replication timing \citep{koren:epigenetically, pohl:functional}. In
animal and plant genomes, centromeres are large heterochromatic zones, but
many yeast species have {\em point centromeres}, which are sequence
elements as small as 125~bp \citep{cottarel:125-base-pair}. The relative
simplicity of yeast centromeres has allowed their functional dissection, and
the abundance of sequenced yeast species has shed light on the evolution of
centromeric elements across hundreds of millions of years of evolution
\citep{Gordon:mechanisms}.

\enlargethispage{-65.1pt}

The Hemiascomycetes yeasts comprise a highly important taxon of model
organisms in genetics and genomics \citep{dujon:yeast,
hittinger:saccharomyces}, and some are crucial in biotechnology applications
such as recombinant protein expression \citep{boer:yeast}. Most yeast plasmid
expression systems are dependent on locating and identifying yeast centromeres
because they confer the property of stable segregation to episomal plasmids
\citep{murray:pedigree}. However, efforts to annotate yeast centromeres are
hindered by the extraordinary diversity among species \citep{Malik:major}.
Mapping centromeres in diverse species has been attempted, usually through
phylogenetic tools \citep{Gordon:mechanisms, souciet:comparative} or chromatin
immunoprecipitation \citep{Lefrancois:efficient}. However, both approaches
have drawbacks, the former due to the divergence of underlying functional
motifs and the latter due to non-specific signal. A method of mapping
centromeres that does not rely on evolutionary predictions or rare protein-DNA
interactions would therefore be useful for identifying centromeres in novel
species. These new centromere sequences could then be used, for example, to build new
plasmid-based strain engineering tools in species important for
research and biotechnology.

Chromosome conformation capture tools such as Hi-C and related protocols use
proximity ligation and massively parallel sequencing to probe the
three-dimensional architecture of chromosomes within the genome
\citep{lieberman-aiden:comprehensive, kalhor:genome, duan:three-dimensional}. Hi-C and
related techniques create a {\em contact map}, consisting of a matrix of
genome-wide interaction counts between pairs of loci. Contact maps have
recently been shown to contain long-range contiguity information: Hi-C has
been used in the scaffolding of \textit{de novo} genome assemblies
\citep{burton:chromosome, kaplan:high-throughput}, molecular haplotyping
\citep{selvaraj:whole-genome}, and metagenomic deconvolution
\citep{burton:species-level, beitel:strain}. These methods have also paved the
way for a more systematic analysis of genome architecture, including
long-range gene regulation and chromatin architecture \citep{nora:spatial,
dixon:topological, mizuguchi:cohesin-dependent}. These advances raise the possibility
that contact maps might be used to determine the location of
subchromosomal genomic structures such as centromeres and nucleoli.

A recent study attempted to map centromere locations using Hi-C contact
probability maps \citep{marie-nelly:filling}. This approach exploits the
strong architectural features of yeast genomes to determine centromere
positions and rDNA clusters in \textit{Saccharomyces cerevisiae},
\textit{Naumovozyma castellii}, \textit{Nuraishia capsulata}, and
\textit{Debaryomyces hansenii}. 
In yeasts, centromeres are tethered by microtubules to the spindle
pole body, leading to centromere clustering \citep{mizuguchi:cohesin-dependent}.
Similar clustering is also present in other organisms, such as the
parasite \textit{Plasmodium falciparum} and the plant
\textit{Arabidopsis thaliana} \citep{ay:three-dimensional,
  feng:genome-wide}.
The clustering of elements creates a
distinct peak of interactions between chromosomes in the \textit{trans} Hi-C
matrix, and an X-shape in the \textit{cis}-elements of the inter-chromosomal
contact counts pearson correlation matrix.
\citet{marie-nelly:filling} exploit this
X-shape structure in \textit{trans} contact counts correlation
matrices to first detect a 40~kb window containing each centromere.
In a subsequent step,
they carve out 40~kb-by-40~kb windows of contact counts
for each pair of centromeres and refine
the prediction by fitting a Gaussian on the sum of \textit{trans} elements of
these windows, a procedure similar to those used for single molecule
localization or high resolution microscopy \citep{Ober:localization}. However,
this method has several limitations. First, the procedure relies on the
correct pre-localization of candidate centromeres. This step fails when other
sequences also colocalize (for instance, rDNA sequences). Second, the last
step of the procedure collapses the data of several \textit{trans} interaction
windows into a 1D profile and calls the different centromeres independently
from each other, thus potentially losing some valuable information.

Here we propose a novel method, Centurion, that jointly
calls all centromeres in a genome-wide Hi-C contact map. The key idea is that
a joint optimization can effectively exploit the clustering of centromeres in
3D. We first compare our method to the one described by
Marie-Nelly \textit{et al}.\ on four publicly
available high-resolution Hi-C
contact maps (\textit{S.\ cerevisiae} \citep{duan:genome-wide} and three stages
of \textit{P.\ falciparum} \citep{ay:three-dimensional}). This comparison
demonstrates that Centurion infers centromere positions more accurately than
the previously published method. We then apply our method to Hi-C data from 14
diverse yeast species \citep{burton:species-level}, yielding high-resolution
centromere location predictions for each chromosome in each species. For the
eight species that already have centromere annotations available, our
predictions match very closely with the existing calls. For species with
as-yet uncharacterized centromeres, our predictions will serve as the basis
for targeted experimental validation and could be used to create new plasmid
tools in these yeasts. Our results suggest that Centurion has great potential
to identify the centromere locations of many yeasts for which standard
techniques have failed to date. Furthermore, we demonstrate that Centurion
works well even with very limited sequencing depth Hi-C libraries generated
from pooled samples, making it a practical as well as powerful tool to use on
single microorganisms and metagenomic mixtures. Centurion is freely available
as open source software at
\href{http://cbio.ensmp.fr/centurion}{http://cbio.ensmp.fr/centurion}.

\section{Method}

\begin{figure*}
\includegraphics[width=\linewidth]{figure_1.pdf}
\caption{\textbf{Outline of Centurion's computational workflow}
\textbf{1.} Paired-end Hi-C reads are mapped and filtered to produce
genome-wide contact maps (see Methods). 
\textbf{2.} Contact maps are normalized to correct for technical and
experimental biases \citep{imakaev:iterative}.
\textbf{3.} Peaks in marginalized \textit{trans} contact counts are identified
as candidate centromere locations.
\textbf{4.} If necessary, a heuristic reduces the number of centromere
candidates that will be used to initialize the joint optimization. 
\textbf{5.} A joint optimization procedure finds the best set of centromere
coordinates, one per chromosome, minimizing the squared distance between the
2D Gaussian fits and the observed \textit{trans} contact counts. 
\textbf{6.} For organisms with known centromere locations, the accuracy of
predicted centromere locations is evaluated; otherwise, the method provides \textit{de
novo} centromere calls. 
}
\label{fig:methods}
\end{figure*}



\subsection{Single organism Hi-C data}
We use Hi-C data gathered in two previous studies: an asynchronous budding yeast
(\textit{S. cerevisiae}) sample \citep{duan:three-dimensional} and three different stages of
the human malaria parasite \textit{P. falciparum} \citep{ay:three-dimensional}.
For the budding yeast Hi-C data we download and use the files HindIII + MspI
(intra and inter)
from \url{http://noble.gs.washington.edu/proj/yeast-architecture/sup.html}.
For the three stages of \textit{P. falciparum} we download and use the Hi-C raw contact counts
at 10~kb resolution from GEO archive (Accession codes: GSM1215592, GSM1215593, GSM1215594).

\subsection{Metagenomic Hi-C data}
For Hi-C data from metagenomic samples we use the two synthetic mixtures (M-Y, M-3D)
generated in \citep{burton:species-level}. We also perform additional
sequencing of the M-3D sample using two restriction enzymes that cut more frequently than
the 6-bp cutters HindIII and NcoI used in the original publication. We perform these
additional Hi-C experiments exactly as described in \citep{burton:species-level}
with the exception that we use Sau3AI (a 4-bp cutter that recognizes ``GATC'') and AflIII
(a 6-bp cutter that recognizes ``ACRYGT'') to fragment the DNA. We then combine the reads
from these two libraries (Sau3AI and AflIII) to produce
Hi-C contact maps.

We process the Hi-C libraries from these metagenomic samples in a
similar fashion to the 
Hi-C data from the above mentioned single organism samples, with the exception of two differences.
First, we map the reads to a meta-reference genome that concatenates the reference genomes
of all the organisms in the corresponding sample. This mapping strategy discards contacts
which cannot be uniquely assigned to a single organism, thereby reducing contamination
between contact maps. Second, because of the longer read lengths
for the metagenomic libraries compared to single organisms (80--101~bp versus 20--50~bp), we
post-process the non-mapped reads that contain a cleavage site for the restriction enzyme
used for the library generation, as previously described \citep{ay:identifying}. This post-processing
increases the number of informative contacts extracted from the metagenomic Hi-C libraries by
5-15\% depending on the read length and the cleavage site frequency. The resulting set of
informative contacts are processed further at appropriate resolution, as described below.

\subsection{Assembling the \textit{K. wickerhamii} genome}

Two input genome assemblies are used for creating the new \textit{K.
wickerhamii}
reference genome. The first is the publicly available \textit{K. wickerhamii} reference
genome originally sequenced by Baker \textit{et al}.~\cite{baker:extensive},
and the
second is the \textit{K. wickerhamii} associated cluster from
Burton \textit{et al}.~\cite{burton:species-level}. These assemblies are merged with CISA
\citep{lin:cisa} and then
merged using the mate-pair library from \citep{burton:species-level} using the
``scaffold'' command from IDBA \citep{peng:IDBA}. Hi-C reads are then aligned to
this assembly, and the seven scaffolds containing the 7 \textit{K. wickerhamii}
centromeres are identified. Lastly, this assembly is run through Lachesis
\citep{burton:chromosome}, with a restriction that the seven
centromere-containing scaffolds could not be merged.

\subsection{Data normalization}

Hi-C contact counts are subjected to many biases (GC-content, mappability,
etc) \citep{yaffe:probabilistic}. To correct for technical biases, we apply 
to the raw contact counts an iterative correction and eigenvector decomposition
(ICE) method proposed by \citet{imakaev:iterative},
based on the assumption
that all loci should interact equally. We then rescale the resulting matrix
such that the average normalized contact count is equal to the average raw
contact counts.

\subsection{Centromere calling}

We segment the full genome into $N$ windows of similar length ($N=611$ for
\textit{S. cerevisae} at 20~kb) and summarize the Hi-C data by the contact
count matrix $C\in\RR^{N \times N}$, where $C_{ij}$ is the normalized number
of physical interactions captured between loci in windows $i$ and $j$. 
For each window $i\in[1,N]$
we denote by $B(i) \in [1,L]$ the chromosome to which window $i$ belongs,
$L$ being the total number of chromosomes ($L=16$
for \textit{S. cerevisae}). We also denote by $x_i$ the genomic coordinate of
the center of the $i$-th window. Our objective is to infer the genomic
coordinates $p=(p_1, \ldots, p_L)$ of the centromeres of the $L$ chromosomes.
More precisely, centromeres usually consist of a sequence with a length ranging from several hundred
base pairs
for point centromeres to several thousand base pairs for regional centromeres.
In this work, we infer the mean position of these sequences.

Our main assumption is that, because centromeres colocalize in the nucleus, we
expect loci near centromeres in different chromosomes to be enriched in
Hi-C contacts.
To capture this enrichment, we model the contact counts between
windows $i$ and $j$ of different chromosomes $k$ and $l$ by a 2-D
Gaussian function centered on the corresponding centromeres $p_k$ and
$p_l$:
$$
a\exp \left(- \frac{(x_i - p_k)^2 + (x_j - p_l)^2)}{2\sigma^2}\right) + b\,,
$$
with parameters $a, b$ and $\sigma\geq 0$.
Then, denoting by $\mathcal{D}$ the set
of pairs of windows $(i,j)$ from different chromosomes with non-zero
counts, we jointly estimate the parameters $(a,b,\sigma)$ and the
positions of the $L$ centromeres by a least-squares fit of the Hi-C count data,
namely, by minimising in $a,b,\sigma\geq 0$ and $p=(p_1, \ldots, p_L)$ the
following objective function:
\begin{equation}\label{eq:opt}
\hspace*{-0.65cm}
\sum_{(i,j)\in\mathcal{D}} \left[ C_{ij} - a\exp \left(- \frac{(x_i - p_{B(i)})^2 + (x_j - p_{B(j)})^2)}{2\sigma^2}\right) - b \right]^2\,.
\end{equation}


Note that in this optimization, the position of each centromere is constrained
to be on its corresponding chromosome. Note also that for each non-zero entry
of the contact count matrix, we only fit the Gaussian
centered on the corresponding pair of loci. Thus, when the centromeres are
close to a chromosome boundary, we only fit a truncated
Gaussian.

\subsection{Initializing the optimization problem}

Because the optimization problem (\ref{eq:opt}) is non convex, the local minimum
found by the algorithm depends on the initialization of the parameters, in
particular of the centromeres' positions. We therefore need a heuristic
to initialize centromere positions. Because centromeres tend to interact
in \textit{trans} with other centromeres, a simple heuristic is to choose the position on each chromosome at the
center of the window with the largest total number of \textit{trans} contact
counts. However, we found that this heuristic was often not sufficient, because other loci besides
centromeres, such as
telomeres or rDNA clusters, can exhibit large numbers of \textit{trans} interactions. We therefore implemented another heuristic to
generate other good initializations and to explore more local minima. In
short, on each chromosome we detect a few local maxima (typically, two per
chromosome) of a smoothed \textit{trans} contact counts curve. We then initialize the
optimization by combining each choice of centromere location among the
candidates on each chromosome. If time constraints do not allow us to test all
such initializations (with 2 choices on 14 chromosomes, this corresponds to
$2^{14}=16384$ different initializations), then we can further reduce the
exploration of local minima by starting from the best candidate on each
chromosome (i.e., with the largest number of \textit{trans} contact counts),
optimizing the objective function from this initialization, and then moving to
other "nearby" local minima of the objective function by changing centromere
initialization to another candidate one centromere at a a time, until no
nearby local minimum is better than the one we have converged to.

A Python implementation of the proposed method is available at
\href{http://cbio.ensmp.fr/centurion}{http://cbio.ensmp.fr/centurion}.

\subsection{Measuring the performance}

To measure the performance of the centromere position prediction on datasets
for which we have the
ground truth, we compute the distance in base pairs between the
prediction $pred$ and the segment $(b, e)$ as follows:
$$
\max\left( (b-pred)_+ , (pred - e)_+\right)
$$
where $(u)_+$ is $u$ if $u\geq 0$ , $0$ otherwise.



\section{Results}


\subsection{Validating the method on \textit{S. cerevisiae} and \textit{P.
falciparum}}


\begin{figure*}
\includegraphics[width=\linewidth]{{figure_2}.pdf}
\caption{\textbf{Calling centromeres on \textit{P. falciparum} and \textit{S.
cerevisiae}}
\textbf{A}. Heatmap of the normalized \textit{trans} contact counts for
\textit{S. cerevisiae} Hi-C data at 40~kb overlaid with Centurion's
centromeres calls (black lines). The contact counts were smoothed with a
Gaussian filter ($\sigma =$ 40~kb) for visualization purposes. White lines
indicate chromosome boundaries.
\textbf{B.} Per chromosome errors of Centurion's centromere calls for
\textit{S. cerevisiae} using normalized (black) and raw (blue) Hi-C contact
maps at 40~kb resolution.
\textbf{C.} Heatmap of \textit{trans} contact counts for \textit{P. falciparum} trophozoite data at 40~kb
overlaid with Centurion's centromere calls (dashed black line) and ground
truth (red line) for chr 2, 3, 4 and 12.
\textbf{D.} Average errors of centromere calls for Centurion (black) and
\citet{marie-nelly:filling} method for \textit{S. cerevisiae} data from
\citet{duan:genome-wide} and the three stages of \textit{P. falciparum} when
both methods are initialized with the ground truth centromere coordinates.
}
\label{fig:high_resolution_results}
\end{figure*}



To evaluate the accuracy of our centromere prediction method, we first applied
it to two organisms with known centromere coordinates and available Hi-C data.
The first one is the widely studied budding yeast \textit{S. cerevisiae}. The
genome of \textit{S. cerevisiae} has 16 chromosomes and thus 16 centromeres,
all of which colocalize near the spindle pole body
\citep{jin:high-resolution}. All 32 telomeres of \textit{S. cerevisiae} tether
to the nuclear envelope. The repetitive ribosomal DNA of \textit{S. cerevisiae}
occurs on chromosome XII and is bundled into the nucleolus at the opposite
side of the nucleus from the spindle pole body \citep{venema:ribosome}. These
organizational principles constrain the chromosomes to fold into a distinct
configuration, known as the \textit{Rabl configuration}, which resembles a water lily shape
\citep{zimmer:principles}. The contacts between centromeres in \textit{S.
cerevisiae} chromosomes are known to result in a strong enrichment of
centromere-to-centromere Hi-C links \citep{duan:genome-wide}. We sought to
evaluate Centurion's ability to pinpoint the exact centromere locations
directly from a Hi-C contact map \citep{gotta:clustering}.

Using 40~kb-resolution Hi-C contact maps from Duan
\textit{et al}.~\cite{duan:genome-wide}
(Figure~\ref{fig:high_resolution_results}A and
\ref{fig:high_resolution_results}B), Centurion predicts centromere
coordinates with an average deviation of 11~kb from the known coordinates.
Notably, Centurion's Gaussian fitting procedure allows the centromere calls to
achieve finer resolution than is provided by the input contact maps. Using
20~kb resolution contact maps, the average deviation drops to 9~kb.
Furthermore, we observed that normalizing the contact maps
\citep{imakaev:iterative} yields substantially improved results, reducing the
average deviation to 2.5~kb for both the 20~kb and 40~kb resolution. We
investigated the differences in the prediction accuracy of our method among
the 16 different chromosomes. While our predictions were within 1~kb of the
known centromere coordinates for the chromosomes V, VI, IX, XIII and XV
(respectively, 59~bp, 235~bp, 111~bp, 289~bp and 163~bp away), they were more
than 5~kb away for chromosomes III, VII and XII (respectively, 5011~bp, 5327~bp
and 6457~bp away).
While the cause of this fluctuation of accuracy is not yet known,
chromosomes III and XII house the only major blocks of heterochromatin
in this genome other than telomeres
 (the silent mating loci and rDNA,
respectively), suggesting that linked heterochromatinized loci may interfere
with accurate centromere prediction.

We then applied our method to a second species, the malaria parasite
\textit{P. falciparum}, which is responsible for the most virulent
form of malaria \citep{who:malaria}. We recently used Hi-C to provide a global
picture of the genome architecture of \textit{P. falciparum} at three stages
(ring, schizont and trophozoite) throughout its erythrocytic life cycle in
human blood \citep{ay:three-dimensional}. Centromere coordinates for
\textit{P. falciparum} were only identified systematically relatively recently
\citep{hoeijmakers:plasmodium}. We applied Centurion to the contact maps of
each of these three stages at 10~kb, 20~kb and 40~kb resolutions
(Appendix Fig~\ref{suppfig:error_diff_res_pf}).
As with \textit{S. cerevisiae}, we observe some
variation in the accuracies of our predictions for each chromosome. However,
overall, the accuracy is very high. At 10~kb resolution, for example,
Centurion's centromere predictions fall within the known centromere location
for all 14 chromosomes during the schizont stage, 13 out of 14 for the ring stage and for
11 out of 14 chromosomes in the trophozoite stage. Overall, across the three
different stages Centurion correctly localizes 90\%, 64\%, and 45\% of
centromeres at 10~kb, 20~kb and 40~kb resolution, respectively. For the
incorrectly called centromeres, the average distance from Centurion's
prediction and the edge of the centromere is 495~bp, 1308~bp, and 2319~bp,
respectively.

We next sought to understand the sources of error in our predictions. Looking
closely at the contact counts matrices in the neighborhood of centromeres for
which the prediction is not accurate, we observed that loci in proximity to
centromeres seem to exhibit unusually sparse interaction counts. For example,
Figure~\ref{fig:high_resolution_results}C shows that in the trophozoite stage,
the centromere of chr~1 is close to a chromosome boundary and the chr~4
centromere is close to a locus with few interacting bins. The latter case
leads to bias from the normalization procedure because the few nonzero entries
in this sparse region are over-corrected. We also investigated whether the
accuracy of our prediction varies by life cycle stage and matrix resolution
(Appendix Fig~\ref{suppfig:raw_vs_normed_pf}). Many chromosomes are given
consistently poor centromere calls across all life cycle stages and at all
resolutions, corroborating the observations above that the predictions tend to
be influenced by biases intrinsic to the genome around those centromeres, such
as mappability or GC content.

We next compared the accuracy of our predictions to that of a previously
published method \citep{marie-nelly:filling}. Marie-Nelly
\textit{et al}.\
method often works well for identifying centromeres using Hi-C libraries with
very high sequencing depth; however, when Hi-C sequencing depth is limited or
when loci other than centromeres strongly cluster,
the first step of the procedure, called ``pre-localization,''
sometimes fails to identify the correct fixed size window in which the
centromeres reside.
We hypothesized that
the joint centromere calling by Centurion, which leverages data from all
chromosomes at once, might alleviate this instability. To test this
hypothesis, we applied the Marie-Nelly \textit{et al}.\
method to the same four
datasets (one \textit{S. cerevisiae} and three \textit{P. falciparum})
described above. As shown in Appendix 
Figure~\ref{suppfig:marie_nelly_vs_us_HR}, in each of these four
datasets Centurion identifies centromeres with better accuracy than the
Marie-Nelly \textit{et al}.\ method.
For instance, the colocalization of rDNA
clusters and virulence genes in \textit{P. falciparum} drastically changes the pattern
of the correlation matrix used by Marie-Nelly
\textit{et al}.\ to pre-localize
their centromere calls, thus confounding their prediction (Appendix
Fig.~\ref{suppfig:pearson_corr_pf}).

We also asked whether the improvement of Centurion over
Marie-Nelly \textit{et al}.\ method is due to the
initialization step, or
due to different objective functions used by each method. We initialized both
optimization problems with the ground truth and computed the resulting error.
Our results (Figure~\ref{fig:high_resolution_results}D) showed that
Centurion's error is still between 4- and 10- fold lower, thus demonstrating
the benefit of jointly calling centromeres.

\subsection{Resolution, sequencing depth and prediction accuracy}


\begin{figure*}
\includegraphics[width=\linewidth]{{figure_3}.pdf}
\caption{\textbf{Impact of Hi-C library sequencing depth on the stability of
the centromere calls}
Average variance of the results of Centurion on 500 generated datasets
obtained by downsampling the raw contact counts to the desired coverage. 
}
\label{fig:downsampled}
\end{figure*}



To assess the stability of our predictions, we simulated 500 bootstrapped data
sets of \textit{S. cerevisiae} and of each stage of \textit{P. falciparum}
with an expected total number of reads equal to the contact counts matrices.
These bootstrapped samples were obtained by drawing a contact count for each
pair of loci i and j from a Poisson distribution of intensity $c_{ij}$. We
then ran the optimization process on the bootstrapped data sets, starting with
initial values randomly placed within 40 kb of the centromere calls from our
optimization in Appendix Tables~\ref{supptable:sc_results},
\ref{supptable:rings_results}, \ref{supptable:trophs_results} and
\ref{supptable:schizonts_results}. Our results show that the optimization is
very stable (average variance of 25~bp for ring, 6~bp for schizont and 12~bp
for trophozoite), suggesting that the stochastic sampling of the sequencing
procedure does not significantly affect centromere predictions.

We then sought to investigate the extent to which the matrix resolution and
sequencing depth affect the accuracy of Centurion's predictions. As already
seen in Appendix Figures~\ref{suppfig:error_diff_res_sc} and
\ref{suppfig:error_diff_res_pf}, different species give different
results: for \textit{S. cerevisiae}, increasing the matrix resolution to 10~kb
results in lowered accuracy of centromere calls, while in \textit{P.
falciparum} the call quality improves slightly. We speculated that our ability
to call centromeres in a given species at a given resolution may depend on the
choice of restriction enzyme, the sequencing depth, and the resolution of the
contact map.

We next evaluated the effect of depth of sequence coverage on the quality of
our centromere predictions. We generated 500 low-coverage datasets by randomly
downsampling the raw contact counts. We then ran the optimization process on
these downsampled datasets, initializing with perturbed calls as before. We
observe that the low coverage centromere calls remain highly stable and
accurate. As illustrated in Figure~\ref{fig:downsampled}, results across all
data sets only begin to degrade when downsampling to less than 10\% of the
total number of reads, which corresponds to less than one count per bin on
average. Centurion is thus applicable to call centromeres at low cost or for
low-abundance species in metagenomic samples.

\subsection{Centromere calls on a metagenomic dataset}


\begin{figure*}
\includegraphics[width=\linewidth]{{figure_4}.pdf}
\caption{\textbf{Centromere calling on a metagenomic sample} \textbf{A.}
Heatmap of the \textit{trans} contact counts for \textit{K.\ wickerhamii}
overlaid with \textit{de novo}
centromere calls (black lines). The contact counts were smoothed with a
Gaussian filter ($\sigma = 40~kb$) for visualization purposes. White lines
indicate chromosome boundaries.
\textbf{B.} Box plots indicating the error (in kb) for each chromosome in
Centurion's centromere calls for eight yeasts with known centromere
coordinates from the combined metagenomic Hi-C samples M-3D and M-Y of
\citep{burton:species-level} on the 20~kb contact count matrices.}
\label{fig:metagenomic_results}
\end{figure*}


We next sought to call centromeres in several species simultaneously by
combining Centurion with metagenomic Hi-C libraries. We previously
\citep{burton:species-level} generated two Hi-C datasets from synthetic
mixtures: one containing 16 yeast strains (including four strains of
\textit{S. cerevisiae}), and one containing a mixture of 8 yeasts and 10
prokaryotic species. The two samples contain a total of 19 yeast species, some
of which are much better characterized than others: centromere positions are
already known for eight species (\textit{K. lactis}, \textit{L. kluyveri},
\textit{L. thermotolerans}, \textit{S. cerevisiae}, \textit{S. kudriavzevii},
\textit{S. mikatae}, \textit{S. pombe}, \textit{S. rouxii}) and partially for
one more (\textit{S. bayanus}) \citep{scannell:awesome, souciet:comparative,
mcdowall:pombase, dujon:genome}.

We aligned the reads from the metagenomic Hi-C datasets to these yeast
species' reference genomes (see Appendix~\ref{suppnotes:initialization}). The
quality of the individual species datasets differ greatly because the
organisms vary in abundance in the metagenomic samples, and because many
sequences are shared nearly identically between organisms, making the number
of uniquely mappable reads for each organism range between 109~k for one of
the \textit{S. cerevisiae} strains to 26~M for the bacteria \textit{V.
fischeri}. Consequently, the sparsity of the matrices is variable (Appendix
Tables~\ref{supptable:m2_qc} and \ref{supptable:my_qc}). Furthermore,
some contact counts matrices include at least
one interaction count for more than 99\% of all possible locus pairs, whereas
other matrices are below 5\%. Similarly, in the 40~kb matrices, the average
number of interchromosomal contact counts per bin varies from less than 0.004
to more than 200. In particular, the matrices for the four \textit{S.
cerevisiae} strains are very sparse: the reference genomes of the four strains
are very similar to one another; thus, we are not able to map reads uniquely.
We therefore discarded those strains from our analysis, as well as organisms
with incomplete reference genomes. We applied Centurion to the remaining 14
yeasts (\textit{E. gossypii}, \textit{K. lactis}, \textit{K. wickerhamii},
\textit{L. kluyveri}, \textit{L. waltii}, \textit{S. bayanus}, \textit{S.
kudriavzevii}, \textit{S. mikatae}, \textit{S. paradoxus}, \textit{S.
stipitis}, \textit{P. pastoris}, \textit{L. thermotolerans}, \textit{S.
pombe}, \textit{S. rouxii}) on both 20~kb and 40~kb contact maps.

Across these 14 species Centurion performs well, both on high-coverage
datasets (\textit{K. lactis}, \textit{L. kluyveri}, \textit{S. bayanus}) and
low-coverage datasets (\textit{S. mikatae}), at 20~kb and 40~kb, finding
centromeres at an average deviation from the ground truth of 10~kbp
(Figure~\ref{fig:metagenomic_results}B and Appendix
Figure~\ref{suppfig:metagenomic_sample_40}).
Given this success with yeasts with
known centromere positions, we next made \textit{de novo} centromere calls for the
other 6 yeast species present in the metagenomic samples. These regions,
visualized in Appendix Figures~\ref{suppfig:KL_calls}, \ref{suppfig:LK_calls},
\ref{suppfig:SB_calls}, \ref{suppfig:SM_calls}, \ref{suppfig:SK_calls},
\ref{suppfig:LT_calls}, \ref{suppfig:zP_calls}, \ref{suppfig:ZR_calls}, are strong candidates for
experimental validation by other approaches. One feature that is shared by
centromeres across all studied fungi is that they reside in regions of early
replication timing \citep{koren:epigenetically, pohl:functional}. Thus if our
centromere calls lie in regions of advanced replication timing in a species
for which replication timing has been profiled but centromeres have not
yet been identified, this data could be used to assess the validity of
our predictions. Accordingly, we
overlaid the positions of our centromere calls in \textit{P. pastoris}, where
replication has been recently profiled \citep{liachko:gc-rich}. In all four
chromosomes, \textit{P.\ pastoris} centromere predictions lay in regions of
early replication timing (Appendix Fig.~\ref{suppfig:ppas_timing}),
lending support to our predictions.

\subsection{The effect of the choice of restriction enzyme}

In addition to the resolution of our contact matrices, the underlying
resolution of the Hi-C data itself may limit the accuracy of our predictions.
Hi-C reads can only occur near the recognition site of the restriction enzyme
used in the Hi-C assay; indeed, the best resolution we can hope to achieve is
a matrix in which each corresponds to one restriction enzyme fragment. Some
restriction enzymes cut much more frequently than others. Thus, we speculated
that a Hi-C experiment using enzymes that cut more frequently might yield more
accurate results than an experiment using less frequently cutting enzymes.

To address this question, we compare the accuracy of centromere calling from
two Hi-C libraries created from a single metagenomic sample using different
combinations of restriction enzymes. The first library was created using the
two 6~bp-cutters, HindIII and NcoI. The second library uses Sau3AI, which has a
4~bp recognition site, and AflIII, which has a 6~bp recognition site with two
degenerate sites, making it effectively a 5~bp cutter. Digestion with
HindIII/NcoI yields a total of 8324 restriction fragments, whereas digestion
with Sau3AI/AflIII yields 42359 restriction fragments. We corrected for the
difference in Hi-C sequencing depth between Sau3AI/AflIII and the NcoI/HindIII
libraries by generating downsampled datasets with an equal number of reads
from each sequencing library. We then normalized the datasets and applied
Centurion. The sample includes three species for which we possess the ground
truth centromere locations, only one of which (\textit{L. thermotolerans})
had enough reads in both the NcolI/HindIII (~63000 reads) and the pooled
Sau3AI/AflIII (~55000 reads) datasets to correctly call the centromeres. The
error on the downsampled Sau3AI/AflIII datasets (8~kbp) was on average half as
large as the error on the the NoclI/HindIII datasets (16~kbp). Thus, we
conclude that using a restriction enzyme with more frequent cutting sites
enables more precise centromere calls at fine scales.

\section{Discussion}

While centromeres are a fundamental element in the biology of genomes, their
identification in diverse species has proven difficult due to sequence
divergence and limitations of available tools. In this work, we have
developed a novel method, Centurion, that uses centromere colocalization and
the pattern it creates in Hi-C contact maps to jointly call centromeres for
all chromosomes of an organism.
We first established the feasibility of this approach by demonstrating
that Centurion accurately calls regional centromeres on the parasite
\textit{P.\ falciparum} and the yeast \textit{S.\ pombe} as well as
point centromeres on several other yeasts with known centromere
coordinates.
For the species with high depth Hi-C sequencing, Centurion often identified
centromeres within 1~kb of the actual coordinates (41 times out of 58 for
three stages of \textit{P. falciparum} and \textit{S. cerevisiae} data). We
then used Centurion to infer centromeres of multiple yeast species (8 with
known, 6 with unknown centromere coordinates) from two metagenomic Hi-C
samples. Our results showed that Centurion still accurately identifies
centromere coordinates from samples with only limited sequencing depth. Thus,
Centurion can be used to accurately and efficiently identify centromere
locations in yeast species.

The task of centromere identification from Hi-C data has been attempted
recently by others \citep{marie-nelly:filling}. Centurion offers a few key
differences compared to the previous approach. The first difference is in the
pre-localization of candidate centromeres.
Marie-Nelly \textit{et al}.'s
method uses only the \textit{cis} Pearson correlation information independently per
chromosome to identify the initial candidates. However, the pattern created by
centromeres in the Pearson correlation matrix can be very similar to the patterns
generated by other genomic elements such as rDNA coding regions or by specific
gene clusters (e.g., virulence genes in \textit{P. falciparum}). 
Because
Marie-Nelly \textit{et al}.'s method restricts
the further search for the best
centromere coordinate to only the candidates from the pre-localization step,
an inaccurate candidate (e.g., an rDNA region instead of a centromere) will
prevent the method from finding the correct centromere location. Centurion, on
the other hand, utilizes \textit{trans} contact information jointly across all
chromosomes for its pre-localization step. Furthermore, Centurion allows
multiple candidates per chromosome during the second step of the optimization,
thereby leaving room for correcting potential errors in the pre-localization
step. The second difference between the two methods is in how they use the
submatrices that correspond to \textit{trans} contact maps flanking the pairs
of candidate centromeres from the pre-localization step. For an organism with
N chromosomes, Marie-Nelly \textit{et al}.'s
method carves out the N-1
\textit{trans} submatrices for each chromosome, sums these N-1 matrices and
then collapses the sum into a 1D vector of row/column sums. Then,
independently for each chromosome, the method fits a Gaussian to this 1D
vector, and the resulting peak corresponds to the predicted centromere
location. In this procedure, both the summation of N-1 matrices and the
collapsing of the resulting matrix into a 1D vector of sums result in loss of
important information embedded in 2D maps. Furthermore, performing the
Gaussian fit separately for each chromosome does not fully take into account
the joint co-localization of the other N-1 centromeres. To address these
issues, Centurion infers a 2D Gaussian fit that best explains the observed
\textit{trans} contact counts, jointly optimizing these 2D fits for all pairs
of centromeres. Both of these improvements in the pre-localization and the
optimization steps allow Centurion to perform better specifically for the
cases with limited sequencing depth.
Our approach could be improved in several respects. First, better
modeling of zero contact counts may improve inference for organisms
with many repeated sequences in the peri-centromeric regions, or data
sets with low sequencing depth. Second, one could model contact counts
as a Gaussian distribution centered on the pairs of centromere
locations. Maximising the log likelihood of such a model 
might yield improved performance.
Last, as described here, our method requires reference genomes for the
metagenomic samples. It would be possible to first build reference
genomes directly from the Hi-C data, using methods like Lachesis
\citep{burton:chromosome} or Graal \citep{marie-nelly:high-quality},
and then infer centromeres locations using the inferred
references. However, the inherent structure of Hi-C contact counts for
organisms with colocalizing centromeres will likely present a
challenge for these methods because pericentromeric sequences on
different chromosomes are likely to appear to be adjacent to one
another.

Finally, our new centromere predictions have practical applications.
Autonomously replicating plasmids and artificial chromosomes are useful tools
for research and strain engineering \citep{boer:yeast}. Identification of
centromeres in new species will facilitate building such constructs over an
expanded species range. \textit{P. pastoris}, for example, is a common
industrial chassis \citep{cregg:expression}, but existing plasmid tools in the
species have elevated loss rates \citep{liachko:autonomously} that could be
stabilized by addition of a centromere. Many of our centromere calls were
accurate to $<1$~kb, making experimental validation possible.

\section{Funding}

This work was supported by the European Research Council [SMAC-ERC-280032 to
J-P.V., N.V.]; the European Commission [HEALTH-F5-2012-305626 to J-P.V.,
N.V.]; the French National Research Agency [ ANR-11-BINF-0001 to J-P.V.,
N.V.]; the National Institute of Health/National Human Genome Research
Institute [HG006283 to J.S., T32HG000035 to J.N.B.];
National Institute of Health/National Institute of General Medical Sciences [P41 GM103533 to I.L.,
M.J.D., W.S.N.; R01AI106775 to F. A., W.S.N.]; National Science Foundation
[1243710 to I.L., M.J.D.]. M.J.D. is a Rita Allen Foundation Scholar and a Senior
Fellow in the Genetic Networks program at the Canadian Institute for Advanced
Research.

\section{Acknowledgements}

We thank Celia Payen for providing the yeast centromere annotations,
St\'{e}fan van der Walt for advice on peak detection algorithms, Fabrice
Varoquaux for help on understanding the specificity of \textit{A. thaliana}
genome
and Chlo\'{e}
Azencott for helpful comments on the manuscript.


        

\include{8_software/chapter8}        

% --------------------------------------------------------------
%:                  BACK MATTER: appendices, refs,..
% --------------------------------------------------------------

% the back matter: appendix and references close the thesis


%: ----------------------- bibliography ------------------------

% The section below defines how references are listed and formatted
% The default below is 2 columns, small font, complete author names.
% Entries are also linked back to the page number in the text and to external URL if provided in the BibTex file.

% PhDbiblio-url2 = names small caps, title bold & hyperlinked, link to page 
\begin{multicols}{2} % \begin{multicols}{ # columns}[ header text][ space]
\begin{tiny} % tiny(5) < scriptsize(7) < footnotesize(8) < small (9)

%\bibliographystyle{Latex/Classes/PhDbiblio-url2} % Title is link if provided
\renewcommand{\bibname}{References} % changes the header; default: Bibliography

\bibliographystyle{plainnat}
\bibliography{refs} 

\end{tiny}
\end{multicols}

% --------------------------------------------------------------
% Various bibliography styles exit. Replace above style as desired.

% in-text refs: (1) (1; 2)
% ref list: alphabetical; author(s) in small caps; initials last name; page(s)
%\bibliographystyle{Latex/Classes/PhDbiblio-case} % title forced lower case
%\bibliographystyle{Latex/Classes/PhDbiblio-bold} % title as in bibtex but bold
%\bibliographystyle{Latex/Classes/PhDbiblio-url} % bold + www link if provided

%\bibliographystyle{Latex/Classes/jmb} % calls style file jmb.bst
% in-text refs: author (year) without brackets
% ref list: alphabetical; author(s) in normal font; last name, initials; page(s)

% in-text refs: author (year) without brackets
% (this works with package natbib)


% --------------------------------------------------------------

% according to Dresden med fac summary has to be at the end
%
% Thesis Abstract -----------------------------------------------------


%\begin{abstractslong}    %uncommenting this line, gives a different abstract heading
\begin{abstracts}        %this creates the heading for the abstract page

Here is the story of my life ...

\end{abstracts}
%\end{abstractlongs}


% ---------------------------------------------------------------------- 


%: Declaration of originality
%\include{9_backmatter/declaration}



\end{document}
